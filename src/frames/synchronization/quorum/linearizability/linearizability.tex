% SPDX-License-Identifier: CC-BY-SA-4.0
% Author: Matthieu Perrin
% Part: 
% Section: 
% Sub-section: 
% Frame: 

\begingroup

\SetKwFunction{Acheter}{acheter}

\begin{frame}{Linéarisabilité}

  Soit $M = \langle C, R, Q, q_0, \tau, \rho \rangle$ une machine à états. 
  
  \begin{block}{Définition -- Linéarisabilité}
    Une exécution $H$ est \structure{linéarisable} pour $M$ s'il existe une exécution séquentielle\\[-2mm]
    $$
    H_S = q_0 \xrightarrow{c_1 \rightarrow r_1} q_1 \xrightarrow{c_2 \rightarrow r_2} q_2 \xrightarrow{\dots} \dots
    $$

    \vspace{-2mm} telle que :
    \begin{description}
    \item[Sémantique :] pour tout $i\ge 1$, $\tau(q_{i-1}, c_{i}) = q_{i}$ et $\rho(q_{i-1}, c_{i}) = r_{i}$.
    \item[Opérations :] $H$ et $H_S$ contiennent les mêmes opérations $o_i = c_i \rightarrow r_i$ ;
    \item[Temps réel :] si l'opération $o_i$ se termine avant que $o_j$ ne commence dans $H$,\\
      alors $o_i$ précède $o_j$ dans $H_S$ ;
    \end{description}
    
    \centering
    \begin{tikzpicture}[x=10mm,y=7mm]
      \draw[densely dotted] (0,3) node[left]{$Q$}   -- (9,3);
      \draw[process]        (0,2) node[left]{$p_1$} to (9,2);
      \draw[process]        (0,1) node[left]{$p_2$} to (9,1);
      \draw[process]        (0,0) node[left]{$p_3$} to (9,0);

      \node[alert,     operation] (o1) at (2,1) {$\Acheter(a)\rightarrow\xmark$};
      \node[example,   operation] (o2) at (4,0) {$\Acheter(a)\rightarrow\cmark$};
      \node[structure, operation] (o3) at (6,2) {$\Acheter(a)\rightarrow\xmark$};

      \scriptsize
      \node[below right] at (0,3) {$\{a, b\}$};
      \draw[alert,     lin point] (o1.east) -- +(0,2) node[below right] {$\{b\}$};
      \draw[example,   lin point] (o2.west) -- +(0,3) node[below right] {$\{b\}$};
      \draw[structure, lin point] (o3.west) -- +(0,1) node[below right] {$\{b\}$};
    \end{tikzpicture}

  \end{block}

  \on[bottom=-2mm, text]{
    \begin{citing}
    \item[PMJ16] M. Herlihy, J. Wing. \emph{Linearizability: A correctness condition for concurrent objects}. TOPLAS. 1990
    \end{citing}
  }  

\end{frame}

\endgroup
\endinput
