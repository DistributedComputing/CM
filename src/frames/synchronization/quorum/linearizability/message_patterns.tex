% SPDX-License-Identifier: CC-BY-SA-4.0
% Author: Matthieu Perrin
% Part: 
% Section: 
% Sub-section: 
% Frame: 

\begingroup

\SetKwFunction{Read}{read}
\SetKwFunction{Write}{write}
\SetKwData{X}{x}
\SetKwData{Y}{y}

\begin{frame}{Motifs en messages et en mémoire}

  \on[top]{
    Deux processus $p_1$ et $p_2$ partagent de l'information de manière concurrente
  }

  \onBlock[top=5mm]{Motifs en passage de messages}{}

  \on[y=5mm, x=-38mm]{
    \begin{tikzpicture}[x=8mm, y=7mm]
      \draw[process] (0,1) node[left]{$p_1$} to (4,1);
      \draw[process] (0,0) node[left]{$p_2$} to (4,0);

      \draw[message, example]   (1,1) node[above left]{$m_1$} to[bend left=50]  (2,1);
      \draw[message, example]   (1,1)                         --                (3,0);
      \draw[message, structure] (1,0) node[below left]{$m_2$} to[bend right=50] (2,0);
      \draw[message, structure] (1,0)                         --                (3,1);

      \draw (2,-1) node{MP1};
    \end{tikzpicture}
  }
  \on[y=5mm]{
    \begin{tikzpicture}[x=8mm, y=7mm]
      \draw[process] (0,1) node[left]{$p_1$} to (4,1);
      \draw[process] (0,0) node[left]{$p_2$} to (4,0);

      \draw[message, example]   (1,1) node[above left]{$m_1$} to[bend left=50]  (2,1);
      \draw[message, example]   (1,1)                         --                (2,0);
      \draw[message, structure] (1,0) node[below left]{$m_2$} to[bend right=50] (3,0);
      \draw[message, structure] (1,0)                         --                (3,1);

      \draw (2,-1) node{MP2};
    \end{tikzpicture}
  }
  \on[y=5mm, x=38mm]{
    \begin{tikzpicture}[x=8mm, y=7mm]
      \draw[process] (0,1) node[left]{$p_1$} to (4,1);
      \draw[process] (0,0) node[left]{$p_2$} to (4,0);

      \draw[message, example]   (1,1) node[above left]{$m_1$} to[bend left=50]  (3,1);
      \draw[message, example]   (1,1)                         --                (2,0);
      \draw[message, structure] (1,0) node[below left]{$m_2$} to[bend right=50] (3,0);
      \draw[message, structure] (1,0)                         --                (2,1);

      \draw (2,-1) node{MP3};
    \end{tikzpicture}
  }


  \onBlock<2>[y=-13mm]{Motifs en mémoire partagée}{}

  \on<2>[y=-29mm, x=-38mm]{
    \begin{tikzpicture}[x=8mm, y=7mm]
      \draw (2,-1) node{RW1};
      
      \draw[process] (0,1) node[left]{$p_1$} to (4,1);
      \draw[process] (0,0) node[left]{$p_2$} to (4,0);

      \scriptsize
      \node[example,   operation, inner sep=1pt] at (1  ,1) {$\X.\Write(1)$};
      \node[structure, operation, inner sep=1pt] at (1  ,0) {$\Y.\Write(1)$};
      \node[structure, operation, inner sep=1pt] at (2.8,1) {$\Y.\Read() \rightarrow 0$};
      \node[example,   operation, inner sep=1pt] at (2.8,0) {$\X.\Read() \rightarrow 0$};
      
      \draw[alert, thick] (0,1.25) -- (4,-.25);
      \draw[alert, thick] (4,1.25) -- (0,-.25);
    \end{tikzpicture}
  }
  
  \on<2>[y=-29mm]{
    \begin{tikzpicture}[x=8mm, y=7mm]
      \draw (2,-1) node{RW2};
      
      \draw[process] (0,1) node[left]{$p_1$} to (4,1);
      \draw[process] (0,0) node[left]{$p_2$} to (4,0);

      \scriptsize
      \node[example,   operation, inner sep=1pt] at (1  ,1) {$\X.\Write(1)$};
      \node[structure, operation, inner sep=1pt] at (1  ,0) {$\Y.\Write(1)$};
      \node[structure, operation, inner sep=1pt] at (2.8,1) {$\Y.\Read() \rightarrow 0$};
      \node[example,   operation, inner sep=1pt] at (2.8,0) {$\X.\Read() \rightarrow 1$};
    \end{tikzpicture}
  }
  
  \on<2>[y=-29mm, x=38mm]{
    \begin{tikzpicture}[x=8mm, y=7mm]
      \draw (2,-1) node{RW3};
      
      \draw[process] (0,1) node[left]{$p_1$} to (4,1);
      \draw[process] (0,0) node[left]{$p_2$} to (4,0);

      \scriptsize
      \node[example,   operation, inner sep=1pt] at (1  ,1) {$\X.\Write(1)$};
      \node[structure, operation, inner sep=1pt] at (1  ,0) {$\Y.\Write(1)$};
      \node[structure, operation, inner sep=1pt] at (2.8,1) {$\Y.\Read() \rightarrow 1$};
      \node[example,   operation, inner sep=1pt] at (2.8,0) {$\X.\Read() \rightarrow 1$};
    \end{tikzpicture}
  }

\end{frame}

\endgroup
\endinput
