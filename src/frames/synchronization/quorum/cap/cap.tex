% SPDX-License-Identifier: CC-BY-SA-4.0
% Author: Matthieu Perrin
% Part: 
% Section: 
% Sub-section: 
% Frame: 

\begingroup

\SetKwFunction{Read}{read}
\SetKwFunction{Write}{write}

\begin{frame}{Le théorème CAP}

  \begin{block}{Conjecture -- Brewer}
    Il est impossible d'implémenter une base de données distribuée vérifiant :
    \begin{description}[Partition tolerance :]
    \item[Consistency :] toutes les répliques voient les \alert{mêmes données}
    \item[Availability :] chaque requête \alert{finit par} recevoir une réponse
    \item[Partition tolerance :] le système fonctionne même si des \alert{partitions du réseau} ne peuvent plus communiquer
    \end{description}
  \end{block}

  \uncover<2->{
    \begin{block}{Théorème -- Attiya, Bar Noy, Dolev ; Gilbert, Lynch}
      Dans le modèle par passage de message asynchrone avec crash, \\
      il est impossible d'implémenter une mémoire lire/écrire :
      \begin{description}[Partition tolerance :]
      \item[Consistency :] interdisant le motif RW1 (\alert{linéarisabilité}, \alert{c. séquentielle})
      \item[Availability :] dont les méthodes \Read et \Write \alert{terminent}
      \item[Partition tolerance :] même si \alert{$t\ge \frac{n}{2}$} processus tombent en panne
      \end{description}
    \end{block}
  }

  \only<2->{
    \footnoteref{H. Attiya, A. Bar-Noy, D. Dolev. \textit{Sharing memory robustly in message-passing systems}. JACM (1995)}  
    \footnoteref{S. Gilbert and N. Lynch. \textit{Brewer's conjecture and the feasibility of consistent, available, partition-tolerant web services}. ACM SIGACT News (2002)}
  }
  \footnoteref{E. A. Brewer. \textit{Towards robust distributed systems (abstract)}. PODC (2000)}  

\end{frame}

\endgroup
\endinput
