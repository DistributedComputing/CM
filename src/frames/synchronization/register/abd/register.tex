% SPDX-License-Identifier: CC-BY-SA-4.0
% Author: Matthieu Perrin
% Part: 
% Section: 
% Sub-section: 
% Frame: 

\begingroup

\SetKwFunction{Read}{read}
\SetKwFunction{Write}{write}
\SetKwData{Value}{value}
\SetKwData{Time}{time}
\SetKwData{Writer}{writer}

\begin{frame}{Implémentation d'un registre linéarisable}

  \on[top=-3mm]{
    \begin{algorithm}[H]
      \lLVariables{}{
        $\Value_i  \leftarrow \bot$; $\Time_i \leftarrow 0$; $\Writer_i \leftarrow 0$;
      }
      \Method{$\Read()$}{
        \uncover<3->{\Alert<3-4,8>{\textsc{mb}.\SBroadcast $\textsc{synch}()$;}\tcp*[f]{Barrière \Read $\rightarrow$ \Write}}\\
        \Alert<-2>{\Let $v \leftarrow \Value_i$;}\\
        \uncover<8->{\Structure<8>{\textsc{mb}.\SBroadcast $\textsc{w}(v, \Time_i, \Writer_i)$;}\tcp*[f]{Barrière \Read $\rightarrow$ \Read}}\\
        \Alert<-2>{\Return $v$;}
      }
      \Method{$\Write(v)$}{
        \uncover<6->{\Structure<6>{\textsc{mb}.\SBroadcast $\textsc{synch}()$;}\tcp*[f]{Barrière \Write $\rightarrow$ \Write}}\\
        \Example<-6>{\textsc{mb}.\SBroadcast $\textsc{w}(v, \Time_i+1, i)$;}\tcp*[f]{Last-Writer-Wins register}
      }
      \When{\textsc{mb}.\Deliver $\textsc{w}(v, t, w)$ \From $p_j$}{
        \Example<-2,6>{\lIf{$\langle \Time_i, \Writer_i \rangle <_{lex} \langle t, w \rangle$}{
            $\Value_i  \leftarrow v$; $\Time_i \leftarrow t$; $\Writer_i \leftarrow w$;
        }}
      }
    \end{algorithm}
  }

  \ob<1,2>[y=-25mm]{
    \begin{tikzpicture}[y=7mm]
      \draw[process] (0,2) node[left]{$p_1$} to (10,2);
      \draw[process] (0,1) node[left]{$p_2$} to (10,1);
      \draw[process] (0,0) node[left]{$p_3$} to (10,0);
      
      \node[example, operation] (w1) at (2,0) {$\Write(1)$};
      \path[example, message]   (w1.west) edge (2.5,2);
      \path[example, message]   (w1.west) edge (1.5,1);
      \path[example, message]   (w1.west) edge[bend left] (w1.east);

      \node[example, operation] (w2) at (5,2) {$\Write(2)$};
      \path[example, message]   (w2.west) edge [bend right] (w2.east);
      \path<1>[example, message]   (w2.west) edge (4.5,1);
      \path<2>[example, message]   (w2.west) edge[bend below] (9.5,1);
      \path[example, message]   (w2.west) edge (5.5,0);

      \node[alert,   operation] (r)  at (8,1) {$\Read() \rightarrow \alt<1>{2}{1}$};
    \end{tikzpicture}
  }
  
  \ob<3,4>[y=-25mm]{
    \begin{tikzpicture}[y=7mm]
      \draw[process] (0,2) node[left]{$p_1$} to (10,2);
      \draw[process] (0,1) node[left]{$p_2$} to (10,1);
      \draw[process] (0,0) node[left]{$p_3$} to (10,0);
      
      \node[fade, operation]         (w1) at (2,0) {$\Write(1)$};
      \path[fade, message]           (w1.west) edge (2.5,2);
      \path[fade, message]           (w1.west) edge (1.5,1);
      \path[fade, message]           (w1.west) edge[bend left] (w1.east);
      
      \node[alert,   operation]      (r)  at (8,1) {$\Read() \rightarrow \alt<4>{2}{1}$};
      \path[alert,   message, thick] (r.west) edge (8,2);
      \path[alert,   message]        (r.west) edge[bend left] (r.east);
      \path[alert,   message]        (r.west) edge (8,0);
      
      \node[example, operation]      (w2) at (5,2) {$\Write(2)$};
      \path[example, message]        (w2.west) edge[bend right] (w2.east);
      \path<3>[example, message, thick] (w2.west) edge[bend below] (9.5,1);
      \path<4>[example, message, thick] (w2.west) edge[bend below] (r.center);
      \path[example, message]        (w2.west) edge (5.5,0);
      
      \node<3>[alert] at (6,0.5) {MP1};
    \end{tikzpicture}
  }

  \ob<5>[y=-25mm]{
    \begin{tikzpicture}[y=7mm]
      \draw[process] (0,2) node[left]{$p_1$} to (10,2);
      \draw[process] (0,1) node[left]{$p_2$} to (10,1);
      \draw[process] (0,0) node[left]{$p_3$} to (10,0);

      \node[example, operation]      (w2)      at             (5,2) {$\Write(2)$};
      \draw[example, message]        (w2.west) to[bend right] (w2.east);
      \draw[example, message]        (w2.west) to             (4.5,1) node[below]{\small $\langle 1,1 \rangle$};
      \draw[example, message]        (w2.west) to             (5.5,0);
      
      \node[example, operation]      (w1)      at             (2,0) {$\Write(1)$};
      \draw[example, message, thick] (w1.west) to[bend above] (w2.center);
      \draw[example, message]        (w1.west) to             (1.5,1) node[above]{\small $\langle 1,3 \rangle$};
      \draw[example, message]        (w1.west) to[bend left]  (w1.east);
      
      \node[alert,   operation]      (r)       at             (8,1) {$\Read() \rightarrow 1$};
      \draw[alert,   message]        (r.west)  to             (8,2);
      \draw[alert,   message]        (r.west)  to[bend left]  (r.east);
      \draw[alert,   message]        (r.west)  to             (8,0);
    \end{tikzpicture}
  }

  \ob<6>[y=-25mm]{
    \begin{tikzpicture}[y=7mm]
      \draw[process] (0,2) node[left]{$p_1$} to (10,2);
      \draw[process] (0,1) node[left]{$p_2$} to (10,1);
      \draw[process] (0,0) node[left]{$p_3$} to (10,0);

      \node[example, operation]        (w2)        at             (5,2)         {$\Write(2)$};
      \draw[structure, message]        (w2.west)   to[bend right] (w2.center);
      \draw[structure, message]        (w2.west)   to             (4.5,1);
      \draw[structure, message, thick] (w2.west)   to             (5.5,0);
      \draw[example, message]          (w2.center) to[bend right] (w2.east);
      \draw[example, message]          (w2.center) to             (5.5,1)       node[below]{\small $\langle 2,1 \rangle$};
      \draw[example, message]          (w2.center) to             (6.5,0);
      
      \node[example, operation]        (w1)        at             (2,0)         {$\Write(1)$};
      \draw[structure, message]        (w1.west)   to             (2.5,2);
      \draw[structure, message]        (w1.west)   to             (1.5,1);
      \draw[structure, message]        (w1.west)   to[bend left]  (w1.center);
      \draw[example, message, thick]   (w1.center) to[bend above] (4.5,2);
      \draw[example, message]          (w1.center) to             (2.5,1)       node[above]{\small $\langle 1,3 \rangle$};
      \draw[example, message]          (w1.center) to[bend left]  (w1.east);
      
      \node[alert,   operation]        (r)         at             (8,1)         {$\Read() \rightarrow 2$};
      \draw[alert,   message]          (r.west)    to             (8,2);
      \draw[alert,   message]          (r.west)    to[bend left]  (r.east);
      \draw[alert,   message]          (r.west)    to             (8,0);
    \end{tikzpicture}
  }

  \ob<7>[y=-25mm]{
    \begin{tikzpicture}[y=7mm]
      \draw[process] (0,2) node[left]{$p_1$} to (10,2);
      \draw[process] (0,1) node[left]{$p_2$} to (10,1);
      \draw[process] (0,0) node[left]{$p_3$} to (10,0);

      \node[fade, operation]                     (w1)        at               (2,0)         {$\Write(1)$};
      \draw[fade, message]                       (w1.west)   to               (2.0,2);
      \draw[fade, message]                       (w1.west)   to               (1.5,1);
      \draw[fade, message]                       (w1.west)   to[bend left]    (w1.center);
      \draw[fade, message]                       (w1.center) to               (2.8,2);
      \draw[fade, message]                       (w1.center) to               (2.3,1);
      \draw[fade, message]                       (w1.center) to[bend left]    (w1.east);
      
      \node[example, operation, text width=55mm] (w2)        at               (6,2)         {$\Write(2)$};
      \coordinate                                (w2middle)  at               ([xshift=5mm]w2.west);
      \draw[structure, message]                  (w2.west)   to[bend right]   (w2middle);
      \draw[structure, message]                  (w2.west)   to               (3.5,1);
      \draw[structure, message]                  (w2.west)   to               (3.5,0);
      \draw[example, message]                    (w2middle)  to[bend right=8] (w2.east);
      \draw[example, message]                    (w2middle)  to[bend below]   (9.5,1);
      \draw[example, message]                    (w2middle)  to               (4,0);
      
      \node[alert,   operation]                  (r2)        at               (5.5,0)       {$\Read() \rightarrow 2$};
      \draw[alert,   message]                    (r2.west)   to               (5,2);
      \draw[alert,   message]                    (r2.west)   to               (5,1);
      \draw[alert,   message]                    (r2.west)   to[bend left]    (r2.east);
      
      \node[alert,   operation]                  (r1)        at               (8,1)         {$\Read() \rightarrow 1$};
      \draw[alert,   message]                    (r1.west)   to               (8,2);
      \draw[alert,   message]                    (r1.west)   to[bend left]    (r1.east);
      \draw[alert,   message]                    (r1.west)   to               (8,0);
    \end{tikzpicture}
  }

  \on<8>[y=-25mm]{
    \begin{tikzpicture}[y=7mm]

      \node[fade, operation]                     (w1)        at               (2,0)         {$\Write(1)$};
      \draw[fade, message]                       (w1.west)   to               (2.0,2);
      \draw[fade, message]                       (w1.west)   to               (1.5,1);
      \draw[fade, message]                       (w1.west)   to[bend left]    (w1.center);
      \draw[fade, message]                       (w1.center) to               (2.8,2);
      \draw[fade, message]                       (w1.center) to               (2.3,1);
      \draw[fade, message]                       (w1.center) to[bend left]    (w1.east);
      
      \node[example, operation, text width=55mm] (w2)        at               (6,2)         {$\Write(2)$};
      \coordinate                                (w2middle)  at               ([xshift=5mm]w2.west);
      \draw[structure, message]                  (w2.west)   to[bend right]   (w2middle);
      \draw[structure, message]                  (w2.west)   to               (3.5,1);
      \draw[structure, message]                  (w2.west)   to               (3.5,0);
      \draw[example, message]                    (w2middle)  to[bend right=8] (w2.east);
      \draw[example, message]                    (w2middle)  to[bend below]   (9.5,1);
      \draw[example, message]                    (w2middle)  to               (4,0);
      
      \node[alert,     operation]                (r1)        at               (8,1)         {$\Read() \rightarrow 2$};
      \draw[alert,     message]                  (r1.west)   to               (8,2);
      \draw[alert,     message]                  (r1.west)   to[bend left]    (r1.center);
      \draw[alert,     message, thick]           (r1.west)   to               (8,0);
      \draw[structure, message]                  (r1.center) to               (8.5,2);
      \draw[structure, message]                  (r1.center) to[bend left]    (r1.east);
      \draw[structure, message]                  (r1.center) to               (8.5,0);

      \node[alert,     operation]                (r2)        at               (5.5,0)       {$\Read() \rightarrow 2$};
      \draw[alert,     message]                  (r2.west)   to               (5,2);
      \draw[alert,     message]                  (r2.west)   to               (5,1);
      \draw[alert,     message]                  (r2.west)   to[bend left]    (r2.center);
      \draw[structure, message]                  (r2.center) to               (6,2);
      \draw[structure, message, thick]           (r2.center) to[bend above]   (7.5,1);
      \draw[structure, message]                  (r2.center) to[bend left]    (r2.east);

      \draw[process] (0,2) node[left]{$p_1$} to (10,2);
      \draw[process] (0,1) node[left]{$p_2$} to (10,1);
      \draw[process] (0,0) node[left]{$p_3$} to (10,0);

    \end{tikzpicture}
  }

  \obBlock<9>[y=-23mm]{Théorème -- L'algorithme est linéarisable}{\small
    \begin{itemize}
    \item\vspace{-1mm} Posons $\structure{\mathit{ts}(o) = \langle t, w\rangle}$ de son message \textsc{w}, et $\structure{o_1\;\mathcal{R}\;o_2}$ si : 
    \begin{itemize}
    \item $o_1$ se termine avant le début de $o_2$,
    \item $\mathit{ts}(o_1) <_{lex} \mathit{ts}(o_2)$,
    \item ou $\mathit{ts}(o_1) = \mathit{ts}(o_2)$, $o_1$ est une écriture et $o_2$ est une lecture.
    \end{itemize}
    \item\vspace{-1mm} \alert{$\mathcal{R}$ est acyclique}, et peut être étendu en un ordre total de linéarisation
    \end{itemize}
  }

  \footnoteref{H. Attiya, A. Bar-Noy, D. Dolev. \textit{Sharing memory robustly in message-passing systems} JACM (1995)}
  
\end{frame}

\endgroup
\endinput
