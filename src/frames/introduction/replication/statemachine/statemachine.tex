% SPDX-License-Identifier: CC-BY-SA-4.0
% Author: Matthieu Perrin
% Part: 
% Section: 
% Sub-section: 
% Frame: 

\begingroup

\SetKwFunction{Acheter}{acheter}
\SetKwFunction{Vendre}{vendre}
\SetKwFunction{Consulter}{consulter}

\begin{frame}{Abstraction du serveur Web}

  \begin{block}{Définition -- Machine à états}
    \vspace{3mm}
    Une \structure{machine à états} est un sextuplet \alert{$\langle C, R, Q, q_0, \tau, \rho \rangle$} tel que :
    \begin{description}[xxxxx]
    \item[\alert{$C$}] : ensemble de \structure{commandes} 
      \hfill {\footnotesize \example{$C =\{\Acheter(x), \Vendre(x), \Consulter \mid x \in X\}$}}
    \item[\alert{$R$}] : ensemble de \structure{réponses} possibles
      \hfill {\footnotesize \example{$R = \mathcal{P}(X) \cup \{\cmark, \xmark\}$}}
    \item[\alert{$Q$}] : ensemble d'\structure{états} (fini ou infini)
      \hfill {\footnotesize \example{$Q = \mathcal{P}(X)$}}
    \item[\alert{$q_0$}] $\in Q$ : l'\structure{état initial}
      \hfill {\footnotesize \example{$q_0 = \emptyset$}}
    \item[\alert{$\tau$}] $: Q \times C \rightarrow Q$ : la \structure{fonction de transition}
      \hfill {\footnotesize\example{$\tau(q, \Acheter(x)) = q \setminus \{x\}$, ...}}
    \item[\alert{$\rho$}] $: Q \times C \rightarrow R$ : la \structure{fonction de réponse}
      \hfill {\footnotesize\example{$\rho(q, \Consulter) = q$, ...}}
    \end{description}
  \end{block}

  \begin{block}{Remarques}
    \begin{itemize}
    \item Une commande de $C$ est formée d'une \structure{méthode} et d'\structure{arguments}
    \item Les ensembles $C$, $R$ et $Q$ peuvent être finis ou infinis
    \item On appelle \structure{opération} un couple $o = \langle c, r\rangle \in C \times R$, noté $c \rightarrow r$
    \item Les machines à états sont complètes et déterministes
    \end{itemize}
  \end{block}
  
  \on[y=-15mm, x=00mm]{
    \begin{tikzpicture}[automaton, x=25mm]
      \state            (q)  at (0,0) {$q$}; 
      \state[rectangle] (q1) at (1,0) {$\tau(q, c)$}; 
      \path             (q)  edge node {$c \rightarrow \rho(q, c)$} (q1);
    \end{tikzpicture}
  }
  
\end{frame}

\endgroup
\endinput


