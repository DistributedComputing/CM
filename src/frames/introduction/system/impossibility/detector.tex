% SPDX-License-Identifier: CC-BY-SA-4.0
% Author: Matthieu Perrin
% Part: 
% Section: 
% Sub-section: 
% Frame: 

\begingroup

\SetKwFunction{Crashed}{crashed}

\begin{frame}{Implémentation d'un détecteur de fautes}

  \onBlock[top=-5mm]{Problème -- Détecteur de fautes parfait $P$}{
    
    \begin{description}[Complétude :]
    \item[Interface :] Une méthode $\Crashed(p_j)$, appelée par $p_i$
      \begin{algorithm}[H]
        \Interface{$\mathit{P}$}{
          \lMethod{$\Crashed(p_j \in \mathit{Proc}) \in \mathbb{B}$}{\tcp*[h]{appelée par $p_i$}}
        }
      \end{algorithm}
    \item[Vivacité :] $\Crashed$ termine
    \item[Exactitude :] si $p_j$ est \alert{correct},  $\Crashed(p_j) \rightarrow \False$
    \item[Complétude :] si $p_j$ est \alert{déjà tombé en panne}, $\Crashed(p_j) \rightarrow \True$
    \end{description}
  }

  \onBlock<2->[y=-12mm]{Théorème -- Inexistence de $P$}{
    Il n'existe aucune implémentation de $P$ dans le modèle considéré.

    \structure{Démonstration :} par \alert{indistinguabilité} des exécutions suivantes par $p_i$.
  }

  \on<2->[x=-.3\textwidth, bottom=3mm]{
    \begin{tikzpicture}[y=8mm, anchor=center, x=9mm]
      \draw[crashed] (0,1) node[left]{$p_j$} to (.5,1);
      \draw[process] (0,0) node[left]{$p_i$} to (4.5,0);
      
      \node[structure, operation] (pj) at (2.25,0) {$\Crashed(p_j) \rightarrow \True$};
    \end{tikzpicture}
  }

  \on<2->[x=.2\textwidth, bottom=3mm]{
    \begin{tikzpicture}[y=8mm, anchor=center, x=9mm]
      \draw[process] (0,1) node[left]{$p_j$} to (6,1);
      \draw[process] (0,0) node[left]{$p_i$} to (6,0);

      \draw[dashed] (.5,.5) node[left]{\scriptsize réseau lent} -- (4 ,.5);
      
      \node[structure, operation] (pj) at (2.1,0) {$\Crashed(p_j) \rightarrow \True$};

      \begin{scope}[alert]
        \node[event]   (send-1) at (1,1) {};
        \node[event]   (rece-1) at (5,0) {};
        \path[message] (send-1.center) edge[out=-25, in=140] (rece-1.center);
      \end{scope}

      \begin{scope}[example]
        \node[event]   (send-2) at (2.6,1) {};
        \node[event]   (rece-2) at (5.6,0) {};
        \path[message] (send-2.center) edge[out=-10, in=140] (rece-2.center);
      \end{scope}


    \end{tikzpicture}
  }
  
\end{frame}

\endgroup
\endinput
