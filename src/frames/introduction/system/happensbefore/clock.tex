% SPDX-License-Identifier: CC-BY-SA-4.0
% Author: Matthieu Perrin
% Part: 
% Section: 
% Sub-section: 
% Frame: 

\begingroup

\SetKwFunction{Time}{time}

\begin{frame}{Temps et horloges logiques}
 
  \begin{block}{Définitions -- Temps logique}
    \begin{description}[Horloge :]
    \item[Temps :] relation d'ordre \structure{$\rightarrow$} (partielle ou totale) sur les pas d'exécution
    \item[Date :] type de valeurs muni d'un ordre (partiel ou total) \structure{$\le$}
    \item[Horloge :] fonction (algorithmique) \alert{strictement croissante} \structure{$\Time$}
      qui associe une date à certains pas : 
      $\alert{\forall a, b, \quad a \rightarrow b \quad \Rightarrow \quad \Time(a) < \Time(b)}$
    \end{description}
  \end{block}

  \begin{exampleblock}{Exemple -- Temps physique}
    \vspace{-2mm}
    \begin{description}[Temps :]
    \item[Temps :] ordre total en mécanique newtonnienne
    \item[Date :] nombre réel $t$ mesuré en secondes
    \item[Horloge :] à pendule, ressort, quartz, atomique, ...
    \end{description}
  \end{exampleblock}

  \pause
  
  \begin{alertblock}{Exemple -- Horloges de Lamport}
    \vspace{-2mm}
    \begin{description}[Horloge :]
    \item[Date :] horloge \structure{scalaire} \hfill \alert{$\mathit{Date} = \mathbb{N}$}
    \item[Temps :] \vspace{-1mm} capture la relation \structure{happened before} \hfill \alert{$a \hb b  \rightarrow \Time(a) < \Time(b)$}
    \end{description}
    \alert{Attention :} La réciproque n'est pas vraie en général :
    \centering
    $\alert{\exists a, b, a || b \land \Time(a) < \Time(b)} \quad \quad \alert{\exists a, b, a \neq b \land \Time(a) = \Time(b)}$
  \end{alertblock}
  
\end{frame}

\endgroup
\endinput

