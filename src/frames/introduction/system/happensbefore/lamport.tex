% SPDX-License-Identifier: CC-BY-SA-4.0
% Author: Matthieu Perrin
% Part: 
% Section: 
% Sub-section: 
% Frame: 

\begingroup

\begin{frame}{Lamport et la relativité du temps}

  \on[y=3mm, left=.7\textwidth]{
    \begin{shadequote}{Leslie Lamport}
      The \alert{concept of time} is fundamental to our way of
      thinking. \alert{It is derived} from the more basic concept of
      the \alert{order in which events occur}. (...)\\[2mm]
      
      However, we will see that this concept must be carefully reexamined when considering events in a distributed system. (...)\\[2mm]
      
      In a distributed system, it is sometimes impossible to
      say that one of two events occurred first.
      \alert{The relation
        "happened before" is therefore only a partial ordering
        of the events in the system.}
    \end{shadequote}
  }

  \onImage[x=40mm, y=12mm]{%
    width=2.5cm,
    title={Leslie Lamport\footnote{Prix Turing 2013 pour ses apports au calcul réparti}},
    licenselogo={\ccZero{}},
    license={CC-0 -- usage autorisé sans restriction (\href{https://commons.wikimedia.org/wiki/File:Leslie_Lamport.jpg}{Wikimedia})},
    img={Lamport.jpg}
  }

  \footnoteref{L. Lamport. \textit{Time, clocks, and the ordering of events in a distributed system.} CACM (1978)}

\end{frame}

\endgroup
\endinput

