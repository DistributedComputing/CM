% SPDX-License-Identifier: CC-BY-SA-4.0
% Author: Matthieu Perrin
% Part: 
% Section: 
% Sub-section: 
% Frame: 

\begingroup

\begin{frame}{Exécutions concrètes d'un algorithme}

  Exécution concrète = chemin dans un système de transitions.

  \begin{block}{Configuration globale du système}
    La \alert{configuration globale} est définie par un tuple \alert{$\langle q_1, ..., q_i, ..., q_n, M \rangle$} :
    \begin{itemize}
    \item L'état local \alert{$q_i$} de chaque processus $p_i$ contenant :
      \begin{description}
      \item[Pile d'appels :] la valeur de ses variables locales
      \item[Avancement :] la prochaine instruction à exécuter
      \end{description}
    \item L'ensemble \alert{$M$} des messages en cours d'acheminement
    \end{itemize}
  \end{block}

  \begin{block}{Transitions}
    Un \alert{ordonnanceur} (non-déterministe) décide le prochain pas parmi :
    \begin{description}[Réception]
    \item[Pas local] de $p_i$ (accès aux variables locales, appel de fonctions, etc)
      \begin{itemize}
      \item $\structure{\langle q_1, ..., \alert{q_i}, ..., q_n, M \rangle \leadsto \langle q_1, ..., \alert{q_i'}, ..., q_n, M \rangle}$
      \end{itemize}
    \item[Envoi] d'un message $m$ par $p_i$ à $p_j$
      \begin{itemize}
      \item $\structure{\langle q_1, ..., \alert{q_i}, ..., q_n, \alert{M} \rangle \leadsto \langle q_1, ..., \alert{q_i'}, ..., q_n, \alert{M \cup \{\langle p_i, p_j, m \rangle\}} \rangle}$
      \end{itemize}
    \item[Réception] d'un message $m$ par $p_j$ de la part de $p_i$ 
      \begin{itemize}
      \item $\structure{\langle q_1, ..., \alert{q_j}, ..., q_n, \alert{M} \rangle \leadsto \langle q_1, ..., \alert{q_j'}, ..., q_n, \alert{M \setminus \{\langle p_i, p_j, m \rangle\}} \rangle}$
      \end{itemize}
    \end{description}
  \end{block}

\end{frame}

\endgroup
\endinput
