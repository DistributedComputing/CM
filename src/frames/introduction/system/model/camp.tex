% SPDX-License-Identifier: CC-BY-SA-4.0
% Author: Matthieu Perrin
% Part: 
% Section: 
% Sub-section: 
% Frame: 

\begingroup

\begin{frame}{Hypothèses supplémentaires}
  
  \begin{block}{Résumé -- le modèle $\mathcal{CAMP}_{n,t}[H]$}
    \begin{description}
    \item[$\mathcal{CAMP}$ :] Crash-prone Asynchronous Message-Passing
    \item[$n$ :] nombre total de \structure{processus} (ensemble fixe et connu)
    \item[$t$ :] nombre maximal de pannes \structure{crash-stop}
    \item[$H$ :] ensemble d'\structure{hypothèses supplémentaires}. 
    \end{description}
  \end{block}
  
  Si rien n'est spécifié, on utilisera le modèle \alert{$\mathcal{CAMP}_{n,t}[\emptyset]$}.

  \vspace{4mm}
  
  \begin{exampleblock}{Exemples d'hypothèses supplémentaires}
    \begin{itemize}
    \item Borne \alert{$t < \frac{n}{k}$} sur le nombre de pannes
      \begin{itemize}
      \item Si l'hypothèse n'est pas vérifiée, la vivacité n'est plus garantie
      \end{itemize}
    \item Présence d'\alert{objets déjà disponibles}
      \begin{itemize}
      \item Utile pour les réductions : peut-on implémenter $B$ en utilisant $A$ ? 
      \end{itemize}
    \item Synchronie partielle, capacités cryptographiques...
    \end{itemize}
  \end{exampleblock}

\end{frame}

\endgroup
\endinput

