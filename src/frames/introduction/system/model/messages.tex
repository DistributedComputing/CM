% SPDX-License-Identifier: CC-BY-SA-4.0
% Author: Matthieu Perrin
% Part: 
% Section: 
% Sub-section: 
% Frame: 

\begingroup

\begin{frame}{Communication}

  \vspace{-2mm}
  \begin{block}{Interface de communication}
    Les processus communiquent en s'envoyant des messages sur un réseau
    \begin{description}[Réception :]
      \item[Émission :]  un processus $p_i$ peut \structure{envoyer} un message $m$ à $p_j$ en appelant :
      \begin{algorithm}[H]
        \Send $m$ \To $p_j$; \tcp{exécuté par $p_i$}
      \end{algorithm}
      \item[Réception :] plus tard, un \structure{événement de réception} de $m$ est généré chez $p_j$ :
        \begin{algorithm}[H]
          \lWhen{\Receive $m$ \From $p_i$}{\tcp*[h]{code exécuté par $p_j$}}
      \end{algorithm}
    \end{description}
    Pour simplifier, un message à soi-même est reçu immédiatement. 
  \end{block}
  
  \vspace{-2mm}
  \begin{block}{Hypothèse -- Canaux fiables asynchrones}
    \begin{description}[Réception :]
    \item[Validité :] si $p_i$ reçoit $m$ de $p_j$, $p_j$ a envoyé $m$ à $p_i$
    \item[Intégrité :] chaque processus $p_i$ reçoit $m$ au plus une fois
    \item[Fiabilité :] si un processus correct $p_i$ envoie $m$ à un processus correct $p_j$, \\
      $p_j$ reçoit $m$ de $p_i$
    \end{description}
    \alert{Asynchronisme :} ni borne sur la durée de transfert, ni ordre sur la réception
  \end{block}
  
\end{frame}

\endgroup
\endinput


