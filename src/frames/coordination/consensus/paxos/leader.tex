% SPDX-License-Identifier: CC-BY-SA-4.0
% Author: Matthieu Perrin
% Part: 
% Section: 
% Sub-section: 
% Frame: 

\begingroup

\SetKwFunction{Leader}{leader}
\SetKwFunction{Propose}{propose}
\SetKwData{Decided}{decided}

\begin{frame}[fragile]{Un algorithme centralisé}

  \onBlock[top=-4mm]{Le consensus sans pannes}{
    \begin{algorithm}[H]
      \LVariables{}{$\Decided_i \leftarrow \bot$;}
      \Method{$\Propose(v)$}{
        \If{\Alert{$i = \Leader()$}}{
          \textsc{rb}.\Broadcast $\textsc{decide}(v)$;
        }
        \Wait $\Decided_i \neq \bot$;\\
        \Return $\Decided_i$;\\
      }
      \When{\textsc{rb}.\Deliver $\textsc{decide}(v)$ \From $p_{\Leader}$}{
        $\Decided_i \leftarrow v$;
      }
    \end{algorithm}
  }

  \on[x=30mm, y=15mm]{
    \begin{tikzpicture}
      \draw[process] (0,2) node[left]{$p_1$} to (5,2); 
      \draw[process] (0,1) node[left]{$p_2$} to (5,1); 
      \draw[process] (0,0) node[left]{$p_3$} to (5,0); 

      \node[alert,     operation, minimum width=20mm] (p1) at (3,2)   {$\Propose(a) \rightarrow a$};
      \node[structure, operation, minimum width=30mm] (p2) at (2,1)   {$\Propose(b) \rightarrow a$};
      \node[structure, operation, minimum width=40mm] (p3) at (2.5,0) {$\Propose(c) \rightarrow a$};
      \draw[alert,   message]     (p1.west) to[bend left=30] (p1.east);
      \draw[alert,   message]     (p1.west) to[bend left=20] (p2.east);
      \draw[alert,   message]     (p1.west) to[bend left=30] (p3.east);
    \end{tikzpicture}
  }

  \onBlock[anchor=north, y=-7mm]{Définition -- Leader perpétuel}{
    Une fonction $\Leader()$ qui retourne un (identifiant de) processus telle que:
    \begin{itemize}
    \item $\Leader()$ retourne \alert{toujours le même} processus à tous les processus
    \item $\Leader()$ retourne un processus \alert{correct}
    \end{itemize}
    \alert{Problème :} Impossible de savoir que $\Leader()$ est \alert{correct} !
  }

  \footnoteref{E. Chang, R. Roberts. \textit{An improved algorithm for decentralized extrema-finding in circular configurations of processes.} CACM (1979)}
  
\end{frame}

\endgroup
\endinput
