% SPDX-License-Identifier: CC-BY-SA-4.0
% Author: Matthieu Perrin
% Part: 
% Section: 
% Sub-section: 
% Frame: 

\begingroup

\newcommand{\Abort}{\textsc{Abort}}
\newcommand{\Adopt}{\textsc{Adopt}}
\newcommand{\Commit}{\textsc{Commit}}

\begin{frame}{Consensus basé sur un leader}

  \begin{alertblock}{Rappel -- Impossibilité du consensus}
    Il est impossible d'avoir à la fois :
    \begin{description}
    \item[Accord: ] Au plus une valeur décidée
    \item[Validité: ] Toute valeur décidée a été proposée
    \item[Terminaison: ] \alert{\sout{{\color{black} Tout processus correct finit par décider une valeur}}}
    \end{description}
  \end{alertblock}
    
  \begin{block}{Stratégies pour le consensus}
    \begin{itemize}
    \item Garder le consensus toujours sûr
    \item Choisir une valeur commune en espérant la vivacité
    \end{itemize}
  \end{block}
   
  \begin{exampleblock}{Exemples}
    \begin{description}[Ben-Or :]
    \item[Ben-Or :] Choisir une valeur \example{aléatoire}
      \begin{itemize}
      \item Tout processus correct décide \example{avec probabilité $1$}
      \end{itemize}
    \item[Paxos :] Choisir la valeur d'un \example{leader} 
      \begin{itemize}
      \item Les processus décident \example{si le système est assez stable}
      \end{itemize}
    \end{description}
  \end{exampleblock}

  \on[x=40mm, y=-15mm]{
     \begin{tikzpicture}[automaton]\footnotesize
      \state[structure, rectangle, initial right] (a) at (0, 1) {Vérifier l'accord};
      \state[structure, rectangle]                (b) at (0, 2) {Choisir une valeur};
      \state[structure, rectangle, accepting]     (c) at (0, 0) {Décider};
      \path (b) edge[bend right] (a);
      \path (a) edge[bend right] node[swap]{\Abort} (b);
      \path (a) edge[loop left, looseness=3] node{\Adopt} (a);
      \path (a) edge             node{\Commit} (c);
    \end{tikzpicture}
  }
  
\end{frame}

\endgroup
\endinput
