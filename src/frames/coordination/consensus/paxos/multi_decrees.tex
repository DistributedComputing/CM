% SPDX-License-Identifier: CC-BY-SA-4.0
% Author: Matthieu Perrin
% Part: 
% Section: 
% Sub-section: 
% Frame: 

\begingroup

\begin{frame}{Multi-decree Paxos en pratique}

  Machine à états répliquée, pertes de messages...

  \begin{block}{Optimisations possibles}
    \begin{itemize}
    \item Message $\textsc{begin}$ mutualisé entre plusieurs rondes
    \item Message $\textsc{msg}$ envoyé au leader seulement dans Multi-decree Paxos
    \item Pas d'abstraction intermédiaire
    \end{itemize}
  \end{block}

  \begin{block}{Rôles des processus}
    Les processus ne jouent pas tous le même rôle :
    \begin{description}
    \item[Proposer :] autorisé à proposer une valeur
      \begin{itemize}
      \item \vspace{-1mm} Serveur en contact des clients
      \end{itemize}
    \item[Leader :] Coordonne les \emph{proposers}
      \begin{itemize}
      \item \vspace{-1mm} Idéalement, un seul processus, correct. 
      \end{itemize}
    \item[Acceptor :] Participent au quorum du \textsc{mb}.\Broadcast
      \begin{itemize}
      \item \vspace{-1mm} Une majorité doit être correcte
      \end{itemize}
    \item[Learner :] Destinataire du message $\textsc{decide}(v)$ final
      \begin{itemize}
      \item \vspace{-1mm} processus sur lequel est stocké un réplica
      \end{itemize}
    \end{description}
  \end{block}

\end{frame}

\endgroup
\endinput
