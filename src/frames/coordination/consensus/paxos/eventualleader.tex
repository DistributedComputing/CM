% SPDX-License-Identifier: CC-BY-SA-4.0
% Author: Matthieu Perrin
% Part: 
% Section: 
% Sub-section: 
% Frame: 

\begingroup

\SetKwFunction{Leader}{leader}
\SetKwFunction{Sleep}{sleep}
\SetKwFunction{Time}{current\_time}
\SetKwData{Lastheard}{last\_heard}

\begin{frame}[fragile]{Élection de leader}

  \on[top=-3mm]{
    \begin{algorithm}[H]
      \lLVariables{}{
        $\Lastheard_i \leftarrow [\Time(), ..., \Time()]$;
      }
      \Method{$\Leader()$}{
        \For(\tcp*[f]{ou tri déterministe selon le ping moyen}){$j$ \From $1$ \To $n$}{
          \lIf{$\Time() - \Lastheard_i[j] < 2 \Delta$}{
            \Return $j$;
          }
        }
      }
      \Task{}{
        \While{\True}{
          $\Sleep(\Delta)$;
          \textsc{sta}.\Broadcast $\textsc{heartbeat}()$;
        }
      }
      \When{\textsc{sta}.\Deliver $\textsc{heartbeat}()$ \From $p_j$}{
        $\Lastheard_i[j] \leftarrow \Time()$;
      }
    \end{algorithm}
  }

  \obBlock<1>[anchor=north, y=-2mm]{Analyse en système synchrone}{
    \vspace{-2mm}
    \begin{itemize}
    \item $\Time()$ avance à la même vitesse pour tous les processus
    \item Tous les messages sont acheminés en au plus $\Delta$ unités de temps
    \end{itemize}
  }

  \ob<1>[anchor=north, y=-22mm]{
    \begin{tikzpicture}[anchor=mid, y=6mm, x=20mm]

      \begin{scope}
        \clip (0.8,1.9) rectangle (2.1,2.1);
        \fill[structure!20]                     (0.8,2) +(0,-.1) rectangle +(2,.1);
        \fill[structure!60, path fading=east]   (0.8,2) +(0,-.1) rectangle +(1,.1);
        \fill[structure!20]                     (1.7,2) +(0,-.1) rectangle +(2,.1);
        \fill[structure!60, path fading=east]   (1.7,2) +(0,-.1) rectangle +(1,.1);
      \end{scope}

      \begin{scope}
        \clip (0.8,0.9) rectangle (5.2,1.1);
        \fill[structure!20]                     (0.8,1) +(0,-.1) rectangle +(2,.1);
        \fill[structure!60, path fading=east]   (0.8,1) +(0,-.1) rectangle +(1,.1);
        \fill[alert!20]                         (2.8,1) +(0,-.1) rectangle +(2,.1);
        \fill[alert!60, path fading=east]       (2.8,1) +(0,-.1) rectangle +(1,.1);
        \fill[alert!20]                         (3.8,1) +(0,-.1) rectangle +(2,.1);
        \fill[alert!60, path fading=east]       (3.8,1) +(0,-.1) rectangle +(1,.1);
        \fill[alert!20]                         (4.8,1) +(0,-.1) rectangle +(2,.1);
        \fill[alert!60, path fading=east]       (4.8,1) +(0,-.1) rectangle +(1,.1);
        \fill[structure!20]                     (1.7,1) +(0,-.1) rectangle +(2,.1);
        \fill[structure!60, path fading=east]   (1.7,1) +(0,-.1) rectangle +(1,.1);
      \end{scope}

      \begin{scope}
        \clip (0.8,-.1) rectangle (5.2,0.1);
        \fill[structure!20]                     (0.8,0) +(0,-.1) rectangle +(2,.1);
        \fill[structure!60, path fading=east]   (0.8,0) +(0,-.1) rectangle +(1,.1);
        \fill[alert!20]                         (2.8,0) +(0,-.1) rectangle +(2,.1);
        \fill[alert!60, path fading=east]       (2.8,0) +(0,-.1) rectangle +(1,.1);
        \fill[alert!20]                         (3.8,0) +(0,-.1) rectangle +(2,.1);
        \fill[alert!60, path fading=east]       (3.8,0) +(0,-.1) rectangle +(1,.1);
        \fill[alert!20]                         (4.8,0) +(0,-.1) rectangle +(2,.1);
        \fill[alert!60, path fading=east]       (4.8,0) +(0,-.1) rectangle +(1,.1);
        \fill[structure!20]                     (1.7,0) +(0,-.1) rectangle +(2,.1);
        \fill[structure!60, path fading=east]   (1.7,0) +(0,-.1) rectangle +(1,.1);
      \end{scope}
      
      \foreach \x in {2,...,5}{
        \path (\x,-.2) edge[brace] node{$\Delta$} +(-1,0);
        \draw[dotted] (\x,-.2) -- (\x,2.2);
      }
      \draw[dotted] (1,-.2) -- (1,2.2);

      \draw[crashed] (.8,2) node[left]{$p_1$} to (2.1,2); 
      \draw[process] (.8,1) node[left]{$p_2$} to (5.2,1); 
      \draw[process] (.8,0) node[left]{$p_3$} to (5.2,0); 

      \draw[structure, message] (1.1,2) to[bend left] (1.7,2);
      \draw[structure, message] (1.1,2) to            (1.7,1);
      \draw[structure, message] (1.1,2) to            (1.7,0);
      \draw[alert, message]     (1.2,1) to            (1.8,2);
      \draw[alert, message]     (1.2,1) to[bend left] (1.8,1);
      \draw[alert, message]     (1.2,1) to            (1.8,0);
      \draw[example, message]   (1.3,0) to            (1.9,2);
      \draw[example, message]   (1.3,0) to            (1.9,1);
      \draw[example, message]   (1.3,0) to[bend left] (1.9,0);

      \draw[alert, message]     (2.2,1) to[bend left] (2.8,1);
      \draw[alert, message]     (2.2,1) to            (2.8,0);
      \draw[example, message]   (2.3,0) to            (2.9,1);
      \draw[example, message]   (2.3,0) to[bend left] (2.9,0);

      \draw[alert, message]     (3.2,1) to[bend left] (3.8,1);
      \draw[alert, message]     (3.2,1) to            (3.8,0);
      \draw[example, message]   (3.3,0) to            (3.9,1);
      \draw[example, message]   (3.3,0) to[bend left] (3.9,0);

      \draw[alert, message]     (4.2,1) to[bend left] (4.8,1);
      \draw[alert, message]     (4.2,1) to            (4.8,0);
      \draw[example, message]   (4.3,0) to            (4.9,1);
      \draw[example, message]   (4.3,0) to[bend left] (4.9,0);

    \end{tikzpicture}
  }

  \obBlock<2>[anchor=north, y=-2mm]{Analyse en système asynchrone}{
    \vspace{-2mm}
    \begin{itemize}
    \item Si un processus tombe en panne, il finit par ne plus être élu
    \item Pas de garantie de stabilité
    \end{itemize}
  }

  \ob<2>[anchor=north, y=-22mm]{
    \begin{tikzpicture}[anchor=mid, y=6mm, x=20mm]

      \begin{scope}
        \clip (0.8,1.9) rectangle (2.1,2.1);
        \fill[structure!20]                     (0.8,2) +(0,-.1) rectangle +(2,.1);
        \fill[structure!60, path fading=east]   (0.8,2) +(0,-.1) rectangle +(1,.1);
        \fill[structure!20]                     (1.7,2) +(0,-.1) rectangle +(2,.1);
        \fill[structure!60, path fading=east]   (1.7,2) +(0,-.1) rectangle +(1,.1);
      \end{scope}

      \begin{scope}
        \clip (0.8,0.9) rectangle (5.2,1.1);
        \fill[alert!20]                         (2.8,1) +(0,-.1) rectangle +(2,.1);
        \fill[alert!60, path fading=east]       (2.8,1) +(0,-.1) rectangle +(1,.1);
        \fill[alert!20]                         (3.8,1) +(0,-.1) rectangle +(2,.1);
        \fill[alert!60, path fading=east]       (3.8,1) +(0,-.1) rectangle +(1,.1);
        \fill[alert!20]                         (4.8,1) +(0,-.1) rectangle +(2,.1);
        \fill[alert!60, path fading=east]       (4.8,1) +(0,-.1) rectangle +(1,.1);

        \fill[structure!20]                     (0.8,1) +(0,-.1) rectangle +(2,.1);
        \fill[structure!60, path fading=east]   (0.8,1) +(0,-.1) rectangle +(1,.1);
        \fill[structure!20]                     (3.0,1) +(0,-.1) rectangle +(2,.1);
        \fill[structure!60, path fading=east]   (3.0,1) +(0,-.1) rectangle +(1,.1);
      \end{scope}

      \begin{scope}
        \clip (0.8,-.1) rectangle (5.2,0.1);
        \fill[example!20]                       (1.9,0) +(0,-.1) rectangle +(1,.1);
        \fill[example!60, path fading=east]     (1.9,0) +(0,-.1) rectangle +(1,.1);
        \fill[example!20]                       (2.9,0) +(0,-.1) rectangle +(1,.1);
        \fill[example!60, path fading=east]     (2.9,0) +(0,-.1) rectangle +(1,.1);
        \fill[example!20]                       (3.9,0) +(0,-.1) rectangle +(1,.1);
        \fill[example!60, path fading=east]     (3.9,0) +(0,-.1) rectangle +(1,.1);
        \fill[example!20]                       (4.9,0) +(0,-.1) rectangle +(1,.1);
        \fill[example!60, path fading=east]     (4.9,0) +(0,-.1) rectangle +(1,.1);

        \fill[alert!20]                         (1.4,0) +(0,-.1) rectangle +(1,.1);
        \fill[alert!60, path fading=east]       (1.4,0) +(0,-.1) rectangle +(1,.1);
        \fill[alert!20]                         (3.4,0) +(0,-.1) rectangle +(1,.1);
        \fill[alert!60, path fading=east]       (3.4,0) +(0,-.1) rectangle +(1,.1);
        \fill[alert!20]                         (3.5,0) +(0,-.1) rectangle +(1,.1);
        \fill[alert!60, path fading=east]       (3.5,0) +(0,-.1) rectangle +(1,.1);
        \fill[alert!20]                         (5.0,0) +(0,-.1) rectangle +(1,.1);
        \fill[alert!60, path fading=east]       (5.0,0) +(0,-.1) rectangle +(1,.1);
        
        \fill[structure!20]                     (0.8,0) +(0,-.1) rectangle +(1,.1);
        \fill[structure!60, path fading=east]   (0.8,0) +(0,-.1) rectangle +(1,.1);
        \fill[structure!20]                     (1.2,0) +(0,-.1) rectangle +(1,.1);
        \fill[structure!60, path fading=east]   (1.2,0) +(0,-.1) rectangle +(1,.1);
      \end{scope}
      
      \draw[crashed] (.8,2) node[left]{$p_1$} to (2.1,2); 
      \draw[process] (.8,1) node[left]{$p_2$} to (5.2,1); 
      \draw[process] (.8,0) node[left]{$p_3$} to (5.2,0); 

      \draw[structure, message] (1.1,2) to[bend left] (1.7,2);
      \draw[structure, message] (1.1,2) to[bend below](3.0,1);
      \draw[structure, message] (1.1,2) to            (1.2,0);
      \draw[alert, message]     (1.2,1) to[bend left] (1.8,1);
      \draw[alert, message]     (1.2,1) to            (1.5,0);
      \draw[example, message]   (1.3,0) to            (1.9,1);
      \draw[example, message]   (1.3,0) to[bend left] (1.9,0);

      \draw[alert, message]     (2.2,1) to[bend left] (2.8,1);
      \draw[alert, message]     (2.2,1) to            (3.4,0);
      \draw[example, message]   (2.3,0) to            (2.9,1);
      \draw[example, message]   (2.3,0) to[bend left] (2.9,0);

      \draw[alert, message]     (3.2,1) to[bend left] (3.8,1);
      \draw[alert, message]     (3.2,1) to            (3.5,0);
      \draw[example, message]   (3.3,0) to            (3.9,1);
      \draw[example, message]   (3.3,0) to[bend left] (3.9,0);

      \draw[alert, message]     (4.2,1) to[bend left] (4.8,1);
      \draw[alert, message]     (4.2,1) to            (5.0,0);
      \draw[example, message]   (4.3,0) to            (4.9,1);
      \draw[example, message]   (4.3,0) to[bend left] (4.9,0);

    \end{tikzpicture}
  }
  
  \obBlock<3>[anchor=north, y=-2mm]{Cas pratique : Système partiellement synchrone}{
    \begin{description}[Périodes instables :] 
    \item[Périodes stables :] réseau plus ou moins synchrone, peu de crash
      \begin{itemize}
      \item \structure{élection} d'un leader stable
      \end{itemize}
    \item[Périodes instables :] pics de latence, partitions, crashs
      \begin{itemize}
      \item \structure{désaccord} sur le leader
      \end{itemize}
    \end{description}
    \centering
    \alert{Comment s'assurer que deux leaders prennent des décisions cohérentes ?}
  }
  
  \onBlock<4>[anchor=north, y=-2mm]{Définition -- Eventual leader $\Omega$}{
    \begin{algorithm}[H]
      \Interface{$\Omega$}{
        \lMethod{$\Leader()$}{\tcp*[f]{retourne un identifiant de processus}}
      }
    \end{algorithm}
    \begin{itemize}
    \item $\Leader()$ \alert{finira par} retourner \structure{le même} processus à tous les processus
    \item $\Leader()$ \alert{finira par} retourner un processus \structure{correct}
    \end{itemize}
  }

  \only<4>{
    \footnoteref{T. Chandra, V. Hadzilacos, S. Toueg. \textit{The weakest failure detector for solving consensus.} JACM (1996)}
    \footnoteref{T. Chandra, S. Toueg. \textit{Unreliable failure detectors for reliable distributed systems.} JACM (1996)}
  }
  
\end{frame}

\endgroup
\endinput
