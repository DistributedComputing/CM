% SPDX-License-Identifier: CC-BY-SA-4.0
% Author: Matthieu Perrin
% Part: 
% Section: 
% Sub-section: 
% Frame: 

\begingroup

\SetKwFunction{Draw}{draw}
\SetKwFunction{Rand}{random}
\SetKwData{Count}{count}
\SetKwData{Sum}{sum}

\begin{frame}{Un common coin simple $(t < \sqrt{n})$}

  \on[top=-3mm]{
    \begin{algorithm}[H]
      \lLVariables{}{
        $\Count_i \leftarrow 0$; $\Sum_i \leftarrow 0$;
      }
      \Method{$\Draw() \in \mathbb{B}$}{
        \textsc{sta}.\Broadcast $\textsc{draw}(\Rand())$; \tcp*[f]{Tirage local dans $\{0, 1\}$}\\
        \Wait $\Count_i \ge n-t$; \tcp*[f]{Récolte $n - t$ valeur}\\
        \Return $\Sum_i \ge \frac{\Count_i}{2}$; \tcp*[f]{Retourne la valeur la plus reçue}
      }
      \When{\textsc{sta}.\Deliver $\textsc{draw}(r_j)$ \From $p_j$}{
        $\Sum_i \leftarrow \Sum_i + r_j$;
        $\Count_i \leftarrow \Count_i + 1$;\\
      }
    \end{algorithm}
  }

  \obExampleBlock<2>[y=10mm, anchor=north]{Exemple pour $n=9$, $t=2$, et $3$ processus tirent \example{$\Rand() = 1$}}{
    \begin{itemize}
    \item Messages \textsc{draw} envoyés : 
      \begin{itemize}
      \item \alert{$6$} processus envoient \alert{$\textsc{draw}(0)$}
      \item \example{$3$} processus envoient \example{$\textsc{draw}(1)$}
      \end{itemize}
    \item Messages \textsc{draw} reçus : 
      \begin{itemize}
      \item Chaque processus reçoit au moins $6-2 = \alert{4}$ messages \alert{$\textsc{draw}(0)$}
      \item Chaque processus reçoit au plus \example{$3$} messages \example{$\textsc{draw}(1)$}
      \end{itemize}
    \item Tous les processus reçoivent plus de \alert{$\textsc{draw}(0)$} que de \example{$\textsc{draw}(1)$}
      \begin{itemize}
      \item Tous les processus retournent $0$
      \end{itemize}
    \end{itemize}
  }

  \obBlock<3>[y=10mm, anchor=north]{Analyse}{
    \begin{itemize}
    \item\vspace{-1mm} Le nombre de messages \example{$\textsc{draw}(1)$} est \structure{\footnotesize$\displaystyle \sum_{i=1}^n r_i$}
    \item\vspace{-1mm} On cherche $\rho > 0$ tel que
      \alert{\footnotesize$\displaystyle \mathbb{P}\left( \sum_{i=1}^n r_i\leq \frac{n-t}{2}\right) \geq \rho $} et
      \example{\footnotesize$\displaystyle \mathbb{P}\left( \sum_{i=1}^n r_i\geq \frac{n+t}{2}\right) \geq \rho $}
    \end{itemize}
  }
  
  \onBlock<4>[y=10mm, anchor=north]{Théorème Central-Limite}{
    Pour $n$ variables aléatoires $r_1, ..., r_n$ réelles d'espérance $\mu$ et d'écart-type $\sigma$ :\\[-2mm]
    $$\forall \alpha>0,  \lim_{n \to \infty}\mathbb{P}\left(\frac{\sum_{i=1}^n r_i - n \mu}{\sigma\sqrt{n}}\leq \alpha\right)  =  \Phi(\alpha)$$
  }
  
  \on<3->[bottom=-3mm]{
    \begin{tikzpicture}
      \def\mu{0}       % moyenne
      \def\sigma{1}    % écart-type
      \def\A{4.5}      % facteur d'échelle vertical (optionnel)
      \def\B{0.12mm}      % facteur d'échelle vertical (optionnel)

      \uncoverb<3>{
        \draw[fill=alert!30]   (-5,0) rectangle (-4,1*\B);
        \draw[fill=alert!30]   (-4,0) rectangle (-3,9*\B);
        \draw[fill=alert!30]   (-3,0) rectangle (-2,36*\B);
        \draw[fill=alert!30]   (-2,0) rectangle (-1,84*\B);
        \draw[fill=black!20]   (-1,0) rectangle ( 0,126*\B);
        \draw[fill=black!20]   ( 0,0) rectangle ( 1,126*\B);
        \draw[fill=example!30] ( 1,0) rectangle ( 2,84*\B);
        \draw[fill=example!30] ( 2,0) rectangle ( 3,36*\B);
        \draw[fill=example!30] ( 3,0) rectangle ( 4,9*\B);
        \draw[fill=example!30] ( 4,0) rectangle ( 5,1*\B);
      }
      \uncover<4>{
        \draw[domain=-5:5, samples=500, smooth] plot (\x,{\A*(1/(\sigma*sqrt(2*pi)))*exp(-((\x-\mu)^2)/(2*\sigma^2))});
        \begin{scope}[background]
          \fill<4>[alert!30,   domain=-4:-1, samples=150, smooth] plot (\x,{\A*(1/(\sigma*sqrt(2*pi)))*exp(-((\x-\mu)^2)/(2*\sigma^2))}) -- (-1,0) -- (-4,0) -- cycle;
          \fill<4>[black!20,   domain=-1:1,  samples=100, smooth] plot (\x,{\A*(1/(\sigma*sqrt(2*pi)))*exp(-((\x-\mu)^2)/(2*\sigma^2))}) -- ( 1,0) -- (-1,0) -- cycle;
          \fill<4>[example!30, domain=1:4,   samples=150, smooth] plot (\x,{\A*(1/(\sigma*sqrt(2*pi)))*exp(-((\x-\mu)^2)/(2*\sigma^2))}) -- ( 4,0) -- ( 1,0) -- cycle;
        \end{scope}
      }

      \scriptsize
      \uncoverb<3>{
        \draw (-4.5,0) node[below]{$0$}; 
        \draw (-3.5,0) node[below]{$1$}; 
        \draw (-2.5,0) node[below]{$2$}; 
        \draw (-1.5,0) node[below]{$3$}; 
        \draw (-0.5,0) node[below]{$4$}; 
        \draw ( 0.5,0) node[below]{$5$}; 
        \draw ( 1.5,0) node[below]{$6$}; 
        \draw ( 2.5,0) node[below]{$7$}; 
        \draw ( 3.5,0) node[below]{$8$}; 
        \draw ( 4.5,0) node[below]{$9 = n$}; 
      }
      \uncover<4>{
        \draw (-5,0) node[below]{$0$}; 
        \draw ( 0,0) node[below]{$n/2$}; 
        \draw ( 5,0) node[below]{$t^2 = n$}; 
      }

      \draw[->]             (-5,0) -- ( 5.2,0) node[right]{$s$};
      \draw[->]             (-5,0) -- (-5  ,2) node[below left]{$\displaystyle\mathbb{P}\left(\sum_{i=1}^n r_i = s\right)$};
      \draw[densely dotted] (-1,0) -- (-1,2);
      \draw[densely dotted] ( 0,0) -- ( 0,2);
      \draw[densely dotted] ( 1,0) -- ( 1,2);
      \path[latex-latex]    (-1,2) edge node[above]{$t\only<4>{\simeq \sqrt{n}}$} (1,2);
      \node[right]       at ( 1,2) {$\frac{n+t}{2}$};
      \node[left]        at (-1,2) {$\frac{n-t}{2}$};

      \footnotesize
      \uncoverb<3>{
        \node   at (-3,1) {Si \alert{  $\displaystyle \sum_{i=1}^n r_i \leq \frac{n-t}{2}$}, accord sur \alert{$0$}};
        \node   at ( 3,1) {Si \example{$\displaystyle \sum_{i=1}^n r_i \geq \frac{n+t}{2}$}, accord sur \example{$1$}};
      }
      \uncover<4>{
        \node[alert]   at (-3,1) {$\displaystyle \lim_{n \to \infty}\mathbb{P}\left( \sum_{i=1}^n r_i\leq \frac{n-\sqrt{n}}{2} \right) \simeq 0.159$};
        \node[example] at ( 3,1) {$\displaystyle \lim_{n \to \infty}\mathbb{P}\left( \sum_{i=1}^n r_i\geq \frac{n+\sqrt{n}}{2} \right) \simeq 0.159$};
      }

    \end{tikzpicture}
  }

  \onlyb<-2>{
    \footnoteref{M. Rabin. \textit{Randomized byzantine generals.} SFCS. (1983)}
    \footnoteref{J. Aspnes, M. Herlihy. \textit{Fast randomized consensus using shared memory.} J. of Algorithms. (1990)}
  }

\end{frame}

\endgroup
\endinput
