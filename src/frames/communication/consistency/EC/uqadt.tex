% SPDX-License-Identifier: CC-BY-SA-4.0
% Author: Matthieu Perrin
% Part: Communication
% Section: Cohérence faible
% Frame: Eventual Consistency

\begingroup

\SetKwFunction{SetKey}{setKey}
\SetKwFunction{DeleteKey}{deleteKey}
\SetKwFunction{Get}{get}
\SetKwFunction{GetKeys}{getKeys}

\begin{frame}{Machine à états à requêtes/mises à jour}

  Une machine à états $M=\langle C, R, Q, q_0, \tau, \rho \rangle$ est dite \structure{à requêtes/mises à jour} si
  toutes ses commandes $c\in C$ sont dans l'une des deux catégories : 
  \begin{description}
  \item[Requête :] \alert{$c \in \mathit{Query}$} ne modifie pas l'état
    \hfill $\alert{\forall q\in Q, \tau(q, c) = q}$\\
  \item[Mise à jour :] \alert{$c \in \mathit{Update}$} retourne $\bot$
    \hfill $\alert{\exists \bot\in R, \forall q\in Q, \rho(q, c) = \bot}$\\
  \end{description}

  \vspace{2mm}
  On étend la notation aux exécutions $E$ d'une implémentation répliquée de $M$ :
  \alert{$\mathit{Query}(E) = \{ c\rightarrow r \in E \mid c \in \mathit{Query} \}$ \hfill $\mathit{Update}(E) = \{ c\rightarrow r \in E \mid c \in \mathit{Update} \}$}

  \pause
  \vspace{2mm}

  \begin{exampleblock}{Exemple -- Bases de données clé-valeur}
    Une \example{base de données clé-valeur} sur un ensemble de clés $K$ et un ensemble de valeurs $V$
    est une machine à états à requêtes/mises à jour avec : 
    \begin{itemize}
    \item Un état $q\in Q$ est une \example{fonction partielle de $K$ dans $V$}
    \item Les mise à jour sont de la forme \example{$\SetKey(k,v)$} ou \example{$\DeleteKey(k)$}
      \begin{center}
        $\tau(q, \SetKey(k,v)) = q[k \mapsto v]  \quad\quad \tau(q, \DeleteKey(k)) = q \setminus (\{k\}\times V)$
      \end{center}
    \item Les requêtes sont de la forme \example{$\Get(k)$} ou \example{$\GetKeys()$}
      \begin{center}
        $\rho(q, \Get(k)) = q(k) ~~ (\text{ou } \bot \text{ si } k \notin dom(q))  \quad \rho(q, \GetKeys()) = dom(q)$
      \end{center}
    \end{itemize}
  \end{exampleblock}

\end{frame}

\endgroup
\endinput
