% SPDX-License-Identifier: CC-BY-SA-4.0
% Author: Matthieu Perrin
% Part: 
% Section: 
% Sub-section: 
% Frame: 

\begingroup

\begin{frame}{Problème de causalité}

  \on[y=-5mm]{
    \begin{tikzpicture}
      \node[faded background picture=Seine,  text width=\paperwidth/3] (A) at (-\paperwidth/3,0) {};
      \node[faded background picture=Jardin, text width=\paperwidth/3] (B) at (0,0) {};
      \node[faded background picture=Salon,  text width=\paperwidth/3] (C) at (\paperwidth/3,0) {};
      \node[anchor=south, nosep] at (A.south) {\includegraphics[height=22mm]{Alice}};
      \node[anchor=south, nosep] at (B.south) {\includegraphics[height=22mm]{Bob}};
      \node[anchor=south, nosep] at (C.south) {\includegraphics[height=22mm]{Carole}};
    \end{tikzpicture}
  }

  \on[x=-\paperwidth/3, y=15mm]{
    \begin{chat}[color=structure]{Bob, Carole}
      \only<1->{\chatRecv[color=alert]{Je suis à New-York toute la semaine}}
      \only<3->{\chatSend{Tu as tellement de chance :)}}
      \only<4->{\chatRecv[color=example]{Qui m'aide à retrouver mon chat ? :'-(}}
    \end{chat}
  }
 
  \on[y=15mm]{
    \begin{chat}[color=example]{Alice, Carole}
      \only<2->{\chatSend{Qui m'aide à retrouver mon chat ? :'-(}}
      \only<3->{\chatRecv[color=structure]{Tu as tellement de chance :)}}
      \only<4->{\chatRecv[color=alert]{Je suis à New-York toute la semaine}}
    \end{chat}
  }

  \on[x=\paperwidth/3, y=15mm]{
    \begin{chat}[color=alert]{Alice, Bob}
      \only<1->{\chatSend{Je suis à New-York toute la semaine}}
      \only<2->{\chatRecv[color=example]{Qui m'aide à retrouver mon chat ? :'-(}}
      \only<3->{\chatRecv[color=structure]{Tu as tellement de chance :)}}
    \end{chat}
  }

  \on<1>[x=\paperwidth/3, y=-15mm]{
    \chatBubble[color=alert]{Je suis à New-York toute la semaine}
  }

  \on<2>[y=-15mm]{
    \chatBubble[color=example]{Qui m'aide à retrouver mon chat ? :'-(}
  }

  \on<3>[x=-\paperwidth/3, y=-15mm]{
    \chatBubble[color=structure]{Tu as tellement de chance :)}
  }

  \on<4>[y=27mm]{
    \begin{tikzpicture}
      \path[thick, -latex] ( 3.2,5)   edge[bend right=35] (-3.2,5);
      \path[thick, -latex] (-3.2,4.9) edge[bend left]     (-3,4.5);
      \path[thick, -latex] (-2.8,4.1) edge                (-1.3,4.1);
    \end{tikzpicture}
  }

  
\end{frame}

\endgroup
\endinput
