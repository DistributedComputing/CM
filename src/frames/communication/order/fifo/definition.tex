% SPDX-License-Identifier: CC-BY-SA-4.0
% Author: Matthieu Perrin
% Part: 
% Section: 
% Sub-section: 
% Frame: 

\begingroup

\begin{frame}{FIFO Broadcast}
  
  \begin{block}{Propriétés héritées de \textsc{sta}.\Broadcast}
    \vspace{-1mm}
    \begin{description}[FIFO-Terminaison :]
    \item[FIFO-Validité :]    Si $p_i$ \textsc{fifo}.\Deliver $m$ de $p_j$, $p_j$ a \textsc{fifo}.\Broadcast $m$
    \item[FIFO-Intégrité :]   $p_i$ \textsc{fifo}.\Deliver $m$ au plus une fois
    \item[FIFO-Terminaison :] Si un correct \textsc{fifo}.\Broadcast $m$, \\ tous les corrects \textsc{fifo}.\Deliver $m$.
    \end{description}
  \end{block}

  \begin{alertblock}{Nouvelle propriété d'ordre}
    \vspace{-1mm}
    \begin{description}[FIFO-Ordering :]
    \item[\alert{FIFO-Ordering :}] Si un processus $p_i$ \textsc{fifo}.\Broadcast $m$ avant $m'$, \\alors aucun processus $p_j$ ne \textsc{fifo}.\Deliver $m'$ avant $m$
    \end{description}

    \centering
    \begin{tikzpicture}[y=8mm]
      \draw[process] (0,1) node[left]{$p_1$} to (10,1);
      \draw[process] (0,0) node[left]{$p_2$} to (10,0);

      \draw[structure]          (1,1) node (m1) {} node[above] {$m$} ;
      \draw[structure, message] (m1.center) to[bend left] (2,1) ;
      \draw[structure, message] (m1.center) to            (5,0) ;

      \draw[example]            (3,1) node (m2) {} node[above] {$m'$} ;
      \draw[example, message]   (m2.center) to[bend left] (4,1) ;
      \draw[example, message]   (m2.center) to            (4,0) ;

      \draw[alert, thick] (6,1.25) -- (9,-.25);
      \draw[alert, thick] (9,1.25) -- (6,-.25);
    \end{tikzpicture}
  \end{alertblock}

  \vspace{-1mm}
  \begin{block}{Remarque}
    \vspace{-1mm}
    On peut renforcer l'abstraction par des propriétés de vivacité :
    \begin{itemize}
    \item \textsc{rb-fifo}.\Broadcast, \textsc{urb-fifo}.\Broadcast
    \end{itemize}
  \end{block}

  
\end{frame}

\endgroup
\endinput
