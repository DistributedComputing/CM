% SPDX-License-Identifier: CC-BY-SA-4.0
% Author: Matthieu Perrin
% Part: Communication
% Section: Cohérence faible
% Frame: Eventual Consistency

\begingroup

\SetKwData{State}{state}
\SetKwFunction{Apply}{apply}
\SetKwFunction{Add}{add}
\SetKwFunction{Get}{get}

\begin{frame}{Implémentation des CRDT}

  \onBlock[top=-4mm]{Théorème -- Implémentation d'un CRDT $\langle C, R, Q, q_0, \tau, \rho \rangle$}{
    L'algorithme ci-dessus implémente eventual consistency.
    \begin{algorithm}[H]
      \Algo{\textsc{op\_based}<$C, R, Q, q_0, \tau, \rho$>}{
        \lLVariables{}{
          \Structure{$\State_i \leftarrow q_0$}
        }
        \Command{$\Apply(c \in C) \in R$}{
          \lIf{$c \in \mathit{Update}$}{
            \Structure{\textsc{rb}.\Broadcast} $\textsc{update}(c)$}%
        }
        \Structure{\Return $\rho(\State_i, c)$}%
      }
      \lWhen{\textsc{rb}.\Deliver $\textsc{update}(c)$ \From $p_j$}{
        \Structure{$\State_i \leftarrow \tau(\State_i, c)$}%
      }
    \end{algorithm}
  }
  
  \onExampleBlock<1>[bottom=-1mm]{Exemple d'exécution}{
    \begin{tikzpicture}[y=8mm]
      \draw[process] (0,4) node[left]{$p_1$} node[replica below] {$0$} to (10,4) node[replica, below left] {$5$}; 
      \draw[process] (0,3) node[left]{$p_2$} node[replica]       {$0$} to (10,3) node[replica, above left] {$5$}; 
      \draw[process] (0,2) node[left]{$p_3$} node[replica]       {$0$} to (10,2) node[replica, above left] {$5$}; 
      \draw[process] (0,1) node[left]{$p_4$} node[replica]       {$0$} to (10,1) node[replica, above left] {$5$}; 
      
      \node[alert,   operation]   (a3) at (4,3)    {$\Add(3)$};
      \draw[alert,   message]     (a3.west) to                (4.5, 4);
      \draw[alert,   message]     (a3.west) to[bend left=30]  (a3.east);
      \draw[alert,   message]     (a3.west) to                (4,   2);
      \draw[alert,   message]     (a3.west) to                (6,   1);
      
      \node[example, operation]   (a2) at (2,2)    {$\Add(2)$};
      \draw[example, message]     (a2.west) to                (3,   4);
      \draw[example, message]     (a2.west) to                (2.5, 3);
      \draw[example, message]     (a2.west) to[bend right=30] (a2.east);
      \draw[example, message]     (a2.west) to                (2.5, 1);

      \node[structure, operation] (g2) at (6,4)    {$\Get() \rightarrow 5$};
      \node[structure, operation] (g3) at (4,1)    {$\Get() \rightarrow 2$};
      \node[structure, operation] (g4) at (8,1)    {$\Get() \rightarrow 5$};
    \end{tikzpicture}
  }

  \obExampleBlock<2>[bottom=-1mm]{Exemple d'exécution}{
    \begin{tikzpicture}[y=8mm]
      \draw[process] (0,4) node[left]{$p_1$} node[replica below] {$0$} to (10,4) node[replica, below left] {$5$}; 
      \draw[process] (0,3) node[left]{$p_2$} node[replica]       {$0$} to (10,3) node[replica, above left] {$5$}; 
      \draw[process] (0,2) node[left]{$p_3$} node[replica]       {$0$} to (10,2) node[replica, above left] {$5$}; 
      \draw[process] (0,1) node[left]{$p_4$} node[replica]       {$0$} to (10,1) node[replica, above left] {$5$}; 
      
      \node[alert,   operation]   (a3) at (2,3)    {$\Add(3)$};
      \draw[alert,   message]     (a3.west) to             (2.5, 4);
      \draw[alert,   message]     (a3.west) to[bend left=30]  (a3.east);
      \draw[alert,   message]     (a3.west) to[bend below=20] (3, 2);
      \draw[alert,   message]     (a3.west) to[out=-40, in=180, distance=15] (4, 2.6) to[out=0, in=90, distance=30] (6, 1);
      
      \node[example, operation]   (a2) at (2,2)    {$\Add(2)$};
      \draw[example, message]     (a2.west) to[out=40, in=180, distance=15] (4, 2.4) to[out=0, in=-90, distance=30] (6, 4);
      \draw[example, message]     (a2.west) to[bend above=20] (3, 3);
      \draw[example, message]     (a2.west) to[bend right=30] (a2.east);
      \draw[example, message]     (a2.west) to             (2.5, 1);

      \node[structure, operation] (g1) at (4,4)    {$\Get() \rightarrow 3$};
      \node[structure, operation] (g2) at (8,4)    {$\Get() \rightarrow 5$};
      \node[structure, operation] (g3) at (4,1)    {$\Get() \rightarrow 2$};
      \node[structure, operation] (g4) at (8,1)    {$\Get() \rightarrow 5$};
    \end{tikzpicture}
  }
  
\end{frame}

\endgroup
\endinput
