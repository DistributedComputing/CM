% SPDX-License-Identifier: CC-BY-SA-4.0
% Author: Matthieu Perrin
% Part: Communication
% Section: Cohérence faible
% Frame: Eventual Consistency

\begingroup

\SetKwFunction{Read}{read}
\SetKwFunction{Write}{write}
\SetKwData{Value}{value}
\SetKwData{Time}{time}
\SetKwData{Writer}{writer}

\begin{frame}{Registre Last-Writer-Wins (LWW)}

  \onAlertBlock<1>[top=-3mm, right=.42\textwidth]{Estampille de Lamport}{
    \vspace{-5mm}
    $$\alert{\langle \Time_i, \Writer_i \rangle}$$
    \vspace{-5mm}
    \begin{description}[$\Writer_i$ :]
    \item[$\Time_i$ :] horloge de Lamport
    \item[$\Writer_i$ :] processus écrivain
    \end{description}

    \begin{itemize}
    \item Compatible avec $\hb$
    \item Identifie chaque écriture
    \item Ordre lexicographique total
    \end{itemize}
  }

  \obBlock<2>[top=-3mm, right=.42\textwidth]{Cohérence séquentielle}{
    Une exécution est \structure{séquentiellement cohérente} si son résultat observable est
    le même que celui d’un entrelacement de son ordre de processus.
  }

  \obBlock<3->[top=-3mm, right=.42\textwidth]{Absence de composabilité}{
    \begin{itemize}
    \item CS n'est pas composable
    \item Quelle cohérence pour la mémoire?
    \end{itemize}
  }
  
  \on[top]{
    \begin{algorithm}[H]
      \LVariables{}{
        $\Value_i  \leftarrow \bot$; \Alert<1>{$\Time_i   \leftarrow 0$; $\Writer_i \leftarrow 0$};
      }
      \lMethod{$\Read()$}{\Return $\Value_i$;}
      \Method{$\Write(v)$}{
        \textsc{rb}.\SBroadcast $\textsc{w}(v, ~\Alert<1>{\Time_i+1})$;
      }
      \When{\textsc{rb}.\Deliver $\textsc{w}(v, t)$ \From $p_j$}{
        \If{\Alert<1>{$\Time_i < t \lor (\Time_i = t \land \Writer_i < j)$}}{
          $\Value_i  \leftarrow v$; \Alert<1>{$\Time_i \leftarrow t$; $\Writer_i \leftarrow j$};
        }
      }
    \end{algorithm}
  }
  
  \obExampleBlock<1>[y=-2mm, anchor=north]{Exemple d'exécution}{
    \centering
    \begin{tikzpicture}[y=10mm]
      \draw[process] (0,2) node[left]{$p_1$} to (10,2);
      \draw[process] (0,1) node[left]{$p_2$} to (10,1);
      \draw[process] (0,0) node[left]{$p_3$} to (10,0);

      \begin{scope}[background]
        \fill[alert!30]     (3,2.1) rectangle ( 6,1.9);
        \fill[structure!30] (6,2.1) rectangle (10,1.9);
        \fill[alert!30]     (2,1.1) rectangle ( 4,0.9);
        \fill[example!30]   (4,1.1) rectangle ( 7,0.9);
        \fill[structure!30] (7,1.1) rectangle (10,0.9);
        \fill[example!30]   (4,0.1) rectangle ( 8,-.1);
        \fill[structure!30] (8,0.1) rectangle (10,-.1);
      \end{scope}
      
      \node[structure, operation]     (wc) at (6,2) {$\Write(c)$};
      \node[structure, below left] at (wc.west) {$\langle 2, 1 \rangle$};
      \path[structure, message]       (wc.west) edge[bend left]  (wc.east);
      \path[structure, message]       (wc.west) edge             (7,1);
      \path[structure, message]       (wc.west) edge             (8,0);
      
      \node[alert, operation]         (wa) at (2,1) {$\Write(a)$};
      \node[alert, below left] at     (wa.west) {$\langle 1, 2 \rangle$};
      \path[alert, message]           (wa.west) edge             (3,2);
      \path[alert, message]           (wa.west) edge[bend left]  (wa.east);
      \path[alert, message]           (wa.west) edge[bend below] (6,0);
      
      \node[example, operation]       (wb) at (4,0) {$\Write(b)$};
      \node[example, above left] at   (wb.west) {$\langle 1, 3 \rangle$};
      \path[example, message]         (wb.west) edge             (8,2);
      \path[example, message]         (wb.west) edge             (4,1);
      \path[example, message]         (wb.west) edge[bend right] (wb.east);
    \end{tikzpicture}
  }

  \onAlertBlock<2>[y=-2mm, anchor=north]{Théorème -- Le registre LWW est séquentiellement cohérent}{
    \begin{itemize}
    \item Soit \structure{$ts(o)$} : \alert{$ts(\Write(v)) = \langle \Time_i+1, i \rangle$} et \alert{$ts(\Read()) = \langle \Time_i, \Writer_i \rangle$}
    \item Soit \structure{$o \lessdot_1 o'$} si \alert{$ts(o) < ts(o')$}
    \item Soit \structure{$o \lessdot_2 o'$} si \alert{$ts(o) = ts(o')$}, \alert{$o=\Write(v)$} et \alert{$o'=\Read()$}
    \item \alert{$\lessdot_1 \cup \lessdot_2 \cup \xrightarrow{po}$} est un ordre partiel strict, on l'étend en un ordre total \structure{$<$}
    \item Chaque lecture retourne la dernière valeur écrite selon $<$
    \end{itemize}
  }

  \obExampleBlock<3>[y=-2mm, anchor=north]{Exemples d'exécution}{
    \centering
    \begin{tikzpicture}[y=10mm]
      \draw[process] (0,1) node[left]{$p_1$} to (10,1);
      \draw[process] (0,0) node[left]{$p_2$} to (10,0);
      
      \node[alert, operation]         (wa) at (3,1) {$x.\Write(a)$};
      \path[alert, message]           (wa.west) edge[bend left]  (wa.east);
      \path[alert, message]           (wa.west) edge[bend below] (8,0);
      \node[structure, operation]     (r1) at (6,1) {$y.\Read() \rightarrow \bot$};
      
      \node[example, operation]       (wb) at (3,0) {$y.\Write(b)$};
      \path[example, message]         (wb.west) edge[bend above] (8,1);
      \path[example, message]         (wb.west) edge[bend right] (wb.east);
      \node[structure, operation]     (r2) at (6,0) {$x.\Read() \rightarrow \bot$};
    \end{tikzpicture}
  }

  \obExampleBlock<4>[y=-2mm, anchor=north]{Exemples d'exécution -- Problème d'ordre FIFO}{
    \centering
    \begin{tikzpicture}[y=10mm]
      \draw[process] (0,1) node[left]{$p_1$} to (10,1);
      \draw[process] (0,0) node[left]{$p_2$} to (10,0);
      
      \node[alert, operation]         (wa) at (1.5,1) {$x.\Write(a)$};
      \path[alert, message]           (wa.west) edge[bend left]  (wa.east);
      \path[alert, message]           (wa.west) edge[bend below] (9.5,0);
      
      \node[example, operation]       (wb) at (4,1) {$y.\Write(b)$};
      \path[example, message]         (wb.west) edge[bend left] (wb.east);
      \path[example, message]         (wb.west) edge (3.5,0);

      \node[structure, operation]     (r1) at (5,0) {$y.\Read() \rightarrow b$};
      \node[structure, operation]     (r2) at (7.75,0) {$x.\Read() \rightarrow \bot$};
    \end{tikzpicture}
  }

  \obExampleBlock<5>[y=-2mm, anchor=north]{Exemples d'exécution -- Problème de causalité}{
    \centering
    \begin{tikzpicture}[y=10mm]
      \draw[process] (0,2) node[left]{$p_1$} to (10,2);
      \draw[process] (0,1) node[left]{$p_2$} to (10,1);
      \draw[process] (0,0) node[left]{$p_3$} to (10,0);
      
      \node[alert, operation]         (wa) at (1.25,1) {$x.\Write(a)$};
      \path[alert, message]           (wa.west) edge (.6,2);
      \path[alert, message]           (wa.west) edge[bend left]  (wa.east);
      \path[alert, message]           (wa.west) edge[bend below] (9.75,0);

      \node[structure, operation]     (r1) at (2,2) {$x.\Read() \rightarrow a$};
      \node[example, operation]       (wb) at (4.5,2) {$y.\Write(b)$};
      \path[example, message]         (wb.west) edge[bend left] (wb.east);
      \path[example, message]         (wb.west) edge            (5,1);
      \path[example, message]         (wb.west) edge            (4,0);

      \node[structure, operation]     (r2) at (5.5,0) {$y.\Read() \rightarrow b$};
      \node[structure, operation]     (r3) at (8.1,0) {$x.\Read() \rightarrow \bot$};
    \end{tikzpicture}
  }

\end{frame}

\endgroup
\endinput
