% SPDX-License-Identifier: CC-BY-SA-4.0
% Author: Matthieu Perrin
% Part: 
% Section: 
% Sub-section: 
% Frame: 

\begingroup

\SetKwData{Delivered}{delivered}

\begin{frame}{Implémentation de Uniform Reliable Broadcast}
  
  \on[y=15mm]{
    \begin{algorithm}[H]
      \lLVariables{}{$\Delivered_i \leftarrow \emptyset$}
      \Method{\textsc{urb}.\Broadcast$(m)$}{
        \nl\textsc{sta}.\Broadcast $\textsc{msg}(m, i)$\;
      }
      
      \When{\textsc{sta}.\Deliver $\textsc{msg}(m, j)$ \From $p_k$}{
        \nl\If{$m \notin \Delivered_i$}{
          \nl\Alert<4-5>{\textsc{sta}.\Broadcast $\textsc{msg}(m, j)$}\;
          \nl\Alert<4-5>{\textsc{urb}.\Deliver $m$ \From $p_j$}\;
          \nl$\Delivered_i \leftarrow \Delivered_i \cup \{m\}$
        }
      }
    \end{algorithm}
  }

  \obExampleBlock<-3>[anchor=north]{Exemples d'exécution}{}
  
  \ob<1>[y=-25mm]{
    \begin{tikzpicture}[y=8mm]
      \draw[process] (0,2) node[left]{$p_1$} to (10,2);
      \draw[process] (0,1) node[left]{$p_2$} to (10,1);
      \draw[process] (0,0) node[left]{$p_3$} to (10,0);

      \node[alert,     operation, text width=10mm] (o1) at (2,2) {};
      \node[structure, operation, text width=10mm] (o2) at (4,1) {};
      \node[example,   operation, text width=10mm] (o3) at (6,0) {};

      \path[alert,     message]   ([xshift=0mm]o1.west)  edge[bend left] ([xshift=4mm]o1.west);
      \path[alert,     message]   ([xshift=6mm]o1.west)  edge            (o2.west);
      \path[alert,     message]   ([xshift=8mm]o1.west)  edge[bend below,out=-20] (8 ,0);

      \path[structure, message]   ([xshift=2mm]o2.west)  edge            (5 ,2);
      \path[structure, message]   ([xshift=4mm]o2.west)  edge[bend left] ([xshift=8mm]o2.west);
      \path[structure, message]   ([xshift=10mm]o2.west) edge            (o3.west);
      
      \path[example,   message]   ([xshift=2mm]o3.west)  edge            (7,2);
      \path[example,   message]   ([xshift=4mm]o3.west)  edge            (8,1);
      \path[example,   message]   ([xshift=6mm]o3.west)  edge[bend left] ([xshift=10mm]o3.west);
    \end{tikzpicture}
  }

  \ob<2>[y=-25mm]{
    \begin{tikzpicture}[y=8mm]
      \draw[crashed] (0,2) node[left]{$p_1$} to (2.5,2);
      \draw[process] (0,1) node[left]{$p_2$} to (10,1);
      \draw[process] (0,0) node[left]{$p_3$} to (10,0);

      \node[alert,     operation, text width=10mm] (o1) at (2,2) {};
      \node[structure, operation, text width=10mm] (o2) at (4,1) {};
      \node[example,   operation, text width=10mm] (o3) at (6,0) {};

      \path[alert,     message]   ([xshift=0mm]o1.west)  edge[bend left] ([xshift=4mm]o1.west);
      \path[alert,     message]   ([xshift=6mm]o1.west)  edge            (o2.west);

      \path[structure, message]   ([xshift=4mm]o2.west)  edge[bend left] ([xshift=8mm]o2.west);
      \path[structure, message]   ([xshift=10mm]o2.west) edge            (o3.west);
      
      \path[example,   message]   ([xshift=4mm]o3.west)  edge            (8,1);
      \path[example,   message]   ([xshift=6mm]o3.west)  edge[bend left] ([xshift=10mm]o3.west);
    \end{tikzpicture}
  }

  \ob<3>[y=-25mm]{
    \begin{tikzpicture}[y=8mm]
      \draw[crashed] (0,2) node[left]{$p_1$} to (2.5,2);
      \draw[crashed] (0,1) node[left]{$p_2$} to (4.5,1);
      \draw[process] (0,0) node[left]{$p_3$} to (10,0);

      \node[alert,     operation, text width=10mm] (o1) at (2,2) {};
      \node[structure, operation, text width=10mm] (o2) at (4,1) {};
      \node[operation, text width=10mm, draw=none, fill=none] (o3) at (6,0) {};

      \path[alert,     message]   ([xshift=0mm]o1.west)  edge[bend left] ([xshift=4mm]o1.west);
      \path[alert,     message]   ([xshift=6mm]o1.west)  edge            (o2.west);

      \path[structure, message]   ([xshift=4mm]o2.west)  edge[bend left] ([xshift=8mm]o2.west);
      
    \end{tikzpicture}
  }

  \obBlock<4>[anchor=north]{Démonstration : URB-Validité}{
    \begin{center}
      \structure{Si $p_i$ \textsc{urb}.\Deliver $m$ de $p_j$, $p_j$ a \textsc{urb}.\Broadcast $m$}
    \end{center}
    \begin{description}[STA-Terminaison :]
    \item[Ligne 4 :] $p_i$ \textsc{sta}.\Deliver $\textsc{msg}(m, j)$ de $p_k$
    \item[STA-Validité :] $p_k$ \textsc{sta}.\Broadcast $\textsc{msg}(m, j)$
    \item[HB bien-fondé :] il y a un premier \textsc{sta}.\Broadcast $\textsc{msg}(m, j)$
    \item[Lignes 1 et 3 :] ce premier diffuseur est $p_j$ qui a donc \textsc{urb}.\Broadcast $m$
    \end{description}
  }

  \obBlock<5>[anchor=north]{Démonstration : URB-Intégrité}{
    \begin{center}
      \structure{$p_i$ \textsc{urb}.\Deliver $m$ au plus une fois}
    \end{center}
    \begin{description}[STA-Terminaison :]
    \item[Invariant :] $\Delivered_i$ croissant selon $\subseteq$
    \item[Ligne 2 :] avant livraison,  $m\notin\Delivered_i$
    \item[Ligne 5 :] après livraison, $m\in\Delivered_i$
    \end{description}
  }

  \obBlock<6>[anchor=north]{Démonstration : URB-Terminaison}{
    \begin{center}
      \structure{Si un correct $p_i$ \textsc{urb}.\Broadcast $m$, $p_i$ \textsc{urb}.\Deliver $m$}
    \end{center}
    \begin{description}[STA-Terminaison :]
    \item[Ligne 1 :] $p_i$ \textsc{sta}.\Broadcast $\textsc{msg}(m, i)$
    \item[Fiabilité :] $p_i$ \textsc{sta}.\Deliver $\textsc{msg}(m, i)$
    \item[HB bien-fondé :] il y a un premier \textsc{sta}.\Deliver $\textsc{msg}(m, i)$ par $p_i$
    \item[Lignes 2-5 :] $p_i$ \textsc{urb}.\Deliver $m$
    \end{description}
  }

  \obBlock<7>[anchor=north]{Démonstration : URB-Fiabilité}{
    \begin{center}
      \structure{Si $p_i$ \textsc{urb}.\Deliver $m$, tous les corrects $p_k$ \textsc{urb}.\Deliver $m$}
    \end{center}
    \begin{description}[STA-Terminaison :]
    \item[Lignes 3 et 4 :] $p_i$ \textsc{sta}.\Broadcast $\textsc{msg}(m, j)$
    \item[Fiabilité :] $p_k$ \textsc{sta}.\Deliver $\textsc{msg}(m, j)$
    \item[HB bien-fondé :] il y a un premier \textsc{sta}.\Deliver $\textsc{msg}(m, j)$ par $p_k$
    \item[Lignes 2-5 :] $p_k$ \textsc{urb}.\Deliver $m$
    \end{description}
  }

  \obAlertBlock<8>[anchor=north]{Peut-on inverser les lignes 3 et 4 ?}{
    \alert{URB-fiabilité :} Si $p_i$ \textsc{urb}.\Deliver $m$, tous les corrects \textsc{urb}.\Deliver $m$.
  }

  \onBlock<9>[anchor=north]{Spécification de Reliable Broadcast}{
    \begin{description}[RB-Terminaison :]
    \item[RB-Validité :]    Si $p_i$ \textsc{rb}.\Deliver $m$ de $p_j$, $p_j$ a \textsc{rb}.\Broadcast $m$
    \item[RB-Intégrité :]   $p_i$ \textsc{rb}.\Deliver $m$ au plus une fois
    \item[RB-Terminaison :] Si un correct $p_i$ \textsc{rb}.\Broadcast $m$, $p_i$ \textsc{rb}.\Deliver $m$.
    \item[\alert{RB-Fiabilité :}]   Si \alert{un correct} $p_i$ \textsc{rb}.\Deliver $m$, tous les corrects \textsc{rb}.\Deliver $m$.
    \end{description}
  }
  
\end{frame}

\endgroup
\endinput
