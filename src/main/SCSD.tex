% SPDX-License-Identifier: CC-BY-SA-4.0
% Author: Matthieu Perrin

\ifdefined\HANDOUT
  \documentclass[9pt, handout]{beamer}
  \usetheme{Nantes}
\else
  \documentclass[9pt]{beamer}
  \usetheme[sectionpage]{Nantes}
\fi

\usepackage{src/sty/config}

\title[Services de communication et systèmes distribués]{Services de communication \\et systèmes distribués}

\author[Matthieu Perrin]{
  Matthieu Perrin\\
  Laboratoire des Sciences du Numérique de Nantes \\
  UMR CNRS 6004\\
  Bureau 410, bâtiment 34 \\
  \url{matthieu.perrin@univ-nantes.fr}\\
}

\date{
  Nantes Université\\
  Master en Informatique, deuxième année\\
  Parcours ALMA\\
  2024-2025
}


\hypersetup{
  pdftitle  = {\insertshorttitle},
  pdfauthor = {\insertshortauthor},
  pdfsubject  = {Cours de M2 sur la réplication et la tolérance aux fautes},
  pdfkeywords = {cohérence, consensus, fiabilité, machine à états répliquée, pannes franches, réplication}
}

\begin{document}

\begin{frame}[plain,noframenumbering]
  \titlepage
  \on[text, bottom=-5mm]{\scriptsize
    \begin{description}
    \item[Licence :] \href{https://creativecommons.org/licenses/by-sa/4.0/}{CC BY-SA 4.0} (\ccbysa{}) — Matthieu Perrin
    \item[Code source :]\vspace{-.5mm}  \url{https://github.com/DistributedComputing}
    \end{description}
  }
\end{frame}


\part{Introduction}
 
 
\section{Généralités}
 
\subsection{Positionnement dans le master}
\input{introduction/generalities/master/master}
 
\subsection{Bibliographie}
% SPDX-License-Identifier: CC-BY-SA-4.0
% Author: Matthieu Perrin
% Part: 
% Section: 
% Sub-section: 
% Frame: 

\begingroup

\begin{frame}{Bibliographie}

  \begin{block}{Ouvrages}
    \begin{itemize}
    \item
      \textit{Distributed Algorithms for Message-Passing Systems}\\
      Michel Raynal (Springer 2013)
    \item
      \textit{Fault-Tolerant Message-Passing Distributed Systems}\\
      \textit{An Algorithmic Approach}\\
      Michel Raynal (Springer 2017)
    \item
      \textit{Distributed Systems}\\
      \textit{Concurrency and Consistency}\\
      Matthieu Perrin (ISTE Press \& Elsevier 2017)
    \end{itemize}
  \end{block}
  
  \footnoteref{Les articles et ouvrages sont accessibles sur Madoc.}
  
\end{frame}

\endgroup
\endinput

 
\section{Machine à états répliquée}
 
\subsection{Modèle clients/serveur}
% SPDX-License-Identifier: CC-BY-SA-4.0
% Author: Matthieu Perrin
% Part: 
% Section: 
% Sub-section: 
% Frame: 

\begingroup

\SetKwFunction{Acheter}{acheter}
\SetKwFunction{Vendre}{vendre}
\SetKwFunction{Consulter}{consulter}

\begin{frame}{Modèle clients/serveur}

  \onBlock[top=-2mm]{Exemple --  Site de vente en ligne}{
    On veut implémenter \example{AlmaZone}, qui permet de :
    \begin{itemize}
    \item \structure{\Consulter} les articles en vente
    \item \structure{\Vendre} un article (le mettre en vente)
    \item \structure{\Acheter} un article en vente
    \end{itemize}
  }

  \onBlock<2->[y=-2mm]{Solution clients/serveur}{
    \begin{itemize}
    \item \structure{Un serveur :} maintient l'état (ensemble d'articles)
    \item \structure{Plusieurs clients :} interagissent par messages (requête / réponse)
    \end{itemize}
  }

  \on<2->[y=-27mm]{
    \begin{tikzpicture}[y=8mm]
      \draw[process] (.5,2) node[left]{$\mathit{Client}_1$} to (9.5,2);
      \draw[process] (.5,1) node[left]{$\mathit{Serveur}$} node[replica]{$\{a,b\}$} to (9.5,1) node[replica, above left]{$\{b\}$};
      \draw[process] (.5,0) node[left]{$\mathit{Client}_2$} to (9.5,0);

      \begin{scope}[alert]
        \node[operation] (o1) at (3,0) {$\Consulter() \rightarrow \{a, b\}$};
        \node[event]     (s1) at (3,1) {};
        \draw[message]   (o1.west) -- (s1.west);
        \draw[message]   (s1.east) -- (o1.east);
      \end{scope}

      \begin{scope}[structure]
        \node[operation] (o2) at (5,2) {$\Acheter(a) \rightarrow \cmark$};
        \node[event]     (s2) at (5,1) {};
        \draw[message]   (o2.west) -- (s2.west);
        \draw[message]   (s2.east) -- (o2.east);
      \end{scope}

      \begin{scope}[example]
        \node[operation] (o3) at (7,0) {$\Consulter() \rightarrow \{b\}$};
        \node[event]     (s3) at (7,1) {};
        \draw[message]   (o3.west) -- (s3.west);
        \draw[message]   (s3.east) -- (o3.east);
      \end{scope}
    \end{tikzpicture}
  }
  
\end{frame}

\endgroup
\endinput

% SPDX-License-Identifier: CC-BY-SA-4.0
% Author: Matthieu Perrin
% Part: 
% Section: 
% Sub-section: 
% Frame: 

\begingroup

\SetKwFunction{Acheter}{acheter}
\SetKwFunction{Vendre}{vendre}
\SetKwFunction{Consulter}{consulter}

\begin{frame}{Abstraction du serveur Web}

  \begin{block}{Définition -- Machine à états}
    \vspace{3mm}
    Une \structure{machine à états} est un sextuplet \alert{$\langle C, R, Q, q_0, \tau, \rho \rangle$} tel que :
    \begin{description}[xxxxx]
    \item[\alert{$C$}] : ensemble de \structure{commandes} 
      \hfill {\footnotesize \example{$C =\{\Acheter(x), \Vendre(x), \Consulter \mid x \in X\}$}}
    \item[\alert{$R$}] : ensemble de \structure{réponses} possibles
      \hfill {\footnotesize \example{$R = \mathcal{P}(X) \cup \{\cmark, \xmark\}$}}
    \item[\alert{$Q$}] : ensemble d'\structure{états} (fini ou infini)
      \hfill {\footnotesize \example{$Q = \mathcal{P}(X)$}}
    \item[\alert{$q_0$}] $\in Q$ : l'\structure{état initial}
      \hfill {\footnotesize \example{$q_0 = \emptyset$}}
    \item[\alert{$\tau$}] $: Q \times C \rightarrow Q$ : la \structure{fonction de transition}
      \hfill {\footnotesize\example{$\tau(q, \Acheter(x)) = q \setminus \{x\}$, ...}}
    \item[\alert{$\rho$}] $: Q \times C \rightarrow R$ : la \structure{fonction de réponse}
      \hfill {\footnotesize\example{$\rho(q, \Consulter) = q$, ...}}
    \end{description}
  \end{block}

  \begin{block}{Remarques}
    \begin{itemize}
    \item Une commande de $C$ est formée d'une \structure{méthode} et d'\structure{arguments}
    \item Les ensembles $C$, $R$ et $Q$ peuvent être finis ou infinis
    \item On appelle \structure{opération} un couple $o = \langle c, r\rangle \in C \times R$, noté $c \rightarrow r$
    \item Les machines à états sont complètes et déterministes
    \end{itemize}
  \end{block}
  
  \on[y=-15mm, x=00mm]{
    \begin{tikzpicture}[automaton, x=25mm]
      \state            (q)  at (0,0) {$q$}; 
      \state[rectangle] (q1) at (1,0) {$\tau(q, c)$}; 
      \path             (q)  edge node {$c \rightarrow \rho(q, c)$} (q1);
    \end{tikzpicture}
  }
  
\end{frame}

\endgroup
\endinput



 
\subsection{Réplication des données}
% SPDX-License-Identifier: CC-BY-SA-4.0
% Author: Matthieu Perrin
% Part: 
% Section: 
% Sub-section: 
% Frame: 

\begingroup

\SetKwFunction{Acheter}{acheter}
\SetKwFunction{Vendre}{vendre}
\SetKwFunction{Consulter}{consulter}

\begin{frame}{Nécessité de la réplication de données}

  \begin{block}{Limites du modèle clients/serveur}
    \begin{description}[Tolérance aux pannes]
    \item[Passage à l'échelle :] un seul n\oe ud sature
    \item[Disponibilité :] service indisponible en cas de panne
    \item[Latence :] clients géographiquement éloignés
    \item[Maintenance :] interruptions en cas de mises à jour 
    \end{description}
  \end{block}

  \pause
  
  \begin{block}{Solution : \alert{répliquer} l’état du serveur}
    \centering
    \begin{tikzpicture}[y=5mm]
      \draw[process] (.5,5) node[left]{$\mathit{Client}_1$}  to (9.5,5);
      \draw[process] (.5,3) node[left]{$\mathit{Serveur}_1$} to (9.5,3);
      \draw[crashed] (.5,2) node[left]{$\mathit{Serveur}_2$} to (6,2);
      \draw[process] (.5,0) node[left]{$\mathit{Client}_2$}  to (9.5,0);

      \begin{scope}[alert]
        \node[operation] (o1) at (3,0) {$\Consulter() \rightarrow \{a, b\}$};
        \node[event]     (s1) at (3,2) {};
        \draw[message]   (o1.west) -- (s1.west);
        \draw[message]   (s1.east) -- (o1.east);
        \draw[double]    (3.5,1.75) -- (3.5,3.25);
      \end{scope}

      \begin{scope}[structure]
        \node[operation] (o2) at (5,5) {$\Acheter(a) \rightarrow \cmark$};
        \node[event]     (s2) at (5,3) {};
        \draw[message]   (o2.west) -- (s2.west);
        \draw[message]   (s2.east) -- (o2.east);
        \draw[double]    (5.5,1.75) -- (5.5,3.25);
      \end{scope}

      \begin{scope}[example]
        \node[operation] (o3) at (7,0) {$\Consulter() \rightarrow \{b\}$};
        \node[event]     (s3) at (7,3) {};
        \draw[message]   (o3.west) -- (s3.west);
        \draw[message]   (s3.east) -- (o3.east);
        \draw[double]    (7.5,2.75) -- (7.5,3.25);
      \end{scope}
      
      \node[below] at (1,2) {$\{a,b\}$};
      \node[above] at (1,3) {$\{a,b\}$};
      \node[above] at (9,3) {$\{b\}$};
    \end{tikzpicture}
  \end{block}
  
\end{frame}

\endgroup
\endinput



% SPDX-License-Identifier: CC-BY-SA-4.0
% Author: Matthieu Perrin
% Part: 
% Section: 
% Sub-section: 
% Frame: 

\begingroup

\SetKwFunction{Acheter}{acheter}
\SetKwFunction{Vendre}{vendre}
\SetKwFunction{Consulter}{consulter}
\SetKwFunction{Apply}{apply}

\begin{frame}{Machine à états répliquée}

  \vspace{-1mm}
  \begin{block}{Définition -- Machine à états répliquée}
    Algorithme \structure{universel} qui simule une machine à états $\langle C,R,Q,q_0,\tau,\rho\rangle$

    \vspace{2mm}
    \begin{algorithm}[H]
      \Interface{$\mathit{SMR}$<$C,R,Q,q_0,\tau,\rho$>}{
        \lMethod{$\Apply(c \in C) \in R$ }{\tcp*[f]{exécute la commande $c$}}
      }
    \end{algorithm}
  \end{block}

  \vspace{-1mm}
  \begin{exampleblock}{Utilisation pour un site Web}
  \begin{tikzpicture}[y=8mm, x=9mm]
    \draw[fade on=<2>, process] (0,3) node[left]{$\mathit{Client}_1$}  to (10,3);
    \draw[             process] (0,2) node[left]{$\mathit{Serveur}_1$} to (10,2);
    \draw[             process] (0,1) node[left]{$\mathit{Serveur}_2$} to (10,1);
    \draw[fade on=<2>, process] (0,0) node[left]{$\mathit{Client}_2$}  to (10,0);

    \begin{scope}[structure]
      \node[fade on=<2>, operation] (o1) at (2.5,3) {\scriptsize\hspace{7mm}$\Consulter() \rightarrow \{a, b\}$\hspace{7mm}};
      \node[             operation] (s1) at (2.5,2) {\scriptsize$\Apply(\Consulter()) \rightarrow \{a, b\}$};
      \draw[fade on=<2>, message]   (o1.west) -- (s1.west);
      \draw[fade on=<2>, message]   (s1.east) -- (o1.east);
    \end{scope}
    
    \begin{scope}[alert]
      \node[fade on=<2>, operation] (o1) at (7.5,3) {\scriptsize\hspace{7mm}$\Consulter() \rightarrow \{b\}$\hspace{7mm}};
      \node[             operation] (s1) at (7.5,2) {\scriptsize$\Apply(\Consulter()) \rightarrow \{b\}$};
      \draw[fade on=<2>, message]   (o1.west) -- (s1.west);
      \draw[fade on=<2>, message]   (s1.east) -- (o1.east);
    \end{scope}

    \begin{scope}[example]
      \node[fade on=<2>, operation] (o2) at (5,0) {\scriptsize\hspace{7mm}$\Acheter(a) \rightarrow \cmark$\hspace{7mm}};
      \node[             operation] (s2) at (5,1) {\scriptsize$\Apply(\Acheter(a)) \rightarrow \cmark$};
      \draw[fade on=<2>, message]   (o2.west) -- (s2.west);
      \draw[fade on=<2>, message]   (s2.east) -- (o2.east);
    \end{scope}
  \end{tikzpicture}
  \end{exampleblock}

  \vspace{-1mm}
  \begin{block}{But}
    \centering
    \vspace{-3mm}
    \alert{Comment mettre en place la réplication \\ de manière transparente pour les clients ?}
  \end{block}
  
  \footnoteref{L. Lamport. \textit{The Implementation of Reliable Distributed Multiprocess Systems.} Computer Networks (1978)}

\end{frame}

\endgroup
\endinput

% SPDX-License-Identifier: CC-BY-SA-4.0
% Author: Matthieu Perrin
% Part: 
% Section: 
% Sub-section: 
% Frame: 

\begingroup

\begin{frame}{Différentes stratégies de réplication}

  \vspace{-3mm}
  \uncover{\hfill\alert{Ce cours :}}
  \begin{itemize}
  \item \structure{Placement}  \hfill\alert{pas de différence algorithmique}
    \begin{description}[mono-cloud :]
    \item[intra-site :] entre n\oe uds d'un même site
    \item[géo :] data centers de sites/régions différentes
    \end{description}

  \item<2->\vspace{1mm} \structure{Contrôle}  \hfill\alert{pas de différence algorithmique}
    \begin{description}[mono-cloud :]
    \item[mono-cloud :] un fournisseur, même écosystème
    \item[multi-cloud :] plusieurs fournisseurs, interopérabilité à gérer
    \end{description}

  \item<3->\vspace{1mm} \structure{Mode} \hfill\alert{réplication active car plus sûre}
    \begin{description}[mono-cloud :]
    \item[active :] toutes les répliques traitent les requêtes
    \item[passive :] un \structure{primaire} traite, des \structure{secondaires} recopient
    \end{description}

  \item<4->\vspace{1mm} \structure{Couverture} \hfill\alert{réplication totale car plus simple}
    \begin{description}[mono-cloud :]
    \item[totale :] chaque réplique stocke tout l’état
    \item[partielle :] données découpées en fragments (shards)
    \end{description}

  \item<5->\vspace{1mm} \structure{Cohérence}  \hfill\alert{progression par étapes : faible puis forte}
    \begin{description}[mono-cloud :]
    \item[forte :] sûreté équivalente à un serveur unique
    \item[faible :] convergence, vues temporaires divergentes
    \end{description}

  \item<6->\vspace{1mm} \structure{Tolérance aux fautes} \hfill\alert{crash car plus simple}
    \begin{description}[mono-cloud :]
    \item[crash :] un processus s'arrête, ne répond plus
    \item[byzantin :] un processus agit arbitrairement (erreurs ou malveillance)
    \end{description}
  \end{itemize}

\end{frame}

\endgroup
\endinput



 
\section{Système asynchrone à passage de messages}
 
\subsection{Modèle de calcul}
% SPDX-License-Identifier: CC-BY-SA-4.0
% Author: Matthieu Perrin
% Part: 
% Section: 
% Sub-section: 
% Frame: 

\begingroup

\begin{frame}{Processus}

  \begin{block}{Système réparti fermé}
    \begin{itemize}
    \item Ensemble fixe et connu de $n$ processus \structure{$p_1$, ..., $p_n$}
    \item Chaque $p_i$ connaît son \structure{identifiant} $i$ (et $n$ si besoin)
    \item Mémoire locale \structure{privée} pour chaque processus
    \end{itemize}
  \end{block}

  \begin{block}{Processus séquentiel, asynchrone}
    \begin{itemize}
      \item \structure{Séquentiel} : une seule action à la fois (pas de threads internes)
      \item \structure{Moniteur} : \Wait{condition} bloque jusqu'à \textit{condition}
      \item \structure{Asynchrone} : vitesses arbitraires, pas d’horloge globale
    \end{itemize}
  \end{block}

  \begin{block}{Pannes (\textit{crash-stop})}
    \begin{itemize}
      \item \structure{En panne} : le processus s’arrête définitivement
      \item \structure{Correct} : ne crash pas (prend une infinité d’étapes)
      \item Au plus \structure{$t$} pannes dans une exécution
    \end{itemize}
  \end{block}

\end{frame}

\endgroup
\endinput



% SPDX-License-Identifier: CC-BY-SA-4.0
% Author: Matthieu Perrin
% Part: 
% Section: 
% Sub-section: 
% Frame: 

\begingroup

\begin{frame}{Communication}

  \vspace{-2mm}
  \begin{block}{Interface de communication}
    Les processus communiquent en s'envoyant des messages sur un réseau
    \begin{description}[Réception :]
      \item[Émission :]  un processus $p_i$ peut \structure{envoyer} un message $m$ à $p_j$ en appelant :
      \begin{algorithm}[H]
        \Send $m$ \To $p_j$; \tcp{exécuté par $p_i$}
      \end{algorithm}
      \item[Réception :] plus tard, un \structure{événement de réception} de $m$ est généré chez $p_j$ :
        \begin{algorithm}[H]
          \lWhen{\Receive $m$ \From $p_i$}{\tcp*[h]{code exécuté par $p_j$}}
      \end{algorithm}
    \end{description}
    Pour simplifier, un message à soi-même est reçu immédiatement. 
  \end{block}
  
  \vspace{-2mm}
  \begin{block}{Hypothèse -- Canaux fiables asynchrones}
    \begin{description}[Réception :]
    \item[Validité :] si $p_i$ reçoit $m$ de $p_j$, $p_j$ a envoyé $m$ à $p_i$
    \item[Intégrité :] chaque processus $p_i$ reçoit $m$ au plus une fois
    \item[Fiabilité :] si un processus correct $p_i$ envoie $m$ à un processus correct $p_j$, \\
      $p_j$ reçoit $m$ de $p_i$
    \end{description}
    \alert{Asynchronisme :} ni borne sur la durée de transfert, ni ordre sur la réception
  \end{block}
  
\end{frame}

\endgroup
\endinput



% SPDX-License-Identifier: CC-BY-SA-4.0
% Author: Matthieu Perrin
% Part: 
% Section: 
% Sub-section: 
% Frame: 

\begingroup

\begin{frame}{Hypothèses supplémentaires}
  
  \begin{block}{Résumé -- le modèle $\mathcal{CAMP}_{n,t}[H]$}
    \begin{description}
    \item[$\mathcal{CAMP}$ :] Crash-prone Asynchronous Message-Passing
    \item[$n$ :] nombre total de \structure{processus} (ensemble fixe et connu)
    \item[$t$ :] nombre maximal de pannes \structure{crash-stop}
    \item[$H$ :] ensemble d'\structure{hypothèses supplémentaires}. 
    \end{description}
  \end{block}
  
  Si rien n'est spécifié, on utilisera le modèle \alert{$\mathcal{CAMP}_{n,t}[\emptyset]$}.

  \vspace{4mm}
  
  \begin{exampleblock}{Exemples d'hypothèses supplémentaires}
    \begin{itemize}
    \item Borne \alert{$t < \frac{n}{k}$} sur le nombre de pannes
      \begin{itemize}
      \item Si l'hypothèse n'est pas vérifiée, la vivacité n'est plus garantie
      \end{itemize}
    \item Présence d'\alert{objets déjà disponibles}
      \begin{itemize}
      \item Utile pour les réductions : peut-on implémenter $B$ en utilisant $A$ ? 
      \end{itemize}
    \item Synchronie partielle, capacités cryptographiques...
    \end{itemize}
  \end{exampleblock}

\end{frame}

\endgroup
\endinput


% SPDX-License-Identifier: CC-BY-SA-4.0
% Author: Matthieu Perrin
% Part: 
% Section: 
% Sub-section: 
% Frame: 

\begingroup

\begin{frame}{Exemple}

  \on{
    \begin{tikzpicture}[y=20mm]

      \uncoverb<2,4>{
        \draw (4.5, -1) node{Motif de messages MP2};
      }
      \uncoverb<3>{
        \draw (4.5, -1) node{Motif de messages MP3};
      }
      
      \draw[process, fade ob=<4>] (0,2) node[left]{$p_1$} to (10,2);
      \draw[process, fade ob=<3>] (0,1) node[left]{$p_2$} to (10,1);
      \draw[process, fade ob=<2>] (0,0) node[left]{$p_3$} to (10,0);
      
      \draw[alert, fade ob=<4>] (1,2) node (m1) {} node[below left]{$m_1$};
      \path[message, alert,     fade ob=<{3,4}>, thick] (m1.center) edge             (6,1); 
      \path[message, alert,     fade ob=<{2,4}>, thick] (m1.center) edge             (5,0); 
      \path[message, alert,     fade ob=<{4}>  , thick] (m1.center) edge[bend left]  (7,2); 

      \draw[example, fade ob=<3>] (1,1) node (m2) {} node[below left]{$m_2$};
      \path[message, example,   fade ob=<{3,4}>, thick] (m2.center) edge             (3,2); 
      \path[message, example,   fade ob=<{2,3}>, thick] (m2.center) edge             (7,0); 
      \path[message, example,   fade ob=<{3}>  , thick] (m2.center) edge[bend left]  (4,1); 

      \draw[structure, fade ob=<2>] (1,0) node (m3) {} node[above left]{$m_3$};
      \path[message, structure, fade ob=<{2,4}>, thick] (m3.center) edge             (5,2); 
      \path[message, structure, fade ob=<{2,3}>, thick] (m3.center) edge             (8,1); 
      \path[message, structure, fade ob=<{2}>  , thick] (m3.center) edge[bend right] (9,0); 
    \end{tikzpicture}
  }

\end{frame}

\endgroup
\endinput

  
\subsection{Exécutions réparties}
% SPDX-License-Identifier: CC-BY-SA-4.0
% Author: Matthieu Perrin
% Part: 
% Section: 
% Sub-section: 
% Frame: 

\begingroup

\begin{frame}{Exécutions concrètes d'un algorithme}

  Exécution concrète = chemin dans un système de transitions.

  \begin{block}{Configuration globale du système}
    La \alert{configuration globale} est définie par un tuple \alert{$\langle q_1, ..., q_i, ..., q_n, M \rangle$} :
    \begin{itemize}
    \item L'état local \alert{$q_i$} de chaque processus $p_i$ contenant :
      \begin{description}
      \item[Pile d'appels :] la valeur de ses variables locales
      \item[Avancement :] la prochaine instruction à exécuter
      \end{description}
    \item L'ensemble \alert{$M$} des messages en cours d'acheminement
    \end{itemize}
  \end{block}

  \begin{block}{Transitions}
    Un \alert{ordonnanceur} (non-déterministe) décide le prochain pas parmi :
    \begin{description}[Réception]
    \item[Pas local] de $p_i$ (accès aux variables locales, appel de fonctions, etc)
      \begin{itemize}
      \item $\structure{\langle q_1, ..., \alert{q_i}, ..., q_n, M \rangle \leadsto \langle q_1, ..., \alert{q_i'}, ..., q_n, M \rangle}$
      \end{itemize}
    \item[Envoi] d'un message $m$ par $p_i$ à $p_j$
      \begin{itemize}
      \item $\structure{\langle q_1, ..., \alert{q_i}, ..., q_n, \alert{M} \rangle \leadsto \langle q_1, ..., \alert{q_i'}, ..., q_n, \alert{M \cup \{\langle p_i, p_j, m \rangle\}} \rangle}$
      \end{itemize}
    \item[Réception] d'un message $m$ par $p_j$ de la part de $p_i$ 
      \begin{itemize}
      \item $\structure{\langle q_1, ..., \alert{q_j}, ..., q_n, \alert{M} \rangle \leadsto \langle q_1, ..., \alert{q_j'}, ..., q_n, \alert{M \setminus \{\langle p_i, p_j, m \rangle\}} \rangle}$
      \end{itemize}
    \end{description}
  \end{block}

\end{frame}

\endgroup
\endinput

% SPDX-License-Identifier: CC-BY-SA-4.0
% Author: Matthieu Perrin
% Part: 
% Section: 
% Sub-section: 
% Frame: 

\begingroup

\tikzset{
  sshape/.style={
    text width=19mm,
    rectangle,
    rounded corners,
    fill=black!10,
  },
}

\newcommand\scontent[4]{
  \begin{tabular}{@{}r@{}c@{~}l@{}}
    \tiny $p_1$  & \tiny : & \tiny \texttt{#1} \\[-.5mm]
    \tiny $p_2$  & \tiny : & \tiny \texttt{#2} \\[-.5mm]
    \tiny $\X_2$ & \tiny : & \tiny \texttt{#3} \\[-.5mm]
    \tiny $M$    & \tiny : & \tiny \texttt{#4}
  \end{tabular}
}

\SetKwData{Done}{done}
\SetKwData{X}{value}

\begin{frame}{Système de transitions}

  \on[top]{
    \begin{algorithm}[H]
      \LVariables{}{
        $\Done_i \leftarrow \False$\;
        $\X_i \leftarrow 0$\;
      }
      \Process{$p_1$}{
        \nl \Send $m$ \To $p_2$\;
      }
      \Process{$p_2$}{
        \nl $\X_2 \leftarrow 1$\;
        \nl \Wait $\Done_2$\;
        \nl$\Print(\X_2)$\;
      }
      \When{$p_2$ \Receives $m$ \From $p_1$}{
        $\X_2 \leftarrow 2$\; 
        $\Done_2 \leftarrow \True$\;
      }
    \end{algorithm}
  }
  
  \on[x=.25\textwidth]{
    \begin{tikzpicture}[automaton, x=15mm, y=17mm]
      \state[structure on=<2->,  sshape, initial]   (00) at (1,4) {\scontent{\texttt{start}} {\texttt{start}} {0} {$\emptyset$}};

      \state[structure on=<2>,   sshape]            (11) at (0,3) {\scontent{\texttt{term}}  {\texttt{start}} {0} {$\{\langle p_1, p_2, m \rangle\}$}};
      \state[structure on=<2>,   sshape]            (12) at (0,2) {\scontent{\texttt{term}}  {\texttt{start}} {2} {$\emptyset$}};
      \state[structure on=<2>,   sshape]            (13) at (0,1) {\scontent{\texttt{term}}  {\texttt{L3}}    {1} {$\emptyset$}};
      \state[structure on=<2>,   sshape, accepting] (14) at (0,0) {\scontent{\texttt{term}}  {\texttt{term}}  {1} {$\emptyset$}};

      \state[structure ob=<3-4>, sshape]            (21) at (2,3) {\scontent{\texttt{start}} {\texttt{L3}}    {1} {$\emptyset$}};
      \state[structure ob=<3>,   sshape]            (22) at (2,2) {\scontent{\texttt{term}}  {\texttt{L3}}    {1} {$\{\langle p_1, p_2, m \rangle\}$}};
      \state[structure ob=<3>,   sshape]            (23) at (2,1) {\scontent{\texttt{term}}  {\texttt{L3}}    {2} {$\emptyset$}};
      \state[structure ob=<3>,   sshape, accepting] (24) at (2,0) {\scontent{\texttt{term}}  {\texttt{term}}  {2} {$\emptyset$}};
      
      \path[alert]     (00) edge node[left]{$p_1$ : \Send $m$}     (11);
      \path[structure] (11) edge node[left]{$p_2$ : \Receive $m$}  (12);
      \path[structure] (12) edge node[left]{$p_2$ : \texttt{L2}}   (13);
      \path[structure] (13) edge node[left]{$p_2$ : $\Print(1)$} (14);

      \path[structure] (00) edge node[right]{~$p_2$ : \texttt{L2}}   (21);
      \path[alert]     (21) edge node[right]{$p_1$ : \Send $m$}     (22);
      \path[structure] (22) edge node[right]{$p_2$ : \Receive $m$}  (23);
      \path[structure] (23) edge node[right]{$p_2$ : $\Print(2)$} (24);

      \path[structure] (21) edge[loop right, looseness=2] node[right]{$p_2$ : \texttt{L3}} (21);
    \end{tikzpicture}
  }

  \on<2>[x=-.3\textwidth, bottom=3mm]{
    \begin{tikzpicture}[y=8mm, anchor=center, x=9mm]
      \draw[process] (0,1) node[left]{$p_1$} to (5,1);
      \draw[process] (0,0) node[left]{$p_2$} to (5,0);
      \scriptsize
      \node[example, operation,minimum height=5mm] (p2) at (3.5,0) {$\X_2 \leftarrow 1$};
      \begin{scope}[alert]
        \node[event]     (send-m) at (.5,1) {};
        \node[operation,minimum height=5mm] (rece-m) at (1.5,0) {$\X_2 \leftarrow 2$};
        \path[message]   (send-m.center) edge node{$m$} (rece-m.west);
      \end{scope}
    \end{tikzpicture}
  }

  \ob<3>[x=-.3\textwidth, bottom=3mm]{
    \begin{tikzpicture}[y=8mm, anchor=center, x=9mm]
      \draw[process] (0,1) node[left]{$p_1$} to (5,1);
      \draw[process] (0,0) node[left]{$p_2$} to (5,0);
      \scriptsize
      \node[example, operation, minimum height=5mm] (p2) at (2.5,0) {$\X_2 \leftarrow 1$ \hspace{2cm}};
      \begin{scope}[alert]
        \node[event]     (send-m) at (.5,1) {};
        \node[operation] (rece-m) at (3,0) {$\X_2 \leftarrow 2$};
        \path[message]   (send-m.center) edge node{$m$} (rece-m.west);
      \end{scope}
    \end{tikzpicture}
  }

  \ob<4>[x=-.3\textwidth, bottom=3mm]{
    \begin{tikzpicture}[y=8mm, anchor=center, x=9mm]
      \draw[crashed] (0,1) node[left]{$p_1$} to (1,1);
      \draw[process] (0,0) node[left]{$p_2$} to (5,0);
      \scriptsize
      \node[example,   operation, minimum height=5mm] (p2) at (2.5,0) {$\X_2 \leftarrow 1$ \hspace{2.2cm}};
      \node[structure, operation, minimum height=4mm] (wu) at (3.2,0) {\Wait \False...};
    \end{tikzpicture}
  }
  
\end{frame}

\endgroup
\endinput

% SPDX-License-Identifier: CC-BY-SA-4.0
% Author: Matthieu Perrin
% Part: 
% Section: 
% Sub-section: 
% Frame: 

\begingroup

\begin{frame}{Exemple}

  \on{
    \begin{tikzpicture}[y=20mm]

      \uncoverb<2,4>{
        \draw (4.5, -1) node{Motif de messages MP2};
      }
      \uncoverb<3>{
        \draw (4.5, -1) node{Motif de messages MP3};
      }
      
      \draw[process, fade ob=<4>] (0,2) node[left]{$p_1$} to (10,2);
      \draw[process, fade ob=<3>] (0,1) node[left]{$p_2$} to (10,1);
      \draw[process, fade ob=<2>] (0,0) node[left]{$p_3$} to (10,0);
      
      \draw[alert, fade ob=<4>] (1,2) node (m1) {} node[below left]{$m_1$};
      \path[message, alert,     fade ob=<{3,4}>, thick] (m1.center) edge             (6,1); 
      \path[message, alert,     fade ob=<{2,4}>, thick] (m1.center) edge             (5,0); 
      \path[message, alert,     fade ob=<{4}>  , thick] (m1.center) edge[bend left]  (7,2); 

      \draw[example, fade ob=<3>] (1,1) node (m2) {} node[below left]{$m_2$};
      \path[message, example,   fade ob=<{3,4}>, thick] (m2.center) edge             (3,2); 
      \path[message, example,   fade ob=<{2,3}>, thick] (m2.center) edge             (7,0); 
      \path[message, example,   fade ob=<{3}>  , thick] (m2.center) edge[bend left]  (4,1); 

      \draw[structure, fade ob=<2>] (1,0) node (m3) {} node[above left]{$m_3$};
      \path[message, structure, fade ob=<{2,4}>, thick] (m3.center) edge             (5,2); 
      \path[message, structure, fade ob=<{2,3}>, thick] (m3.center) edge             (8,1); 
      \path[message, structure, fade ob=<{2}>  , thick] (m3.center) edge[bend right] (9,0); 
    \end{tikzpicture}
  }

\end{frame}

\endgroup
\endinput

 
\subsection{La relation Happened before}
% SPDX-License-Identifier: CC-BY-SA-4.0
% Author: Matthieu Perrin
% Part: 
% Section: 
% Sub-section: 
% Frame: 

\begingroup

\begin{frame}{Lamport et la relativité du temps}

  \on[y=3mm, left=.7\textwidth]{
    \begin{shadequote}{Leslie Lamport}
      The \alert{concept of time} is fundamental to our way of
      thinking. \alert{It is derived} from the more basic concept of
      the \alert{order in which events occur}. (...)\\[2mm]
      
      However, we will see that this concept must be carefully reexamined when considering events in a distributed system. (...)\\[2mm]
      
      In a distributed system, it is sometimes impossible to
      say that one of two events occurred first.
      \alert{The relation
        "happened before" is therefore only a partial ordering
        of the events in the system.}
    \end{shadequote}
  }

  \onImage[x=40mm, y=12mm]{%
    width=2.5cm,
    title={Leslie Lamport\footnote{Prix Turing 2013 pour ses apports au calcul réparti}},
    license={{CC-0} (\ccZero{} --- Usage allowed for any purpose, \href{https://commons.wikimedia.org/wiki/File:Leslie_Lamport.jpg}{Wikimedia})},
    img={Lamport.jpg}
  }

  \on[bottom=5mm, text]{
    \begin{citing}
    \item[L78] Leslie Lamport. \textit{Time, clocks, and the ordering of events in a distributed system.} CACM (1978)
    \end{citing}
  }  
  
\end{frame}

\endgroup
\endinput


% SPDX-License-Identifier: CC-BY-SA-4.0
% Author: Matthieu Perrin
% Part: 
% Section: 
% Sub-section: 
% Frame: 

\begingroup

\begin{frame}{La relation ``happened before'' (ordre causal)}

  \begin{block}{Définition -- Happened before}
    Soient $a$ et $a'$ deux pas. 
    \begin{itemize}
    \item $a\hb a'$ si l'une des trois conditions s'applique :    
      \begin{itemize}
      \item \structure{Ordre de processus :} $a$ précède $a'$ dans l'ordre local d'un processus
      \item \structure{Ordre de messages :} $a$ est l'envoi et $a'$ la réception d'un même message
      \item \structure{Transitivité :} $\exists a''~ a\hb a'' \land a'' \hb a'$
      \end{itemize}
    \item $a$ et $a'$ sont \structure{concurrents} $(a || a')$ si $a\not\hb a'$ et $a'\not\hb a$.
    \item Une \structure{exécution causale} est une classe d'équivalence d'exécutions concrètes induisant le même ordre $\hb$.
    \end{itemize}
    \alert{Attention :} $\hb$ est seulement un ordre \alert{partiel} !
  \end{block}

  \begin{exampleblock}{Programme d'exemple : trois exécutions causales}
    \centering
    \begin{tikzpicture}[y=10mm, anchor=center, x=36mm]

      \node at (0,1) {
        $\alert{s_1 \hb r_1} ~\structure{\hb}~ \example{s_2 \hb r_2}$
      };
      
      \node at (1,1) {
        $\example{s_2 \hb r_2} ~\structure{\hb}~ \alert{s_1 \hb r_1}$
      };
      
      \node at (2,1) {
        $\alert{s_1 \hb r_1} \quad \example{s_2 \hb r_2}$
      };

      \node at (0,0) {
        \begin{tikzpicture}[y=5mm, x=5mm]
          \draw[process] (0,1) node[left]{$p_1$} to (5,1);
          \draw[process] (0,0) node[left]{$p_2$} to (5,0);
          \draw[alert, message]   (1,1) node[above]{$s_1$} -- (2,0) node[below]{$r_1$};
          \draw[example, message] (3,0) node[below]{$s_2$} -- (4,1) node[above]{$r_2$};
        \end{tikzpicture}
      };

      \node at (1,0) {
        \begin{tikzpicture}[y=5mm, x=5mm]
          \draw[process] (0,1) node[left]{$p_1$} to (5,1);
          \draw[process] (0,0) node[left]{$p_2$} to (5,0);
          \draw[alert, message]   (3,1) node[above]{$s_1$} -- (4,0) node[below]{$r_1$};
          \draw[example, message] (1,0) node[below]{$s_2$} -- (2,1) node[above]{$r_2$};
        \end{tikzpicture}
      };

      \node at (2,0) {
        \begin{tikzpicture}[y=5mm, x=5mm]
          \draw[process] (0,1) node[left]{$p_1$} to (5,1);
          \draw[process] (0,0) node[left]{$p_2$} to (5,0);
          \draw[alert, message]   (1.5,1) node[above]{$s_1$} -- (3.5,0) node[below]{$r_1$};
          \draw[example, message] (1.5,0) node[below]{$s_2$} -- (3.5,1) node[above]{$r_2$};
        \end{tikzpicture}
      };
      
    \end{tikzpicture}
  \end{exampleblock}
  
\end{frame}


\endgroup
\endinput


% SPDX-License-Identifier: CC-BY-SA-4.0
% Author: Matthieu Perrin
% Part: 
% Section: 
% Sub-section: 
% Frame: 

\begingroup

\SetKwFunction{Time}{time}

\begin{frame}{Temps et horloges logiques}
 
  \begin{block}{Définitions -- Temps logique}
    \begin{description}[Horloge :]
    \item[Temps :] relation d'ordre \structure{$\rightarrow$} (partielle ou totale) sur les pas d'exécution
    \item[Date :] type de valeurs muni d'un ordre (partiel ou total) \structure{$\le$}
    \item[Horloge :] fonction (algorithmique) \alert{strictement croissante} \structure{$\Time$}
      qui associe une date à certains pas : 
      $\alert{\forall a, b, \quad a \rightarrow b \quad \Rightarrow \quad \Time(a) < \Time(b)}$
    \end{description}
  \end{block}

  \begin{exampleblock}{Exemple -- Temps physique}
    \vspace{-2mm}
    \begin{description}[Temps :]
    \item[Temps :] ordre total en mécanique newtonnienne
    \item[Date :] nombre réel $t$ mesuré en secondes
    \item[Horloge :] à pendule, ressort, quartz, atomique, ...
    \end{description}
  \end{exampleblock}

  \pause
  
  \begin{alertblock}{Exemple -- Horloges de Lamport}
    \vspace{-2mm}
    \begin{description}[Horloge :]
    \item[Date :] horloge \structure{scalaire} \hfill \alert{$\mathit{Date} = \mathbb{N}$}
    \item[Temps :] \vspace{-1mm} capture la relation \structure{happened before} \hfill \alert{$a \hb b  \rightarrow \Time(a) < \Time(b)$}
    \end{description}
    \alert{Attention :} La réciproque n'est pas vraie en général :
    \centering
    $\alert{\exists a, b, a || b \land \Time(a) < \Time(b)} \quad \quad \alert{\exists a, b, a \neq b \land \Time(a) = \Time(b)}$
  \end{alertblock}
  
\end{frame}

\endgroup
\endinput


% SPDX-License-Identifier: CC-BY-SA-4.0
% Author: Matthieu Perrin
% Part: 
% Section: 
% Sub-section: 
% Frame: 

\begingroup

\SetKwFunction{Time}{time}
\SetKwData{TS}{ts}

\begin{frame}{Horloge logique de Lamport}
  
  \begin{algorithm}[H]
    \lLVariables{}{
      $\TS_i \leftarrow 0$\tcp*[f]{date scalaire locale (entier)}
    }
    \When{dating an internal step $a$}{
      $\TS_i \leftarrow \TS_i + 1$
      \tcp*[r]{$\Time(a) = \TS_i$}
    }
    \When{sending $m$ \To $p_j$}{
      $\TS_i \leftarrow \TS_i + 1$\;
      \Send $\textsc{stamped}(m, \TS_i)$ \To $p_j$
      \tcp*[r]{$\Time(s_m) = \TS_i$}
    }
    \When{\Receive $\textsc{stamped}(m, t_j)$ \From $p_j$}{
      $\TS_i \leftarrow \max(\TS_i, t_j) + 1$\;
      \KWStyle{receive} $m$
      \tcp*[r]{$\Time(r_m) = \TS_i > t_j$}
    }
  \end{algorithm}

  \begin{exampleblock}{Exemple d'exécution \only<2>{: zones temporelles}\only<3>{: exécution indistinguable}}
    \only<-2>{
      \begin{tikzpicture}[y=8mm, x=12mm, shorten >=2pt]
        \draw[process] (0,2) node[left]{$p_1$} to (8,2);
        \draw[process] (0,1) node[left]{$p_2$} to (8,1);
        \draw[process] (0,0) node[left]{$p_3$} to (8,0);

        \draw    (0.5,2) node[alert] (11) {$\bullet$} node[above]{1};
        \draw    (2  ,2) node[alert] (21) {$\bullet$} node[above]{2};
        \draw    (3  ,2) node[alert] (31) {$\bullet$} node[above]{3};
        \draw    (5.5,2) node[alert] (41) {$\bullet$} node[above]{4};
        \node at (6  ,2)             (51) {};
        \draw    (7  ,2) node[alert] (61) {$\bullet$} node[above]{6};
        \node at (7.5,2)             (71) {};

        \draw    (1.5,1) node[alert] (12) {$\bullet$} node[above]{1};
        \draw    (2.5,1) node[alert] (22) {$\bullet$} node[below]{2};
        \node at (3  ,1)             (32) {};
        \draw    (3.5,1) node[alert] (42) {$\bullet$} node[below]{4};
        \draw    (4.5,1) node[alert] (52) {$\bullet$} node[below]{5};
        \draw    (5.5,1) node[alert] (62) {$\bullet$} node[above]{6};
        \node at (6.5,1)             (72) {};

        \draw    (1  ,0) node[alert] (13) {$\bullet$} node[below]{1};
        \draw    (2  ,0) node[alert] (23) {$\bullet$} node[below]{2};
        \draw    (3  ,0) node[alert] (33) {$\bullet$} node[below]{3};
        \draw    (4  ,0) node[alert] (43) {$\bullet$} node[below]{4};
        \draw    (5  ,0) node[alert] (53) {$\bullet$} node[below]{5};
        \node at (6  ,0)             (63) {};
        \draw    (7  ,0) node[alert] (73) {$\bullet$} node[below]{7};

        \path[alert, message] (21.center) edge            (52.center);
        \path[alert, message] (31.center) edge            (42.center);
        \path[alert, message] (12.center) edge            (23.center);
        \path[alert, message] (53.center) edge            (61.center);
        \path[alert, message] (33.center) edge[bend left] (43.center);
        \path[alert, message] (62.center) edge            (73.center);

        \coordinate (0top) at (0 ,2.5);
        \coordinate (0bot) at (0 ,-.5);

        \begin{scope}[background]
          \uncover<2>{
            \fill[structure!80, rounded corners] (0bot) rectangle (8,2.5);
            \fill[structure!70, rounded corners] (0top) -- (71.west |- 0top) -- (71.west) -- (72.west) -- (73.west) -- (73.west |- 0bot) -- (0bot);
            \fill[structure!60, rounded corners] (0top) -- (61.west |- 0top) -- (61.west) -- (62.west) -- (63.west) -- (63.west |- 0bot) -- (0bot);
            \fill[structure!50, rounded corners] (0top) -- (51.west |- 0top) -- (51.west) -- (52.west) -- (53.west) -- (53.west |- 0bot) -- (0bot);
            \fill[structure!40, rounded corners] (0top) -- (41.west |- 0top) -- (41.west) -- (42.west) -- (43.west) -- (43.west |- 0bot) -- (0bot);
            \fill[structure!30, rounded corners] (0top) -- (31.west |- 0top) -- (31.west) -- (32.west) -- (33.west) -- (33.west |- 0bot) -- (0bot);
            \fill[structure!20, rounded corners] (0top) -- (21.west |- 0top) -- (21.west) -- (22.west) -- (23.west) -- (23.west |- 0bot) -- (0bot);
            \fill[structure!10, rounded corners] (0top) -- (11.west |- 0top) -- (11.west) -- (12.west) -- (13.west) -- (13.west |- 0bot) -- (0bot);
          }
        \end{scope}
        
      \end{tikzpicture}
    }
    \onlyb<3>{
      \begin{tikzpicture}[y=8mm, x=12mm, shorten >=2pt]
        \draw[process] (0,2) node[left]{$p_1$} to (8,2);
        \draw[process] (0,1) node[left]{$p_2$} to (8,1);
        \draw[process] (0,0) node[left]{$p_3$} to (8,0);

        \draw    (1  ,2) node[alert] (11) {$\bullet$} node[above]{1};
        \draw    (2  ,2) node[alert] (21) {$\bullet$} node[above]{2};
        \draw    (3  ,2) node[alert] (31) {$\bullet$} node[above]{3};
        \draw    (4  ,2) node[alert] (41) {$\bullet$} node[above]{4};
        \node at (5  ,2)             (51) {};
        \draw    (6  ,2) node[alert] (61) {$\bullet$} node[above]{6};

        \draw    (1  ,1) node[alert] (12) {$\bullet$} node[above]{1};
        \draw    (2  ,1) node[alert] (22) {$\bullet$} node[below]{2};
        \node at (3  ,1)             (32) {};
        \draw    (4  ,1) node[alert] (42) {$\bullet$} node[below]{4};
        \draw    (5  ,1) node[alert] (52) {$\bullet$} node[below]{5};
        \draw    (6  ,1) node[alert] (62) {$\bullet$} node[above]{6};

        \draw    (1  ,0) node[alert] (13) {$\bullet$} node[below]{1};
        \draw    (2  ,0) node[alert] (23) {$\bullet$} node[below]{2};
        \draw    (3  ,0) node[alert] (33) {$\bullet$} node[below]{3};
        \draw    (4  ,0) node[alert] (43) {$\bullet$} node[below]{4};
        \draw    (5  ,0) node[alert] (53) {$\bullet$} node[below]{5};
        \node at (5  ,0)             (63) {};
        \draw    (7  ,0) node[alert] (73) {$\bullet$} node[below]{7};

        \path[alert, message] (21.center) edge            (52.center);
        \path[alert, message] (31.center) edge            (42.center);
        \path[alert, message] (12.center) edge            (23.center);
        \path[alert, message] (53.center) edge            (61.center);
        \path[alert, message] (33.center) edge[bend left] (43.center);
        \path[alert, message] (62.center) edge            (73.center);

        \coordinate (0top) at (0 ,2.5);
        \coordinate (0bot) at (0 ,-.5);
        
        \begin{scope}[background]
          \fill[structure!80, rounded corners] (0bot) rectangle (8,2.5);
          \fill[structure!70, rounded corners] (0bot) rectangle (6.5,2.5);
          \fill[structure!60, rounded corners] (0bot) rectangle (5.5,2.5);
          \fill[structure!50, rounded corners] (0bot) rectangle (4.5,2.5);
          \fill[structure!40, rounded corners] (0bot) rectangle (3.5,2.5);
          \fill[structure!30, rounded corners] (0bot) rectangle (2.5,2.5);
          \fill[structure!20, rounded corners] (0bot) rectangle (1.5,2.5);
          \fill[structure!10, rounded corners] (0bot) rectangle (0.5,2.5);
          \fill[structure!10, rounded corners] (0bot) -- (0.5,-0.5) -- (0.5,2.5) -- (0top);
        \end{scope}
        
      \end{tikzpicture}
    }
  \end{exampleblock}
  
\end{frame}

\endgroup
\endinput

 
\subsection{Un premier résultat d'impossibilité}
% SPDX-License-Identifier: CC-BY-SA-4.0
% Author: Matthieu Perrin
% Part: 
% Section: 
% Sub-section: 
% Frame: 

\begingroup

\SetKwFunction{Crashed}{crashed}

\begin{frame}{Implémentation d'un détecteur de fautes}

  \onBlock[top=-5mm]{Problème -- Détecteur de fautes parfait $P$}{
    
    \begin{description}[Complétude :]
    \item[Interface :] Une méthode $\Crashed(p_j)$, appelée par $p_i$
      \begin{algorithm}[H]
        \Interface{$\mathit{P}$}{
          \lMethod{$\Crashed(p_j \in \mathit{Proc}) \in \mathbb{B}$}{\tcp*[h]{appelée par $p_i$}}
        }
      \end{algorithm}
    \item[Vivacité :] $\Crashed$ termine
    \item[Exactitude :] si $p_j$ est \alert{correct},  $\Crashed(p_j) \rightarrow \False$
    \item[Complétude :] si $p_j$ est \alert{déjà tombé en panne}, $\Crashed(p_j) \rightarrow \True$
    \end{description}
  }

  \onBlock<2->[y=-12mm]{Théorème -- Inexistence de $P$}{
    Il n'existe aucune implémentation de $P$ dans le modèle considéré.

    \structure{Démonstration :} par \alert{indistinguabilité} des exécutions suivantes par $p_i$.
  }

  \on<2->[x=-.3\textwidth, bottom=3mm]{
    \begin{tikzpicture}[y=8mm, anchor=center, x=9mm]
      \draw[crashed] (0,1) node[left]{$p_j$} to (.5,1);
      \draw[process] (0,0) node[left]{$p_i$} to (4.5,0);
      
      \node[structure, operation] (pj) at (2.25,0) {$\Crashed(p_j) \rightarrow \True$};
    \end{tikzpicture}
  }

  \on<2->[x=.2\textwidth, bottom=3mm]{
    \begin{tikzpicture}[y=8mm, anchor=center, x=9mm]
      \draw[process] (0,1) node[left]{$p_j$} to (6,1);
      \draw[process] (0,0) node[left]{$p_i$} to (6,0);

      \draw[dashed] (.5,.5) node[left]{\scriptsize réseau lent} -- (4 ,.5);
      
      \node[structure, operation] (pj) at (2.1,0) {$\Crashed(p_j) \rightarrow \True$};

      \begin{scope}[alert]
        \node[event]   (send-1) at (1,1) {};
        \node[event]   (rece-1) at (5,0) {};
        \path[message] (send-1.center) edge[out=-25, in=140] (rece-1.center);
      \end{scope}

      \begin{scope}[example]
        \node[event]   (send-2) at (2.6,1) {};
        \node[event]   (rece-2) at (5.6,0) {};
        \path[message] (send-2.center) edge[out=-10, in=140] (rece-2.center);
      \end{scope}


    \end{tikzpicture}
  }
  
\end{frame}

\endgroup
\endinput

 
 
\part{Communication}
 
 
\section{Problèmes de cohérence}
 
\subsection{Service de messagerie instantanée}
% SPDX-License-Identifier: CC-BY-SA-4.0
% Author: Matthieu Perrin
% Part: 
% Section: 
% Sub-section: 
% Frame: 

\begingroup

\begin{frame}{Services de messagerie \onlyb<6>{: Hangouts}\onlyb<7>{: WhatsApp}\onlyb<8>{: Skype}}

  \on[y=-5mm]{
    \begin{tikzpicture}
      \node     [faded background picture=Seine,  text width=\paperwidth/2] (A) at (-\paperwidth/4,0) {};
      \node<-2> [faded background picture=Jardin, text width=\paperwidth/2] (B) at ( \paperwidth/4,0) {};
      \node<5-> [faded background picture=Rue,    text width=\paperwidth/2] (B) at ( \paperwidth/4,0) {};
      \node<3-4>[faded background picture=Metro,  text width=\paperwidth/2] (B) at ( \paperwidth/4,0) {};

      \node[anchor=south, nosep] at (A.south) {\includegraphics[height=22mm]{Alice}};
      \node[anchor=south, nosep] at (B.south) {\includegraphics[height=22mm]{Bob}};

      \uncoverb<6>{
        \node at (0,2.5) {\includegraphics[width=8mm]{Hangout}};
      }
      \uncoverb<7>{
        \node at (0,2.5) {\includegraphics[width=8mm]{Whatsapp}};
      }
      \uncoverb<8>{
        \node at (0,2.5) {\includegraphics[width=8mm]{Skype}};
      }
    \end{tikzpicture}
  }

  \on[x=-.25\paperwidth, y=15mm]{
    \begin{chat}[color=structure]{Bob}
      \only<2->{\chatSend{Pause café ?}}
      \onlyb<8>{\chatRecv[color=example]{Évidemment !}}
      \only<4->{\chatSend{Tu n'as pas répondu. \\ Tu m'en veux ?}}
      \onlyb<7>{\chatRecv[color=example]{Évidemment !}}
      \only<5>{\node at (0,-25mm) {\Large ?};}
    \end{chat}
  }

  \on[x=.25\paperwidth, y=15mm]{
    \begin{chat}[color=example]{Alice}
      \only<2->{\chatRecv[color=structure]{Pause café ?}}
      \only<3-5,7->{\chatSend{Évidemment !}}
      \onlyb<6>{\chatErr{Évidemment !}}
      \onlyb<6->{\chatRecv[color=structure]{Tu n'as pas répondu. \\ Tu m'en veux ?}}
      \only<5>{\node at (0,-25mm) {\Large ?};}
    \end{chat}
  }

  \ob<2>[x=-.25\paperwidth, y=-15mm]{
    \chatBubble[color=structure]{Pause café ?}
  }

  \on<3>[x=.25\paperwidth, y=-15mm]{
    \chatBubble[color=example]{Évidemment !}
  }

  \on<4>[x=-.25\paperwidth, y=-15mm]{
    \chatBubble[color=structure]{Tu n'as pas répondu. \\ Tu m'en veux ?}
  }

\end{frame}

\endgroup
\endinput

% SPDX-License-Identifier: CC-BY-SA-4.0
% Author: Matthieu Perrin
% Part: 
% Section: 
% Sub-section: 
% Frame: 

\begingroup

\begin{frame}{Critères de cohérence faibles}

  \onBlock[left=.6\textwidth, top=-1mm]{Eventual Consistency}{
    \begin{itemize}
    \item Tous les processus finissent dans le même état
    \item \example{Contre-exemple : WhatsApp}
    \end{itemize}
  }
  
  \onBlock[left=.6\textwidth, y=-3mm]{Validity}{
    \begin{itemize}
    \item Chaque processus finit dans un état qui reflète toutes les écritures
    \item \example{Contre-exemple : Hangouts}
    \end{itemize}
  }
  
  \onBlock[left=.6\textwidth, bottom]{State locality}{
    \begin{itemize}
    \item Tout changement d'état vient de l'exécution d'une écriture
    \item \example{Contre-exemple : Skype}
    \end{itemize}
  }

  \on[x=35mm, y=20mm]{
    \scalebox{.7}{
      \begin{chat}[color=structure, height=3cm]{Bob}
        \chatSend{Pause café ?}
        \chatSend{Tu n'as pas répondu. \\ Tu m'en veux ?}
        \chatRecv[color=example]{Évidemment !}
      \end{chat}
      \begin{chat}[color=example, height=3cm]{Alice}
        \chatRecv[color=structure]{Pause café ?}
        \chatSend{Évidemment !}
        \chatRecv[color=structure]{Tu n'as pas répondu. \\ Tu m'en veux ?}
      \end{chat}
    }
  }

  \on[x=35mm, y=-5mm]{
    \scalebox{.7}{
      \begin{chat}[color=structure, height=3cm]{Bob}
        \chatSend{Pause café ?}
        \chatSend{Tu n'as pas répondu. \\ Tu m'en veux ?}
      \end{chat}
      \begin{chat}[color=example, height=3cm]{Alice}
        \chatRecv[color=structure]{Pause café ?}
        \chatErr{Évidemment !}
        \chatRecv[color=structure]{Tu n'as pas répondu. \\ Tu m'en veux ?}
      \end{chat}
    }
  }

  \on[x=35mm, y=-30mm]{
    \scalebox{.7}{
      \begin{chat}[color=structure, height=3cm]{Bob}
        \chatSend{Pause café ?}
        \chatSend{Tu n'as pas répondu. \\ Tu m'en veux ?}
      \end{chat}
      \begin{chat}[color=structure, height=3cm]{Bob}
        \chatSend{Pause café ?}
        \chatRecv[color=example]{Évidemment !}
        \chatSend{Tu n'as pas répondu. \\ Tu m'en veux ?}
      \end{chat}
    }
  }
  
\end{frame}

\endgroup
\endinput

% SPDX-License-Identifier: CC-BY-SA-4.0
% Author: Matthieu Perrin
% Part: 
% Section: 
% Sub-section: 
% Frame: 

\begingroup

\definecolor{myRed}{RGB}{245,180,170}
\definecolor{myGreen}{RGB}{178,238,167}
\definecolor{myBlue}{RGB}{172,206,240}
\definecolor{myYellow}{RGB}{245,245,202}
\definecolor{myCyan}{RGB}{202,235,235}
\definecolor{myMagenta}{RGB}{236,202,236}

\tikzset{
  weakerthan/.style={
    densely dashed,
    postaction={decorate,decoration={markings, mark=at position .65 with {\arrow{latex}}}},
  },
  criterion/.style={
    draw,
    ultra thin,
    circle,
    inner sep=1pt,
    font=\footnotesize,
    fill=white
  },
  criterion text/.style={
    align=center,
    font=\tiny,
  },
}

\newcommand{\fillinter}[2]{%
  \begin{scope}
    \foreach \c in {#1} {
      \clip \c circle (22mm);
    }
    \fill[#2] circle (30mm);
    \foreach \c in {#1} {
      \draw[black!60] \c circle (22mm);
    }
  \end{scope}
}

\begin{frame}{Carte des critères de cohérence faibles}

  \onBlock[top]{Critères primaires}{
    \begin{itemize}
    \item[{\myRec[draw=black!60, fill=myBlue]}]  localité d'état
    \item[{\myRec[draw=black!60, fill=myRed]}]   Eventual consistency
    \item[{\myRec[draw=black!60, fill=myGreen]}] Validité
    \end{itemize}
  }

  \onBlock[y=-7mm]{Critères secondaires}{
    \begin{itemize}
    \item[{\myRec[draw=black!60, fill=myYellow]}]  Cohérence d'écritures
    \item[{\myRec[draw=black!60, fill=myCyan]}]    Cohérence PRAM
    \item[{\myRec[draw=black!60, fill=myMagenta]}] Sérialisabilité
    \end{itemize}
  }
  
  \onBlock[bottom]{Critères forts}{
    \begin{itemize}
    \item[{\myRec[draw=black!60, fill=black!5]}] Cohérence séquentielle/linéarisabilité
    \end{itemize}
  }

  \on[x=20mm]{
    \begin{tikzpicture}
      \clip circle (30mm);

      \coordinate (EC) at ( 90:10mm);
      \coordinate (VC) at (-30:10mm);
      \coordinate (SL) at (210:10mm);

      \fillinter{}              {black!50}
      \fillinter{(EC)}          {myRed}
      \fillinter{(VC)}          {myGreen}
      \fillinter{(SL)}          {myBlue}
      \fillinter{(SL),(VC)}     {myCyan}
      \fillinter{(EC),(SL)}     {myMagenta}
      \fillinter{(EC),(VC)}     {myYellow}
      \fillinter{(EC),(SL),(VC)}{black!3}
      
      \node[criterion] (CSC)               {}; \node[criterion text, below left ] at (CSC) {Cohérence\\forte}; 
      \node[criterion] (CPC) at (-90:20mm) {}; \node[criterion text, below      ] at (CPC) {Cohérence\\PRAM}; 
      \node[criterion] (CC)  at (-90:15mm) {}; \node[criterion text, left       ] at (CC)  {Cohérence\\causale}; 
      \node[criterion] (CUC) at ( 30:20mm) {}; \node[criterion text, above      ] at (CUC) {Cohérence\\d'écritures}; 
      \node[criterion] (CCv) at ( 30:15mm) {}; \node[criterion text, left       ] at (CCv) {Convergence\\causale}; 
      \node[criterion] (CSE) at (150:20mm) {}; \node[criterion text, right      ] at (CSE) {Serialisabilité}; 
      \node[criterion] (CEC) at ( 90:25mm) {}; \node[criterion text, above      ] at (CEC) {Eventual consistency}; 
      \node[criterion] (CVC) at (-30:25mm) {}; \node[criterion text, below      ] at (CVC) {Validité}; 
      \node[criterion] (CCF) at (-30:20mm) {}; \node[criterion text, above left ] at (CCF) {Cohérence\\causale faible}; 
      \node[criterion] (CSL) at (210:25mm) {}; \node[criterion text, below      ] at (CSL) {Localité\\d'état}; 

      \path (CSL) edge[weakerthan] (CPC) ;
      \path (CSL) edge[weakerthan] (CSE) ;
      \path (CVC) edge[weakerthan] (CPC) ;
      \path (CVC) edge[weakerthan] (CUC) ;
      \path (CEC) edge[weakerthan] (CUC) ;
      \path (CEC) edge[weakerthan] (CSE) ;
      \path (CPC) edge[weakerthan] (CC)  ;
      \path (CUC) edge[weakerthan] (CCv) ;
      \path (CCv) edge[weakerthan] (CSC) ;
      \path (CC)  edge[weakerthan] (CSC) ;
      \path (CSE) edge[weakerthan] (CSC) ;
      \path (CVC) edge[weakerthan] (CCF) ;
      \path (CCF) edge[weakerthan] (CCv) ;
      \path (CCF) edge[weakerthan] (CC)  ;

      \node[circle, minimum size=60mm, inner sep=0, draw=black!75] {};
    \end{tikzpicture}
  }

\end{frame}

\endgroup
\endinput

 
\subsection{Eventual consistency}
% SPDX-License-Identifier: CC-BY-SA-4.0
% Author: Matthieu Perrin
% Part: Communication
% Section: Cohérence faible
% Frame: Eventual Consistency

\begingroup

\SetKwFunction{SetKey}{setKey}
\SetKwFunction{DeleteKey}{deleteKey}
\SetKwFunction{Get}{get}
\SetKwFunction{GetKeys}{getKeys}

\begin{frame}{Machine à états à requêtes/mises à jour}

  Une machine à états $M=\langle C, R, Q, q_0, \tau, \rho \rangle$ est dite \structure{à requêtes/mises à jour} si
  toutes ses commandes $c\in C$ sont dans l'une des deux catégories : 
  \begin{description}
  \item[Requête :] \alert{$c \in \mathit{Query}$} ne modifie pas l'état
    \hfill $\alert{\forall q\in Q, \tau(q, c) = q}$\\
  \item[Mise à jour :] \alert{$c \in \mathit{Update}$} retourne $\bot$
    \hfill $\alert{\exists \bot\in R, \forall q\in Q, \rho(q, c) = \bot}$\\
  \end{description}

  \vspace{2mm}
  On étend la notation aux exécutions $E$ d'une implémentation répliquée de $M$ :
  \alert{$\mathit{Query}(E) = \{ c\rightarrow r \in E \mid c \in \mathit{Query} \}$ \hfill $\mathit{Update}(E) = \{ c\rightarrow r \in E \mid c \in \mathit{Update} \}$}

  \pause
  \vspace{2mm}

  \begin{exampleblock}{Exemple -- Bases de données clé-valeur}
    Une \example{base de données clé-valeur} sur un ensemble de clés $K$ et un ensemble de valeurs $V$
    est une machine à états à requêtes/mises à jour avec : 
    \begin{itemize}
    \item Un état $q\in Q$ est une \example{fonction partielle de $K$ dans $V$}
    \item Les mise à jour sont de la forme \example{$\SetKey(k,v)$} ou \example{$\DeleteKey(k)$}
      \begin{center}
        $\tau(q, \SetKey(k,v)) = q[k \mapsto v]  \quad\quad \tau(q, \DeleteKey(k)) = q \setminus (\{k\}\times V)$
      \end{center}
    \item Les requêtes sont de la forme \example{$\Get(k)$} ou \example{$\GetKeys()$}
      \begin{center}
        $\rho(q, \Get(k)) = q(k) ~~ (\text{ou } \bot \text{ si } k \notin dom(q))  \quad \rho(q, \GetKeys()) = dom(q)$
      \end{center}
    \end{itemize}
  \end{exampleblock}

\end{frame}

\endgroup
\endinput

% SPDX-License-Identifier: CC-BY-SA-4.0
% Author: Matthieu Perrin
% Part: Communication
% Section: Cohérence faible
% Frame: Eventual Consistency

\begingroup

\begin{frame}{Eventual Consistency}

  \vspace{-1mm}
  \begin{block}{Définition -- Eventual Consistency}
    \vspace{-1mm}
    Une exécution $E$ vérifie \structure{EC} si :
    s'il n'y a qu'un \structure{nombre fini} de mises à jour, alors
    il existe un \structure{état de convergence} $q_{\mathrm{cv}}$ tel que 
    \structure{toutes sauf un nombre fini} de requêtes renvoient la même réponse que dans $q_{\mathrm{cv}}$.
    $$\alert{|\mathit{Update}(E)| < \infty \Rightarrow \exists q_{cv}\in Q : |\{ c \rightarrow r \in \mathit{Query}(E) \mid \rho(q_{cv}, c) \neq r\}| < \infty}$$
  \end{block}

  \begin{tikzpicture}[y=4.5mm]
    \draw[process] (0,2) node[left]{$p_1$} node[replica] {$q_0$} to (10,2) node[replica, above left] {$q_{cv}$}; 
    \draw[process] (0,1) node[left]{$p_2$} node[replica] {$q_0$} to (10,1) node[replica, above left] {$q_{cv}$}; 
    \draw[process] (0,0) node[left]{$p_3$} node[replica] {$q_0$} to (10,0) node[replica, above left] {$q_{cv}$}; 

    \draw[double]  (3,-.2) -- (3,2.2);
    \draw[double]  (6,-.2) -- (6,2.2);
    \node at (3,2) [replica] {$q_1$};
    \node at (3,1) [replica] {$q_2$};
    \node at (3,0) [replica] {$q_3$};
    \node at (6,2) [replica] {$q_{cv}$};
    \node at (6,1) [replica] {$q_{cv}$};
    \node at (6,0) [replica] {$q_{cv}$};
    \node[below=2mm] at (1.5,0)  {\small mises à jour (fini)};
    \node[below=2mm] at (4.5,0)  {\small pré-convergence (fini)};
    \node[below=2mm] at (8,0)    {\small convergence (infini)};
    
    \node[structure, operation] at (2,2)    {\tiny $u_2$};
    \node[structure, operation] at (1,1)    {\tiny $u_1$};
    \node[alert,     operation] at (1.5,0)  {\tiny $c_1 \rightarrow \xmark$};
    \node[alert,     operation] at (4.5,2)  {\tiny $c_3 \rightarrow \xmark$};
    \node[example,   operation] at (4,1)    {\tiny $c_2 \rightarrow \cmark$};
    \node[alert,     operation] at (5.25,0) {\tiny $c_4 \rightarrow \xmark$};
    \node[example,   operation] at (8,2)    {\tiny $c_6 \rightarrow \cmark$};
    \node[example,   operation] at (8.5,1)  {\tiny $c_7 \rightarrow \cmark$};
    \node[example,   operation] at (7.5,0)  {\tiny $c_5 \rightarrow \cmark$};
  \end{tikzpicture}
  
  \vspace{-1mm}
  \begin{alertblock}{Remarques}
    \vspace{-1mm}
    Eventual consistency ne donne aucune garantie :
    \begin{itemize}
    \item Ni sur ce qu'il se passe s'il y a une infinité de mises à jour
    \item Ni sur l'état de convergence
    \item Ni sur le temps nécessaire pour atteindre la convergence
    \end{itemize}
  \end{alertblock}

  \on[bottom=-2mm, text]{
    \begin{citing}
    \item[V09] W. Vogels. \emph{Eventually Consistent}. CACM 2009
    \end{citing}
  }  
  
\end{frame}

\endgroup
\endinput

% SPDX-License-Identifier: CC-BY-SA-4.0
% Author: Matthieu Perrin
% Part: 
% Section: 
% Sub-section: 
% Frame: 

\begingroup

\begin{frame}{Vitesse de convergence : Facebook Messenger}

  \on[y=-5mm]{
    \begin{tikzpicture}
      \node[faded background picture=Seine,  text width=\paperwidth/2] (A) at (-\paperwidth/4,0) {};
      \node[faded background picture=Rue, text width=\paperwidth/2] (B) at ( \paperwidth/4,0) {};
      \node[anchor=south, nosep] at (A.south) {\includegraphics[height=22mm]{Alice}};
      \node[anchor=south, nosep] at (B.south) {\includegraphics[height=22mm]{Bob}};
      
      \node at (0,2.5) {\includegraphics[width=8mm]{Facebook}};
      \draw<2->[fill=white] (0,0.5) circle(0.5);
      \onlyb<2>{
        \node at (0,0.5) {\LARGE \showclock{4}{0}};
      }
      \only<3>{
        \node at (0,0.5) {\LARGE \showclock{8}{0}};
      }
    \end{tikzpicture}
  }

  \on[x=-.25\paperwidth, y=10mm]{
    \begin{chat}[color=structure]{Bob}
      \only{\chatSend{Pause café ?}}
      \only<3>{\chatRecv[color=example]{Évidemment !}}
      \only{\chatSend{Tu n'as pas répondu. \\ Tu m'en veux ?}}
      \onlyb<2>{\chatRecv[color=example]{Évidemment !}}
      \onlyb<1>{\node at (0,-25mm) {\Large ?};}
    \end{chat}
  }

  \on[x=.25\paperwidth, y=10mm]{
    \begin{chat}[color=example]{Alice}
      \only{\chatRecv[color=structure]{Pause café ?}}
      \only{\chatSend{Évidemment !}}
      \only<2->{\chatRecv[color=structure]{Tu n'as pas répondu. \\ Tu m'en veux ?}}
      \onlyb<1>{\node at (0,-25mm) {\Large ?};}
    \end{chat}
  }

\end{frame}

\endgroup
\endinput

% SPDX-License-Identifier: CC-BY-SA-4.0
% Author: Matthieu Perrin
% Part: 
% Section: 
% Sub-section: 
% Frame: 

\begingroup

\SetKwFunction{Append}{append}
\SetKwFunction{Read}{read}
\SetKwData{State}{state}

\begin{frame}{Essai d'implémentation d'un service de messagerie}

  \on[top=-2mm]{
    \begin{algorithm}[H]
      \LVariables{}{
        $\State_i \leftarrow \varepsilon$\;
      }

      \Method{$\Read()$}{
        \Return $\State_i$\;
      }

      \Method{$\Append(m)$}{
        \For{$j$ \From $1$ \To $n$}{
          \Send $\textsc{append}(m)$ \To $p_j$\;
        }
      }
      
      \When{\Receive $\textsc{append}(m)$ \From $p_j$}{
        $\State_i \leftarrow \State_i \cdot m$\;
      }
    \end{algorithm}
  }

  \onExampleBlock[y=-14mm]{Exemples d'exécution}{}
  
  \ob<1>[y=-30mm]{
    \begin{tikzpicture}[y=8mm]
      \draw[process] (0,2) node[left]{$p_1$} node[replica below]{$\varepsilon$} to (10,2);
      \draw[process] (0,1) node[left]{$p_2$} node[replica]      {$\varepsilon$} to (10,1);
      \draw[process] (0,0) node[left]{$p_3$} node[replica]      {$\varepsilon$} to (10,0);

      \node[alert, operation] (i1) at (3,1) {\hspace{8mm}$\Append(a)$};
      \draw[alert, message]   ([xshift=0mm]i1.west) to             (3,2);
      \draw[alert, message]   ([xshift=2mm]i1.west) to[bend left]  ([xshift=6mm]i1.west);
      \draw[alert, message]   ([xshift=8mm]i1.west) to             (4,0);

      \node[structure, operation] (r1) at (7,2) {$\Read() \rightarrow \alert{a}$};
      \node[structure, operation] (r2) at (7,1) {$\Read() \rightarrow \alert{a}$};
      \node[structure, operation] (r3) at (7,0) {$\Read() \rightarrow \alert{a}$};
    \end{tikzpicture}
  }

  \ob<2>[y=-30mm]{
    \begin{tikzpicture}[y=8mm]
      \draw[process] (0,2) node[left]{$p_1$} node[replica below]{$\varepsilon$} to (10,2);
      \draw[process] (0,1) node[left]{$p_2$} node[replica]      {$\varepsilon$} to (10,1);
      \draw[process] (0,0) node[left]{$p_3$} node[replica]      {$\varepsilon$} to (10,0);

      \node[alert, operation] (i1) at (2,2) {\hspace{8mm}$\Append(a)$};
      \draw[alert, message]   ([xshift=0mm]i1.west) to[bend left]  ([xshift=4mm]i1.west);
      \draw[alert, message]   ([xshift=6mm]i1.west) to             (1.3,1);
      \draw[alert, message]   ([xshift=8mm]i1.west) to[bend below] (6.5,0);

      \node[example, operation] (i2) at (2,0) {\hspace{8mm}$\Append(b)$};
      \draw[example, message]   ([xshift=0mm]i2.west) to[bend above]    (6.5,2);   
      \draw[example, message]   ([xshift=2mm]i2.west) to                (1.7,1); 
      \draw[example, message]   ([xshift=4mm]i2.west) to[bend right=15] ([xshift=8mm]i2.west); 
      
      \node[structure, operation] (r1) at (5,2) {$\Read() \rightarrow \alert{a}$};
      \node[structure, operation] (r2) at (8,2) {$\Read() \rightarrow \alert{a} \cdot \example{b}$};
      \node[structure, operation] (r3) at (5,0) {$\Read() \rightarrow \example{b}$};
      \node[structure, operation] (r4) at (8,0) {$\Read() \rightarrow \example{b} \cdot \alert{a}$};
    \end{tikzpicture}
  }

  \on<3>[y=-30mm]{
    \begin{tikzpicture}[y=8mm]
      \draw[process] (0,2) node[left]{$p_1$} node[replica below]{$\varepsilon$} to (10,2);
      \draw[crashed] (0,1) node[left]{$p_2$} node[replica]      {$\varepsilon$} to (2,1);
      \draw[process] (0,0) node[left]{$p_3$} node[replica]      {$\varepsilon$} to (10,0);
      
      \begin{scope}[background]
        \node[alert, operation] (i1) at (2.5,1) {\hspace{8mm}$\Append(a)$};
        \draw[alert, message]   ([xshift=0mm]i1.west) to             (3,2);
        \draw[alert, message]   ([xshift=2mm]i1.west) to[bend left]  ([xshift=6mm]i1.west);
      \end{scope}

      \node[structure, operation] (r1) at (7.5,2) {$\Read() \rightarrow \alert{a}$};
      \node[structure, operation] (r2) at (7.5,0) {$\Read() \rightarrow \varepsilon$};
    \end{tikzpicture}
  }

  \onAlertBlock<2->[top, right=.49\textwidth]{Problèmes de cohérence}{
    Désaccord possible sur :
    \begin{itemize}
    \item<2-> l'ordre des commandes
    \item<3-> l'ensemble de commandes
    \end{itemize}
  }
  
\end{frame}

\endgroup
\endinput


  \ob<2>[y=-30mm]{
    \begin{tikzpicture}[y=8mm]
      \draw[process] (0,2) node[left]{$p_1$} node[replica below]{$\varepsilon$} to (10,2);
      \draw[process] (0,1) node[left]{$p_2$} node[replica]      {$\varepsilon$} to (10,1);
      \draw[process] (0,0) node[left]{$p_3$} node[replica]      {$\varepsilon$} to (10,0);

      \node[alert, operation] (i1) at (3,1) {\hspace{8mm}$\Append(a)$};
      \draw[alert, message]   ([xshift=0mm]i1.west) to             (3,2);
      \draw[alert, message]   ([xshift=2mm]i1.west) to[bend left]  ([xshift=6mm]i1.west);
      \draw[alert, message]   ([xshift=8mm]i1.west) to[bend below] (7.5,0);

      \node[structure, operation] (r1) at (7,2) {$\Read() \rightarrow \alert{a}$};
      \node[structure, operation] (r3) at (6,0) {$\Read() \rightarrow \varepsilon$};
    \end{tikzpicture}
  }

  \ob<4>[y=-30mm]{
    \begin{tikzpicture}[y=8mm]
      \draw[process] (0,2) node[left]{$p_1$} node[replica below]{$\varepsilon$} to (10,2);
      \draw[process] (0,1) node[left]{$p_2$} node[replica]      {$\varepsilon$} to (10,1);
      \draw[process] (0,0) node[left]{$p_3$} node[replica]      {$\varepsilon$} to (10,0);

      \node[alert, operation] (i1) at (2,1) {\hspace{8mm}$\Append(a)$};
      \draw[alert, message]   ([xshift=0mm]i1.west) to[bend above] (6.7,2);
      \draw[alert, message]   ([xshift=2mm]i1.west) to[bend left]  ([xshift=6mm]i1.west);
      \draw[alert, message]   ([xshift=8mm]i1.west) to             (2,0);

      \node[example, operation] (i2) at (5,1) {\hspace{8mm}$\Append(b)$};
      \draw[example, message]   ([xshift=0mm]i2.west) to             (3.9,2);   
      \draw[example, message]   ([xshift=2mm]i2.west) to[bend left]  ([xshift=6mm]i2.west); 
      \draw[example, message]   ([xshift=8mm]i2.west) to             (5,0);   
      
      \node[structure, operation] (r1) at (7  ,0) {$\Read() \rightarrow \alert{a} \cdot \example{b}$};
      \node[structure, operation] (r2) at (5.2,2) {$\Read() \rightarrow \example{b}$};
      \node[structure, operation] (r3) at (8.2,2) {$\Read() \rightarrow \example{b} \cdot \alert{a}$};
    \end{tikzpicture}
  }

 
\section{Diffusion fiable de messages}
 
\subsection{Communication fiable entre les processus}
% SPDX-License-Identifier: CC-BY-SA-4.0
% Author: Matthieu Perrin
% Part: 
% Section: 
% Sub-section: 
% Frame: 

\begingroup

\SetKwFunction{Insert}{insert}
\SetKwFunction{Read}{read}
\SetKwData{State}{state}

\begin{frame}{Diffusion de message}

  \begin{block}{Notion de diffusion de message}
    Quand $p_i$ veut exécuter une commande, il averti toutes les répliques en \structure{diffusant} la commande.
    \begin{algorithm}[H]
      \Interface{\textsc{broadcast}}{
        \lMethod{\Broadcast$(m)$}{\tcp*[f]{Envoie $m$ à tous les processus}}
        \lEvent{\Deliver $m$ \From $p_j$}{\tcp*[f]{Reçoit le message $m$ de la part de $p_j$}}
        \Method{\SBroadcast$(m)$}{
          \Broadcast$(m)$;
          \Wait{$m$ delivered locally}\;
        }
      }
    \end{algorithm}
  \end{block}

  \begin{block}{L'implémentation la plus simple : Send-To-All}
    \begin{algorithm}[H]
      \Algo{\textsc{sta}}{
        \Method{\textsc{sta}.\Broadcast$(m)$}{
          \For{$j$ \From $1$ \To $n$}{
            \Send $\textsc{m}(m)$ \To $p_j$\;
          }
        }
        \When{\Receive $\textsc{m}(m)$ \From $p_j$}{
          \textsc{sta}.\Deliver $m$ \From $p_j$\;
        }
      }
    \end{algorithm}
  \end{block}
  
\end{frame}

\endgroup
\endinput

% SPDX-License-Identifier: CC-BY-SA-4.0
% Author: Matthieu Perrin
% Part: 
% Section: 
% Sub-section: 
% Frame: 

\begingroup

\begin{frame}{Problème d'ordre FIFO}

  \on[y=-5mm]{
    \begin{tikzpicture}
      \node     [faded background picture=Seine,  text width=\paperwidth/2] (A) at (-\paperwidth/4,0) {};
      \node<1>  [faded background picture=Jardin, text width=\paperwidth/2] (B) at ( \paperwidth/4,0) {};
      \node<2,3>[faded background picture=Metro,  text width=\paperwidth/2] (B) at ( \paperwidth/4,0) {};
      \node<4>  [faded background picture=Rue,    text width=\paperwidth/2] (B) at ( \paperwidth/4,0) {};
      \node[anchor=south, nosep] at (A.south) {\includegraphics[height=22mm]{Alice}};
      \node[anchor=south, nosep] at (B.south) {\includegraphics[height=22mm]{Bob}};
    \end{tikzpicture}
  }

  \on[x=-\paperwidth/4, y=15mm]{
    \begin{chat}[color=structure]{Bob}
      \chatSend{Qu'est ce que tu es musclé !}
      \only<4->{\chatRecv[color=example]{Toi tu es très belle}}
      \only<4->{\chatRecv[color=example]{Hélas, ça n'a pas toujours été le cas}}
    \end{chat}
  }
 
  \on[x=\paperwidth/4, y=15mm]{
    \begin{chat}[color=example]{Alice}
      \chatRecv[color=structure]{Qu'est ce que tu es musclé !}
      \only<2->{\chatSend{Hélas, ça n'a pas toujours été le cas}}
      \only<3->{\chatSend{Toi tu es très belle}}
    \end{chat}
  }

  \ob<1>[x=-\paperwidth/4, y=-15mm]{
    \chatBubble[color=structure]{Qu'est ce que tu es musclé !}
  }
  
  \ob<2>[x=\paperwidth/4, y=-15mm]{
    \chatBubble[color=example]{Hélas, ça n'a pas toujours été le cas}
  }
 
  \ob<3>[x=\paperwidth/4, y=-15mm]{
    \chatBubble[color=example]{Toi tu es très belle}
  }
   
\end{frame}

\endgroup
\endinput

% SPDX-License-Identifier: CC-BY-SA-4.0
% Author: Matthieu Perrin
% Part: 
% Section: 
% Sub-section: 
% Frame: 

\begingroup

\begin{frame}{Total-Order Broadcast (Atomic Broadcast)}

  \onBlock[top=-5mm]{Propriétés héritées de \textsc{sta}.\Broadcast}{
    \vspace{-1mm}
    \begin{description}[Total-ordering :]
    \item[TO-Validité :]    Si $p_i$ \textsc{to}.\Deliver $m$ de $p_j$, $p_j$ a \textsc{to}.\Broadcast $m$
    \item[TO-Intégrité :]   $p_i$ \textsc{to}.\Deliver $m$ au plus une fois
    \item[TO-Terminaison :] \textsc{to}.\Broadcast par correct $\Rightarrow$ \textsc{to}.\Deliver par corrects.
    \end{description}
  }

  \onAlertBlock[y=8mm]{Nouvelle propriété d'ordre}{
    \vspace{-1mm}
    \begin{description}[Total-ordering :]
    \item[Total-ordering :] Si un processus \textsc{to}.\Deliver $m$ avant $m'$, \\aucun processus ne \textsc{to}.\Deliver $m'$ avant $m$. 
    \end{description}
  }

  \onBlock[bottom=7mm]{Remarques}{
    \vspace{-1mm}
    \begin{itemize}
    \item Total-Ordering $\Rightarrow$ Causal-Ordering $\Rightarrow$ FIFO-Ordering
    \item Total-Ordering $\Rightarrow$ Mutual-Ordering 
    \item \structure{Ajout de propriétés :} \textsc{rb-to}.\Broadcast, \textsc{urb-to}.\Broadcast
    \end{itemize}
  }

  \on[y=-10mm, x=10mm]{
    \begin{tikzpicture}[y=3mm]
      \draw[process]   (0,3) node[left]{$p_1$} to (6,3);
      \draw[process]   (0,2) node[left]{$p_2$} to (6,2);
      \draw[process]   (0,1) node[left]{$p_3$} to (6,1);
      \draw[process]   (0,0) node[left]{$p_4$} to (6,0);

      \draw[structure]          (1,3) node (m1) {} node[above] {$m$};
      \draw[structure, message] (m1.center) to[bend left]  (2,3);
      \draw[structure, message] (m1.center) to[bend left]  (1.5,2);
      \draw[structure, message] (m1.center) to[bend left]  (3,1);
      \draw[structure, message] (m1.center) to[bend left]  (2.5,0);

      \draw[example]            (1,0) node (m2) {} node[below] {$m'$};
      \draw[example, message]   (m2.center) to[bend right] (2.5,3);
      \draw[example, message]   (m2.center) to[bend right] (3,2);
      \draw[example, message]   (m2.center) to[bend right] (1.5,1);
      \draw[example, message]   (m2.center) to[bend right] (2,0);

      \draw[alert, thick] (3.5,3.25) -- (5.5,-.25);
      \draw[alert, thick] (5.5,3.25) -- (3.5,-.25);
    \end{tikzpicture}
  }
  
  \on[bottom=-2mm, text]{
    \begin{citing}
    \item[BJ87]  K. P. Birman, T. A. Joseph. \emph{Reliable Communication in the Presence of Failures}. TOCS, 1987. \vspace{-1mm}
    \item[DSU04] X. Défago, A. Schiper, P. Urbán. \emph{Total Order Broadcast and Multicast Algorithms: Taxonomy and Survey}. ACM CSUR, 2004.
    \end{citing}
  }  

\end{frame}

\endgroup
\endinput

% SPDX-License-Identifier: CC-BY-SA-4.0
% Author: Matthieu Perrin
% Part: 
% Section: 
% Sub-section: 
% Frame: 

\begingroup

\SetKwData{Clock}{clock}
\SetKwData{Delivered}{delivered}
\SetKwData{Tick}{tick}
\SetKwData{Now}{now}
\SetKwData{Update}{update}

\begin{frame}{Implémentation de FIFO Broadcast}

  \ob<-2>[top]{
    \begin{algorithm}[H]
      \lLVariables{}{$\Delivered_i \leftarrow [0, ..., 0]$}
      \Method{$\textsc{fifo}.\Broadcast(m)$}{
        \textsc{sta}.\SBroadcast $\textsc{msg}(m, \Delivered_i[i])$;
      }
      \When{\textsc{sta}.\Deliver $\textsc{msg}(m, \mathit{sn}_j)$ \From $p_j$}{
        \Wait{$\Delivered_i[j] = \mathit{sn}_j$};\\
        \textsc{fifo}.\Deliver $m$ \From $p_j$;\\
        $\Delivered_i[j] \leftarrow \mathit{sn}_j+1$;
      }
    \end{algorithm}
  }

  \on<3>[top]{
    \begin{algorithm}[H]
      \lLVariables{}{\Example{$\Clock_i$;}\tcp*[f]{$\Delivered_i[1..n]$}}
      \Method{$\textsc{fifo}.\Broadcast(m)$}{
        \textsc{sta}.\SBroadcast $\textsc{msg}(m,~\Example{\Clock_i.\Now(i)})$;\tcp*[r]{$\Delivered_i[i]$}
      }
      \When{\textsc{sta}.\Deliver $\textsc{msg}(m, \mathit{date}_j)$ \From $p_j$}{
        \Wait{\Example{$\mathit{date}_j \ge \Clock_i.\Now(j)$}};\tcp*[r]{$\Delivered_i[j]$}
        \textsc{fifo}.\Deliver $m$ \From $p_j$;\\
        \Example{$\Clock_i.\Update(j)$};\tcp*[r]{$\Delivered_i[j] \leftarrow \mathit{sn}+1$}
      }
    \end{algorithm}
  }

  \obExampleBlock<1>[anchor=north]{Exemple d'exécution}{}
  \ob<1>[y=-25mm]{
    \begin{tikzpicture}[y=12mm]
      \draw[process] (0,1) node[left]{$p_1$}                                            to (10,1);
      \draw[process] (0,0) node[left]{$p_2$} node[replica below]{$\Delivered_2[1] = 0$} to (10,0);

      \draw[alert]   (1,1) node (m1) {} node[above] {$m$} ;
      \draw[alert, message]     (m1.center) to[bend left]   (2,1) ;
      \draw[alert, message]     (m1.center) to node[swap]{$sn=0$} (4.5,0) node[replica below]{$1$};

      \draw[example] (3,1) node (m2) {} node[above] {$m'$} ;
      \draw[example, message]   (m2.center) to[bend left]   (4,1) ;
      \draw[example, message]   (m2.center) to node{$sn=1$} (3.5,.05) to[bend left] (5.5,0) node[replica below]{$2$} ;
    \end{tikzpicture}
  }

  \onBlock<2->[anchor=north]{Remarques}{
    \begin{itemize}
    \item FIFO-Ordering : pour tout $j$, $\Delivered_i[j]$ = prochain $sn$ attendu de $p_j$.
    \item On peut renforcer les propriétés en remplaçant \textsc{sta} par \textsc{rb} ou \textsc{urb}
    \item En pratique, \Wait est implémenté par des buffers
    \item<3> Pour tout processus $p_j$, $\Delivered_i[j]$ implémente une horloge logique
      \begin{itemize}
      \item Le temps logique est l'ordre de processus
      \item Les dates sont des entiers (horloge scalaire)
      \end{itemize}
    \end{itemize}
  }
  
\end{frame}

\endgroup
\endinput

 
\subsection{Implémentation d'eventual consistency}
% SPDX-License-Identifier: CC-BY-SA-4.0
% Author: Matthieu Perrin
% Part: 
% Section: 
% Sub-section: 
% Frame: 

\begingroup

\begin{frame}{Problème d'ordre FIFO}

  \on[y=-5mm]{
    \begin{tikzpicture}
      \node     [faded background picture=Seine,  text width=\paperwidth/2] (A) at (-\paperwidth/4,0) {};
      \node<1>  [faded background picture=Jardin, text width=\paperwidth/2] (B) at ( \paperwidth/4,0) {};
      \node<2,3>[faded background picture=Metro,  text width=\paperwidth/2] (B) at ( \paperwidth/4,0) {};
      \node<4>  [faded background picture=Rue,    text width=\paperwidth/2] (B) at ( \paperwidth/4,0) {};
      \node[anchor=south, nosep] at (A.south) {\includegraphics[height=22mm]{Alice}};
      \node[anchor=south, nosep] at (B.south) {\includegraphics[height=22mm]{Bob}};
    \end{tikzpicture}
  }

  \on[x=-\paperwidth/4, y=15mm]{
    \begin{chat}[color=structure]{Bob}
      \chatSend{Qu'est ce que tu es musclé !}
      \only<4->{\chatRecv[color=example]{Toi tu es très belle}}
      \only<4->{\chatRecv[color=example]{Hélas, ça n'a pas toujours été le cas}}
    \end{chat}
  }
 
  \on[x=\paperwidth/4, y=15mm]{
    \begin{chat}[color=example]{Alice}
      \chatRecv[color=structure]{Qu'est ce que tu es musclé !}
      \only<2->{\chatSend{Hélas, ça n'a pas toujours été le cas}}
      \only<3->{\chatSend{Toi tu es très belle}}
    \end{chat}
  }

  \ob<1>[x=-\paperwidth/4, y=-15mm]{
    \chatBubble[color=structure]{Qu'est ce que tu es musclé !}
  }
  
  \ob<2>[x=\paperwidth/4, y=-15mm]{
    \chatBubble[color=example]{Hélas, ça n'a pas toujours été le cas}
  }
 
  \ob<3>[x=\paperwidth/4, y=-15mm]{
    \chatBubble[color=example]{Toi tu es très belle}
  }
   
\end{frame}

\endgroup
\endinput

% SPDX-License-Identifier: CC-BY-SA-4.0
% Author: Matthieu Perrin
% Part: Communication
% Section: Cohérence faible
% Frame: Eventual Consistency

\begingroup

\SetKwFunction{Get}{get}
\SetKwFunction{Add}{add}
\SetKwFunction{Insert}{insert}
\SetKwFunction{SetKey}{setKey}

\begin{frame}{CRDT : Conflict-free Replicated Data Type}

  \vspace{-3mm}
  \begin{block}{Définition -- Conflict-free Replicated Data Type}
    Soit $M=\langle C, R, Q, q_0, \tau, \rho \rangle$ une machine à états à requêtes/mises à jour.
    \begin{itemize}
    \item On dit que deux mises à jour $u, u'$ \structure{commutent} si :
      \begin{center}
        $\alert{\forall q\in Q, \tau(\tau(q, u), u') = \tau(\tau(q, u'), u)}$
      \end{center}
    \item On dit que $M$ est un \structure{CRDT} si toutes ses mises à jour commutent :
      \begin{center}
        $\alert{\forall u, u'\in \mathit{Update}, \forall q\in Q, \tau(\tau(q, u), u') = \tau(\tau(q, u'), u)}$
      \end{center}
    \end{itemize}
  \end{block}

  \pause

  \vspace{-1mm}
  \begin{exampleblock}{Exemples et contre-exemples}
  \vspace{-1mm}
    \begin{itemize}
    \item<2-> Le compteur $\Get/\Add$ est un CRDT :
      \begin{center}
        $\example{\forall n, a, b, \tau(\tau(n, \Add(a)), \Add(b)) = n+a+b = \tau(\tau(n, \Add(b)), \Add(a))}$
      \end{center}
    \item<3-> Le G-set $\Get/\Insert$ est un CRDT (l'insertion commute)
    \item<4-> La base de données clé-valeurs n'est pas un CRDT :
      \begin{center}
        \example{$\SetKey(k, a)$ écrase $\SetKey(k, b)$}
      \end{center}
    \item<5-> Les services de messagerie instantanée ne sont pas des CRDT.
    \end{itemize}
  \end{exampleblock}

  \footnoteref{M. Shapiro, N. Preguiça, C. Baquero, M. Zawirski. \textit{A comprehensive study of Convergent and Commutative Replicated Data Types}. (2011)}
  
\end{frame}

\endgroup
\endinput

% SPDX-License-Identifier: CC-BY-SA-4.0
% Author: Matthieu Perrin
% Part: 
% Section: 
% Sub-section: 
% Frame: 

\begingroup

\SetKwData{Clock}{clock}
\SetKwData{Delivered}{delivered}
\SetKwData{Tick}{tick}
\SetKwData{Now}{now}
\SetKwData{Update}{update}

\begin{frame}{Implémentation de FIFO Broadcast}

  \ob<-2>[top]{
    \begin{algorithm}[H]
      \lLVariables{}{$\Delivered_i \leftarrow [0, ..., 0]$}
      \Method{$\textsc{fifo}.\Broadcast(m)$}{
        \textsc{sta}.\SBroadcast $\textsc{msg}(m, \Delivered_i[i])$;
      }
      \When{\textsc{sta}.\Deliver $\textsc{msg}(m, \mathit{sn}_j)$ \From $p_j$}{
        \Wait{$\Delivered_i[j] = \mathit{sn}_j$};\\
        \textsc{fifo}.\Deliver $m$ \From $p_j$;\\
        $\Delivered_i[j] \leftarrow \mathit{sn}_j+1$;
      }
    \end{algorithm}
  }

  \on<3>[top]{
    \begin{algorithm}[H]
      \lLVariables{}{\Example{$\Clock_i$;}\tcp*[f]{$\Delivered_i[1..n]$}}
      \Method{$\textsc{fifo}.\Broadcast(m)$}{
        \textsc{sta}.\SBroadcast $\textsc{msg}(m,~\Example{\Clock_i.\Now(i)})$;\tcp*[r]{$\Delivered_i[i]$}
      }
      \When{\textsc{sta}.\Deliver $\textsc{msg}(m, \mathit{date}_j)$ \From $p_j$}{
        \Wait{\Example{$\mathit{date}_j \ge \Clock_i.\Now(j)$}};\tcp*[r]{$\Delivered_i[j]$}
        \textsc{fifo}.\Deliver $m$ \From $p_j$;\\
        \Example{$\Clock_i.\Update(j)$};\tcp*[r]{$\Delivered_i[j] \leftarrow \mathit{sn}+1$}
      }
    \end{algorithm}
  }

  \obExampleBlock<1>[anchor=north]{Exemple d'exécution}{}
  \ob<1>[y=-25mm]{
    \begin{tikzpicture}[y=12mm]
      \draw[process] (0,1) node[left]{$p_1$}                                            to (10,1);
      \draw[process] (0,0) node[left]{$p_2$} node[replica below]{$\Delivered_2[1] = 0$} to (10,0);

      \draw[alert]   (1,1) node (m1) {} node[above] {$m$} ;
      \draw[alert, message]     (m1.center) to[bend left]   (2,1) ;
      \draw[alert, message]     (m1.center) to node[swap]{$sn=0$} (4.5,0) node[replica below]{$1$};

      \draw[example] (3,1) node (m2) {} node[above] {$m'$} ;
      \draw[example, message]   (m2.center) to[bend left]   (4,1) ;
      \draw[example, message]   (m2.center) to node{$sn=1$} (3.5,.05) to[bend left] (5.5,0) node[replica below]{$2$} ;
    \end{tikzpicture}
  }

  \onBlock<2->[anchor=north]{Remarques}{
    \begin{itemize}
    \item FIFO-Ordering : pour tout $j$, $\Delivered_i[j]$ = prochain $sn$ attendu de $p_j$.
    \item On peut renforcer les propriétés en remplaçant \textsc{sta} par \textsc{rb} ou \textsc{urb}
    \item En pratique, \Wait est implémenté par des buffers
    \item<3> Pour tout processus $p_j$, $\Delivered_i[j]$ implémente une horloge logique
      \begin{itemize}
      \item Le temps logique est l'ordre de processus
      \item Les dates sont des entiers (horloge scalaire)
      \end{itemize}
    \end{itemize}
  }
  
\end{frame}

\endgroup
\endinput

% SPDX-License-Identifier: CC-BY-SA-4.0
% Author: Matthieu Perrin
% Part: Communication
% Section: Cohérence faible
% Frame: Eventual Consistency

\begingroup

\SetKwFunction{Read}{read}
\SetKwFunction{Write}{write}

\begin{frame}{Implémentation d'une mémoire partagée}
  
  \onBlock[top=-1mm]{Spécification d'une mémoire}{
    On cherche à implémenter une \structure{mémoire} constituée de \structure{registres}
    \begin{algorithm}[H]
      \Interface{\textsc{Register}<$\mathcal{V}$>}{
        \Method{$\Read() \in \mathcal{V}$}{\tcp*[h]{\Return $q$}}
        \Method{$\Write(v \in \mathcal{V})$}{\tcp*[h]{$q \leftarrow v$}}
      }
    \end{algorithm}
  }
  
  \onAlertBlock[y=-10mm, anchor=north]{Problème : $\Write(v_1)$ et $\Write(v_2)$ ne commutent pas}{
    \vspace{2mm}
    \centering
    \begin{tikzpicture}[y=10mm, anchor=center]
      \draw[process] (0,1) node[left]{$p_1$} to (10,1);
      \draw[process] (0,0) node[left]{$p_2$} to (10,0);

      \node[alert,     operation] (i1) at (2,1) {$\Write(1)$};
      \path[alert, message]       (i1.west) edge[bend left]  (i1.east);
      \path[alert, message]       (i1.west) edge             (4,0);

      \node[example, operation] (d2) at (2,0) {$\Write(2)$};
      \path[example, message]   (d2.west) edge             (4,1);
      \path[example, message]   (d2.west) edge[bend right] (d2.east);

      \node[structure, operation] (r1) at (6,1) {$\Read() \rightarrow 2$};
      \node[structure, operation] (r2) at (6,0) {$\Read() \rightarrow 1$};
    \end{tikzpicture}
  }

\end{frame}

\endgroup
\endinput

% SPDX-License-Identifier: CC-BY-SA-4.0
% Author: Matthieu Perrin
% Part: Communication
% Section: Cohérence faible
% Frame: Eventual Consistency

\begingroup

\SetKwFunction{Read}{read}
\SetKwFunction{Write}{write}
\SetKwData{Value}{value}
\SetKwData{Time}{time}
\SetKwData{Writer}{writer}

\begin{frame}{Registre Last-Writer-Wins (LWW)}

  \onAlertBlock<1>[top=-3mm, right=.42\textwidth]{Estampille de Lamport}{
    \vspace{-5mm}
    $$\alert{\langle \Time_i, \Writer_i \rangle}$$
    \vspace{-5mm}
    \begin{description}[$\Writer_i$ :]
    \item[$\Time_i$ :] horloge de Lamport
    \item[$\Writer_i$ :] processus écrivain
    \end{description}

    \begin{itemize}
    \item Compatible avec $\hb$
    \item Identifie chaque écriture
    \item Ordre lexicographique total
    \end{itemize}
  }

  \obBlock<2>[top=-3mm, right=.42\textwidth]{Cohérence séquentielle}{
    Une exécution est \structure{séquentiellement cohérente} si son résultat observable est
    le même que celui d’un entrelacement de son ordre de processus.
  }

  \obBlock<3->[top=-3mm, right=.42\textwidth]{Absence de composabilité}{
    \begin{itemize}
    \item CS n'est pas composable
    \item Quelle cohérence pour la mémoire?
    \end{itemize}
  }
  
  \on[top]{
    \begin{algorithm}[H]
      \LVariables{}{
        $\Value_i  \leftarrow \bot$; \Alert<1>{$\Time_i   \leftarrow 0$; $\Writer_i \leftarrow 0$};
      }
      \lMethod{$\Read()$}{\Return $\Value_i$;}
      \Method{$\Write(v)$}{
        \textsc{rb}.\SBroadcast $\textsc{w}(v, ~\Alert<1>{\Time_i+1})$;
      }
      \When{\textsc{rb}.\Deliver $\textsc{w}(v, t)$ \From $p_j$}{
        \If{\Alert<1>{$\Time_i < t \lor (\Time_i = t \land \Writer_i < j)$}}{
          $\Value_i  \leftarrow v$; \Alert<1>{$\Time_i \leftarrow t$; $\Writer_i \leftarrow j$};
        }
      }
    \end{algorithm}
  }
  
  \obExampleBlock<1>[y=-2mm, anchor=north]{Exemple d'exécution}{
    \centering
    \begin{tikzpicture}[y=10mm]
      \draw[process] (0,2) node[left]{$p_1$} to (10,2);
      \draw[process] (0,1) node[left]{$p_2$} to (10,1);
      \draw[process] (0,0) node[left]{$p_3$} to (10,0);

      \begin{scope}[background]
        \fill[alert!30]     (3,2.1) rectangle ( 6,1.9);
        \fill[structure!30] (6,2.1) rectangle (10,1.9);
        \fill[alert!30]     (2,1.1) rectangle ( 4,0.9);
        \fill[example!30]   (4,1.1) rectangle ( 7,0.9);
        \fill[structure!30] (7,1.1) rectangle (10,0.9);
        \fill[example!30]   (4,0.1) rectangle ( 8,-.1);
        \fill[structure!30] (8,0.1) rectangle (10,-.1);
      \end{scope}
      
      \node[structure, operation]     (wc) at (6,2) {$\Write(c)$};
      \node[structure, below left] at (wc.west) {$\langle 2, 1 \rangle$};
      \path[structure, message]       (wc.west) edge[bend left]  (wc.east);
      \path[structure, message]       (wc.west) edge             (7,1);
      \path[structure, message]       (wc.west) edge             (8,0);
      
      \node[alert, operation]         (wa) at (2,1) {$\Write(a)$};
      \node[alert, below left] at     (wa.west) {$\langle 1, 2 \rangle$};
      \path[alert, message]           (wa.west) edge             (3,2);
      \path[alert, message]           (wa.west) edge[bend left]  (wa.east);
      \path[alert, message]           (wa.west) edge[bend below] (6,0);
      
      \node[example, operation]       (wb) at (4,0) {$\Write(b)$};
      \node[example, above left] at   (wb.west) {$\langle 1, 3 \rangle$};
      \path[example, message]         (wb.west) edge             (8,2);
      \path[example, message]         (wb.west) edge             (4,1);
      \path[example, message]         (wb.west) edge[bend right] (wb.east);
    \end{tikzpicture}
  }

  \onAlertBlock<2>[y=-2mm, anchor=north]{Théorème -- Le registre LWW est séquentiellement cohérent}{
    \begin{itemize}
    \item Soit \structure{$ts(o)$} : \alert{$ts(\Write(v)) = \langle \Time_i+1, i \rangle$} et \alert{$ts(\Read()) = \langle \Time_i, \Writer_i \rangle$}
    \item Soit \structure{$o \lessdot_1 o'$} si \alert{$ts(o) < ts(o')$}
    \item Soit \structure{$o \lessdot_2 o'$} si \alert{$ts(o) = ts(o')$}, \alert{$o=\Write(v)$} et \alert{$o'=\Read()$}
    \item \alert{$\lessdot_1 \cup \lessdot_2 \cup \xrightarrow{po}$} est un ordre partiel strict, on l'étend en un ordre total \structure{$<$}
    \item Chaque lecture retourne la dernière valeur écrite selon $<$
    \end{itemize}
  }

  \obExampleBlock<3>[y=-2mm, anchor=north]{Exemples d'exécution}{
    \centering
    \begin{tikzpicture}[y=10mm]
      \draw[process] (0,1) node[left]{$p_1$} to (10,1);
      \draw[process] (0,0) node[left]{$p_2$} to (10,0);
      
      \node[alert, operation]         (wa) at (3,1) {$x.\Write(a)$};
      \path[alert, message]           (wa.west) edge[bend left]  (wa.east);
      \path[alert, message]           (wa.west) edge[bend below] (8,0);
      \node[structure, operation]     (r1) at (6,1) {$y.\Read() \rightarrow \bot$};
      
      \node[example, operation]       (wb) at (3,0) {$y.\Write(b)$};
      \path[example, message]         (wb.west) edge[bend above] (8,1);
      \path[example, message]         (wb.west) edge[bend right] (wb.east);
      \node[structure, operation]     (r2) at (6,0) {$x.\Read() \rightarrow \bot$};
    \end{tikzpicture}
  }

  \obExampleBlock<4>[y=-2mm, anchor=north]{Exemples d'exécution -- Problème d'ordre FIFO}{
    \centering
    \begin{tikzpicture}[y=10mm]
      \draw[process] (0,1) node[left]{$p_1$} to (10,1);
      \draw[process] (0,0) node[left]{$p_2$} to (10,0);
      
      \node[alert, operation]         (wa) at (1.5,1) {$x.\Write(a)$};
      \path[alert, message]           (wa.west) edge[bend left]  (wa.east);
      \path[alert, message]           (wa.west) edge[bend below] (9.5,0);
      
      \node[example, operation]       (wb) at (4,1) {$y.\Write(b)$};
      \path[example, message]         (wb.west) edge[bend left] (wb.east);
      \path[example, message]         (wb.west) edge (3.5,0);

      \node[structure, operation]     (r1) at (5,0) {$y.\Read() \rightarrow b$};
      \node[structure, operation]     (r2) at (7.75,0) {$x.\Read() \rightarrow \bot$};
    \end{tikzpicture}
  }

  \obExampleBlock<5>[y=-2mm, anchor=north]{Exemples d'exécution -- Problème de causalité}{
    \centering
    \begin{tikzpicture}[y=10mm]
      \draw[process] (0,2) node[left]{$p_1$} to (10,2);
      \draw[process] (0,1) node[left]{$p_2$} to (10,1);
      \draw[process] (0,0) node[left]{$p_3$} to (10,0);
      
      \node[alert, operation]         (wa) at (1.25,1) {$x.\Write(a)$};
      \path[alert, message]           (wa.west) edge (.6,2);
      \path[alert, message]           (wa.west) edge[bend left]  (wa.east);
      \path[alert, message]           (wa.west) edge[bend below] (9.75,0);

      \node[structure, operation]     (r1) at (2,2) {$x.\Read() \rightarrow a$};
      \node[example, operation]       (wb) at (4.5,2) {$y.\Write(b)$};
      \path[example, message]         (wb.west) edge[bend left] (wb.east);
      \path[example, message]         (wb.west) edge            (5,1);
      \path[example, message]         (wb.west) edge            (4,0);

      \node[structure, operation]     (r2) at (5.5,0) {$y.\Read() \rightarrow b$};
      \node[structure, operation]     (r3) at (8.1,0) {$x.\Read() \rightarrow \bot$};
    \end{tikzpicture}
  }

\end{frame}

\endgroup
\endinput

 
\section{Ordre sur les messages}
 
\subsection{Ajout de l'ordre FIFO}
% SPDX-License-Identifier: CC-BY-SA-4.0
% Author: Matthieu Perrin
% Part: 
% Section: 
% Sub-section: 
% Frame: 

\begingroup

\begin{frame}{Problème d'ordre FIFO}

  \on[y=-5mm]{
    \begin{tikzpicture}
      \node     [faded background picture=Seine,  text width=\paperwidth/2] (A) at (-\paperwidth/4,0) {};
      \node<1>  [faded background picture=Jardin, text width=\paperwidth/2] (B) at ( \paperwidth/4,0) {};
      \node<2,3>[faded background picture=Metro,  text width=\paperwidth/2] (B) at ( \paperwidth/4,0) {};
      \node<4>  [faded background picture=Rue,    text width=\paperwidth/2] (B) at ( \paperwidth/4,0) {};
      \node[anchor=south, nosep] at (A.south) {\includegraphics[height=22mm]{Alice}};
      \node[anchor=south, nosep] at (B.south) {\includegraphics[height=22mm]{Bob}};
    \end{tikzpicture}
  }

  \on[x=-\paperwidth/4, y=15mm]{
    \begin{chat}[color=structure]{Bob}
      \chatSend{Qu'est ce que tu es musclé !}
      \only<4->{\chatRecv[color=example]{Toi tu es très belle}}
      \only<4->{\chatRecv[color=example]{Hélas, ça n'a pas toujours été le cas}}
    \end{chat}
  }
 
  \on[x=\paperwidth/4, y=15mm]{
    \begin{chat}[color=example]{Alice}
      \chatRecv[color=structure]{Qu'est ce que tu es musclé !}
      \only<2->{\chatSend{Hélas, ça n'a pas toujours été le cas}}
      \only<3->{\chatSend{Toi tu es très belle}}
    \end{chat}
  }

  \ob<1>[x=-\paperwidth/4, y=-15mm]{
    \chatBubble[color=structure]{Qu'est ce que tu es musclé !}
  }
  
  \ob<2>[x=\paperwidth/4, y=-15mm]{
    \chatBubble[color=example]{Hélas, ça n'a pas toujours été le cas}
  }
 
  \ob<3>[x=\paperwidth/4, y=-15mm]{
    \chatBubble[color=example]{Toi tu es très belle}
  }
   
\end{frame}

\endgroup
\endinput

% SPDX-License-Identifier: CC-BY-SA-4.0
% Author: Matthieu Perrin
% Part: 
% Section: 
% Sub-section: 
% Frame: 

\begingroup

\begin{frame}{Total-Order Broadcast (Atomic Broadcast)}

  \onBlock[top=-5mm]{Propriétés héritées de \textsc{sta}.\Broadcast}{
    \vspace{-1mm}
    \begin{description}[Total-ordering :]
    \item[TO-Validité :]    Si $p_i$ \textsc{to}.\Deliver $m$ de $p_j$, $p_j$ a \textsc{to}.\Broadcast $m$
    \item[TO-Intégrité :]   $p_i$ \textsc{to}.\Deliver $m$ au plus une fois
    \item[TO-Terminaison :] \textsc{to}.\Broadcast par correct $\Rightarrow$ \textsc{to}.\Deliver par corrects.
    \end{description}
  }

  \onAlertBlock[y=8mm]{Nouvelle propriété d'ordre}{
    \vspace{-1mm}
    \begin{description}[Total-ordering :]
    \item[Total-ordering :] Si un processus \textsc{to}.\Deliver $m$ avant $m'$, \\aucun processus ne \textsc{to}.\Deliver $m'$ avant $m$. 
    \end{description}
  }

  \onBlock[bottom=7mm]{Remarques}{
    \vspace{-1mm}
    \begin{itemize}
    \item Total-Ordering $\Rightarrow$ Causal-Ordering $\Rightarrow$ FIFO-Ordering
    \item Total-Ordering $\Rightarrow$ Mutual-Ordering 
    \item \structure{Ajout de propriétés :} \textsc{rb-to}.\Broadcast, \textsc{urb-to}.\Broadcast
    \end{itemize}
  }

  \on[y=-10mm, x=10mm]{
    \begin{tikzpicture}[y=3mm]
      \draw[process]   (0,3) node[left]{$p_1$} to (6,3);
      \draw[process]   (0,2) node[left]{$p_2$} to (6,2);
      \draw[process]   (0,1) node[left]{$p_3$} to (6,1);
      \draw[process]   (0,0) node[left]{$p_4$} to (6,0);

      \draw[structure]          (1,3) node (m1) {} node[above] {$m$};
      \draw[structure, message] (m1.center) to[bend left]  (2,3);
      \draw[structure, message] (m1.center) to[bend left]  (1.5,2);
      \draw[structure, message] (m1.center) to[bend left]  (3,1);
      \draw[structure, message] (m1.center) to[bend left]  (2.5,0);

      \draw[example]            (1,0) node (m2) {} node[below] {$m'$};
      \draw[example, message]   (m2.center) to[bend right] (2.5,3);
      \draw[example, message]   (m2.center) to[bend right] (3,2);
      \draw[example, message]   (m2.center) to[bend right] (1.5,1);
      \draw[example, message]   (m2.center) to[bend right] (2,0);

      \draw[alert, thick] (3.5,3.25) -- (5.5,-.25);
      \draw[alert, thick] (5.5,3.25) -- (3.5,-.25);
    \end{tikzpicture}
  }
  
  \on[bottom=-2mm, text]{
    \begin{citing}
    \item[BJ87]  K. P. Birman, T. A. Joseph. \emph{Reliable Communication in the Presence of Failures}. TOCS, 1987. \vspace{-1mm}
    \item[DSU04] X. Défago, A. Schiper, P. Urbán. \emph{Total Order Broadcast and Multicast Algorithms: Taxonomy and Survey}. ACM CSUR, 2004.
    \end{citing}
  }  

\end{frame}

\endgroup
\endinput

% SPDX-License-Identifier: CC-BY-SA-4.0
% Author: Matthieu Perrin
% Part: 
% Section: 
% Sub-section: 
% Frame: 

\begingroup

\SetKwData{Clock}{clock}
\SetKwData{Delivered}{delivered}
\SetKwData{Tick}{tick}
\SetKwData{Now}{now}
\SetKwData{Update}{update}

\begin{frame}{Implémentation de FIFO Broadcast}

  \ob<-2>[top]{
    \begin{algorithm}[H]
      \lLVariables{}{$\Delivered_i \leftarrow [0, ..., 0]$}
      \Method{$\textsc{fifo}.\Broadcast(m)$}{
        \textsc{sta}.\SBroadcast $\textsc{msg}(m, \Delivered_i[i])$;
      }
      \When{\textsc{sta}.\Deliver $\textsc{msg}(m, \mathit{sn}_j)$ \From $p_j$}{
        \Wait{$\Delivered_i[j] = \mathit{sn}_j$};\\
        \textsc{fifo}.\Deliver $m$ \From $p_j$;\\
        $\Delivered_i[j] \leftarrow \mathit{sn}_j+1$;
      }
    \end{algorithm}
  }

  \on<3>[top]{
    \begin{algorithm}[H]
      \lLVariables{}{\Example{$\Clock_i$;}\tcp*[f]{$\Delivered_i[1..n]$}}
      \Method{$\textsc{fifo}.\Broadcast(m)$}{
        \textsc{sta}.\SBroadcast $\textsc{msg}(m,~\Example{\Clock_i.\Now(i)})$;\tcp*[r]{$\Delivered_i[i]$}
      }
      \When{\textsc{sta}.\Deliver $\textsc{msg}(m, \mathit{date}_j)$ \From $p_j$}{
        \Wait{\Example{$\mathit{date}_j \ge \Clock_i.\Now(j)$}};\tcp*[r]{$\Delivered_i[j]$}
        \textsc{fifo}.\Deliver $m$ \From $p_j$;\\
        \Example{$\Clock_i.\Update(j)$};\tcp*[r]{$\Delivered_i[j] \leftarrow \mathit{sn}+1$}
      }
    \end{algorithm}
  }

  \obExampleBlock<1>[anchor=north]{Exemple d'exécution}{}
  \ob<1>[y=-25mm]{
    \begin{tikzpicture}[y=12mm]
      \draw[process] (0,1) node[left]{$p_1$}                                            to (10,1);
      \draw[process] (0,0) node[left]{$p_2$} node[replica below]{$\Delivered_2[1] = 0$} to (10,0);

      \draw[alert]   (1,1) node (m1) {} node[above] {$m$} ;
      \draw[alert, message]     (m1.center) to[bend left]   (2,1) ;
      \draw[alert, message]     (m1.center) to node[swap]{$sn=0$} (4.5,0) node[replica below]{$1$};

      \draw[example] (3,1) node (m2) {} node[above] {$m'$} ;
      \draw[example, message]   (m2.center) to[bend left]   (4,1) ;
      \draw[example, message]   (m2.center) to node{$sn=1$} (3.5,.05) to[bend left] (5.5,0) node[replica below]{$2$} ;
    \end{tikzpicture}
  }

  \onBlock<2->[anchor=north]{Remarques}{
    \begin{itemize}
    \item FIFO-Ordering : pour tout $j$, $\Delivered_i[j]$ = prochain $sn$ attendu de $p_j$.
    \item On peut renforcer les propriétés en remplaçant \textsc{sta} par \textsc{rb} ou \textsc{urb}
    \item En pratique, \Wait est implémenté par des buffers
    \item<3> Pour tout processus $p_j$, $\Delivered_i[j]$ implémente une horloge logique
      \begin{itemize}
      \item Le temps logique est l'ordre de processus
      \item Les dates sont des entiers (horloge scalaire)
      \end{itemize}
    \end{itemize}
  }
  
\end{frame}

\endgroup
\endinput

 
\subsection{Ajout de l'ordre causal}
% SPDX-License-Identifier: CC-BY-SA-4.0
% Author: Matthieu Perrin
% Part: 
% Section: 
% Sub-section: 
% Frame: 

\begingroup

\begin{frame}{Problème d'ordre FIFO}

  \on[y=-5mm]{
    \begin{tikzpicture}
      \node     [faded background picture=Seine,  text width=\paperwidth/2] (A) at (-\paperwidth/4,0) {};
      \node<1>  [faded background picture=Jardin, text width=\paperwidth/2] (B) at ( \paperwidth/4,0) {};
      \node<2,3>[faded background picture=Metro,  text width=\paperwidth/2] (B) at ( \paperwidth/4,0) {};
      \node<4>  [faded background picture=Rue,    text width=\paperwidth/2] (B) at ( \paperwidth/4,0) {};
      \node[anchor=south, nosep] at (A.south) {\includegraphics[height=22mm]{Alice}};
      \node[anchor=south, nosep] at (B.south) {\includegraphics[height=22mm]{Bob}};
    \end{tikzpicture}
  }

  \on[x=-\paperwidth/4, y=15mm]{
    \begin{chat}[color=structure]{Bob}
      \chatSend{Qu'est ce que tu es musclé !}
      \only<4->{\chatRecv[color=example]{Toi tu es très belle}}
      \only<4->{\chatRecv[color=example]{Hélas, ça n'a pas toujours été le cas}}
    \end{chat}
  }
 
  \on[x=\paperwidth/4, y=15mm]{
    \begin{chat}[color=example]{Alice}
      \chatRecv[color=structure]{Qu'est ce que tu es musclé !}
      \only<2->{\chatSend{Hélas, ça n'a pas toujours été le cas}}
      \only<3->{\chatSend{Toi tu es très belle}}
    \end{chat}
  }

  \ob<1>[x=-\paperwidth/4, y=-15mm]{
    \chatBubble[color=structure]{Qu'est ce que tu es musclé !}
  }
  
  \ob<2>[x=\paperwidth/4, y=-15mm]{
    \chatBubble[color=example]{Hélas, ça n'a pas toujours été le cas}
  }
 
  \ob<3>[x=\paperwidth/4, y=-15mm]{
    \chatBubble[color=example]{Toi tu es très belle}
  }
   
\end{frame}

\endgroup
\endinput

% SPDX-License-Identifier: CC-BY-SA-4.0
% Author: Matthieu Perrin
% Part: 
% Section: 
% Sub-section: 
% Frame: 

\begingroup

\begin{frame}{Total-Order Broadcast (Atomic Broadcast)}

  \onBlock[top=-5mm]{Propriétés héritées de \textsc{sta}.\Broadcast}{
    \vspace{-1mm}
    \begin{description}[Total-ordering :]
    \item[TO-Validité :]    Si $p_i$ \textsc{to}.\Deliver $m$ de $p_j$, $p_j$ a \textsc{to}.\Broadcast $m$
    \item[TO-Intégrité :]   $p_i$ \textsc{to}.\Deliver $m$ au plus une fois
    \item[TO-Terminaison :] \textsc{to}.\Broadcast par correct $\Rightarrow$ \textsc{to}.\Deliver par corrects.
    \end{description}
  }

  \onAlertBlock[y=8mm]{Nouvelle propriété d'ordre}{
    \vspace{-1mm}
    \begin{description}[Total-ordering :]
    \item[Total-ordering :] Si un processus \textsc{to}.\Deliver $m$ avant $m'$, \\aucun processus ne \textsc{to}.\Deliver $m'$ avant $m$. 
    \end{description}
  }

  \onBlock[bottom=7mm]{Remarques}{
    \vspace{-1mm}
    \begin{itemize}
    \item Total-Ordering $\Rightarrow$ Causal-Ordering $\Rightarrow$ FIFO-Ordering
    \item Total-Ordering $\Rightarrow$ Mutual-Ordering 
    \item \structure{Ajout de propriétés :} \textsc{rb-to}.\Broadcast, \textsc{urb-to}.\Broadcast
    \end{itemize}
  }

  \on[y=-10mm, x=10mm]{
    \begin{tikzpicture}[y=3mm]
      \draw[process]   (0,3) node[left]{$p_1$} to (6,3);
      \draw[process]   (0,2) node[left]{$p_2$} to (6,2);
      \draw[process]   (0,1) node[left]{$p_3$} to (6,1);
      \draw[process]   (0,0) node[left]{$p_4$} to (6,0);

      \draw[structure]          (1,3) node (m1) {} node[above] {$m$};
      \draw[structure, message] (m1.center) to[bend left]  (2,3);
      \draw[structure, message] (m1.center) to[bend left]  (1.5,2);
      \draw[structure, message] (m1.center) to[bend left]  (3,1);
      \draw[structure, message] (m1.center) to[bend left]  (2.5,0);

      \draw[example]            (1,0) node (m2) {} node[below] {$m'$};
      \draw[example, message]   (m2.center) to[bend right] (2.5,3);
      \draw[example, message]   (m2.center) to[bend right] (3,2);
      \draw[example, message]   (m2.center) to[bend right] (1.5,1);
      \draw[example, message]   (m2.center) to[bend right] (2,0);

      \draw[alert, thick] (3.5,3.25) -- (5.5,-.25);
      \draw[alert, thick] (5.5,3.25) -- (3.5,-.25);
    \end{tikzpicture}
  }
  
  \on[bottom=-2mm, text]{
    \begin{citing}
    \item[BJ87]  K. P. Birman, T. A. Joseph. \emph{Reliable Communication in the Presence of Failures}. TOCS, 1987. \vspace{-1mm}
    \item[DSU04] X. Défago, A. Schiper, P. Urbán. \emph{Total Order Broadcast and Multicast Algorithms: Taxonomy and Survey}. ACM CSUR, 2004.
    \end{citing}
  }  

\end{frame}

\endgroup
\endinput

% SPDX-License-Identifier: CC-BY-SA-4.0
% Author: Matthieu Perrin
% Part: 
% Section: 
% Sub-section: 
% Frame: 

\begingroup

\SetKwData{Clock}{clock}
\SetKwData{Delivered}{delivered}
\SetKwData{Tick}{tick}
\SetKwData{Now}{now}
\SetKwData{Update}{update}

\begin{frame}{Implémentation de FIFO Broadcast}

  \ob<-2>[top]{
    \begin{algorithm}[H]
      \lLVariables{}{$\Delivered_i \leftarrow [0, ..., 0]$}
      \Method{$\textsc{fifo}.\Broadcast(m)$}{
        \textsc{sta}.\SBroadcast $\textsc{msg}(m, \Delivered_i[i])$;
      }
      \When{\textsc{sta}.\Deliver $\textsc{msg}(m, \mathit{sn}_j)$ \From $p_j$}{
        \Wait{$\Delivered_i[j] = \mathit{sn}_j$};\\
        \textsc{fifo}.\Deliver $m$ \From $p_j$;\\
        $\Delivered_i[j] \leftarrow \mathit{sn}_j+1$;
      }
    \end{algorithm}
  }

  \on<3>[top]{
    \begin{algorithm}[H]
      \lLVariables{}{\Example{$\Clock_i$;}\tcp*[f]{$\Delivered_i[1..n]$}}
      \Method{$\textsc{fifo}.\Broadcast(m)$}{
        \textsc{sta}.\SBroadcast $\textsc{msg}(m,~\Example{\Clock_i.\Now(i)})$;\tcp*[r]{$\Delivered_i[i]$}
      }
      \When{\textsc{sta}.\Deliver $\textsc{msg}(m, \mathit{date}_j)$ \From $p_j$}{
        \Wait{\Example{$\mathit{date}_j \ge \Clock_i.\Now(j)$}};\tcp*[r]{$\Delivered_i[j]$}
        \textsc{fifo}.\Deliver $m$ \From $p_j$;\\
        \Example{$\Clock_i.\Update(j)$};\tcp*[r]{$\Delivered_i[j] \leftarrow \mathit{sn}+1$}
      }
    \end{algorithm}
  }

  \obExampleBlock<1>[anchor=north]{Exemple d'exécution}{}
  \ob<1>[y=-25mm]{
    \begin{tikzpicture}[y=12mm]
      \draw[process] (0,1) node[left]{$p_1$}                                            to (10,1);
      \draw[process] (0,0) node[left]{$p_2$} node[replica below]{$\Delivered_2[1] = 0$} to (10,0);

      \draw[alert]   (1,1) node (m1) {} node[above] {$m$} ;
      \draw[alert, message]     (m1.center) to[bend left]   (2,1) ;
      \draw[alert, message]     (m1.center) to node[swap]{$sn=0$} (4.5,0) node[replica below]{$1$};

      \draw[example] (3,1) node (m2) {} node[above] {$m'$} ;
      \draw[example, message]   (m2.center) to[bend left]   (4,1) ;
      \draw[example, message]   (m2.center) to node{$sn=1$} (3.5,.05) to[bend left] (5.5,0) node[replica below]{$2$} ;
    \end{tikzpicture}
  }

  \onBlock<2->[anchor=north]{Remarques}{
    \begin{itemize}
    \item FIFO-Ordering : pour tout $j$, $\Delivered_i[j]$ = prochain $sn$ attendu de $p_j$.
    \item On peut renforcer les propriétés en remplaçant \textsc{sta} par \textsc{rb} ou \textsc{urb}
    \item En pratique, \Wait est implémenté par des buffers
    \item<3> Pour tout processus $p_j$, $\Delivered_i[j]$ implémente une horloge logique
      \begin{itemize}
      \item Le temps logique est l'ordre de processus
      \item Les dates sont des entiers (horloge scalaire)
      \end{itemize}
    \end{itemize}
  }
  
\end{frame}

\endgroup
\endinput

% SPDX-License-Identifier: CC-BY-SA-4.0
% Author: Matthieu Perrin
% Part: Communication
% Section: Cohérence faible
% Frame: Eventual Consistency

\begingroup

\SetKwFunction{Read}{read}
\SetKwFunction{Write}{write}
\SetKwData{Vals}{vals}
\SetKwData{Time}{time}
\SetKwFunction{V}{v}
\SetKwFunction{TS}{ts}

\begin{frame}{Mémoire causalement convergente}

  \on<2->[y=18mm, x=35mm]{
    \begin{tikzpicture}[anchor=center]
      \node[nosep, text width=40mm, text height=25mm] (execution)  at (0,0) {};
      \node[nosep, text width=25mm, text height=25mm] (concurrent) at (0,0) {};
      \node[nosep, text width=40mm, text height=08mm] (local)      at (0,0) {};
      \node[nosep, text width=08mm, text height=08mm] (present)    at (0,0) {};
      
      \fill[structure!15] (concurrent.north west) rectangle (concurrent.south east);
      \fill[alert!15]     (execution.north west) -- (concurrent.north west) -- (present.north west) -- (present.south west) -- (concurrent.south west) -- (execution.south west);
      \fill[example!15]   (execution.north east) -- (concurrent.north east) -- (present.north east) -- (present.south east) -- (concurrent.south east) -- (execution.south east);
      \fill[structure!30] (present.north west) rectangle (present.south east);
      \fill[alert!30]     (local.north west) rectangle (present.south west);
      \fill[example!30]   (local.north east) rectangle (present.south east);
      
      \draw (concurrent.north west) -- (present.north west) -- (present.south west) -- (concurrent.south west);
      \draw (concurrent.north east) -- (present.north east) -- (present.south east) -- (concurrent.south east);
      \draw (local.north west) -- (local.north east);
      \draw (local.south west) -- (local.south east);
      
      \scriptsize
      \node[alert, below right, align=left]   at (execution.north west) {Passé \\causal};
      \node[alert, below right]               at (local.north west)     {Passé local};
      \node[structure, below]                 at (concurrent.north)     {Présent concurrent};
      \node[structure, below]                 at (present.north)        {Présent};
      \node[example, below left, align=right] at (execution.north east) {Futur \\causal};
      \node[example, above left]              at (local.south east)     {Futur local};

      \tiny
      \draw (-10mm,0)        node[alert]     (plo1) {$\bullet$} node[left]       {};
      \draw (-5mm,0)         node[alert]     (plo2) {$\bullet$} node[below left] {};
      \draw (-10mm, 6mm)     node[alert]     (pca1) {$\bullet$} node[left]       {};
      \draw (-15mm,-8mm)     node[alert]     (pca2) {$\bullet$} node[left]       {};
      \draw (present.center) node[structure] (pres) {$\bullet$} node[below]      {};
      \draw (  0mm, 6mm)     node[structure] (con1) {$\bullet$} node[above]      {};
      \draw ( -3mm,-8mm)     node[structure] (con2) {$\bullet$} node[below]      {};
      \draw (  3mm,-8mm)     node[structure] (con3) {$\bullet$} node[below]      {};
      \draw ( 8mm, 0mm)     node[example]   (floc) {$\bullet$} node[right]      {};
      \draw ( 12mm, 6mm)     node[example]   (fca1) {$\bullet$} node[right]      {};
      \draw ( 15mm,-8mm)     node[example]   (fca2) {$\bullet$} node[right]      {};

      \path[->] (pca1) edge (con1);
      \path[->] (con1) edge (fca1);
      
      \path[->] (plo1) edge (plo2);
      \path[->] (plo2) edge (pres);
      \path[->] (pres) edge (floc);

      \path[->] (pca2) edge (con2);
      \path[->] (con2) edge (con3);
      \path[->] (con3) edge (fca2);

      \path[->] (pca1) edge (plo2);
      \path[->] (pca2) edge (plo1);
      
      \path[->] (floc) edge (fca1);
      \path[->] (pres) edge (fca2);
    \end{tikzpicture}
  }
  
  \on[top=-1mm]{
    \begin{algorithm}[H]
      \LVariables{}{
        $\Vals_i \leftarrow \textsc{map}\{ x \rightarrow \langle \bot, \langle 0, 0 \rangle\rangle \} $;\\
        $\Time_i  \leftarrow 0$;
      }
      \lMethod{$\Read(x)$}{\Return $\Vals_i[x].\V$;}
      \Method{$\Write(x, v)$}{
        \Alert{\textsc{rb-cb}}.\SBroadcast $\textsc{w}(x, v, \Time_i+1)$;
      }
      \When{\Alert{\textsc{rb-cb}}.\Deliver $\textsc{w}(x, v, t)$ \From $p_j$}{
        $\Time_i \leftarrow \max(\Time_i, t)$;\\
        \lIf{$\Vals_i[x].\TS <_{lex} \langle t, j \rangle$}{
          $\Vals_i[x] \leftarrow \langle v, \langle t, j \rangle \rangle$;
        }
      }
    \end{algorithm}
  }

  \obExampleBlock<1>[y=-2mm, anchor=north]{Exemples d'exécution -- Garantie de causalité}{
    \centering
    \begin{tikzpicture}[y=10mm]
      \draw[process] (0,2) node[left]{$p_1$} to (10,2);
      \draw[process] (0,1) node[left]{$p_2$} to (10,1);
      \draw[process] (0,0) node[left]{$p_3$} to (10,0);

      \node[alert, operation]         (wa) at (1.25,1) {$x.\Write(a)$};
      \path[alert, message]           (wa.west) edge (.6,2);
      \path[alert, message]           (wa.west) edge[bend left]  (wa.east);
      \path[alert, message]           (wa.west) edge[bend below] (3.5,0);
      
      \node[structure, operation]     (r1) at (2,2) {$x.\Read() \rightarrow a$};
      \node[example, operation]       (wb) at (4.5,2) {$y.\Write(b)$};
      \path[example, message]         (wb.west) edge[bend left] (wb.east);
      \path[example, message]         (wb.west) edge            (5,1);
      \path[example, message]         (wb.west) edge            (4,0);
      
      \node[structure, operation]     (r2) at (5.5,0) {$y.\Read() \rightarrow b$};
      \node[structure, operation]     (r3) at (8.1,0) {$x.\Read() \rightarrow a$};
    \end{tikzpicture}
  }
  
  \onBlock<2>[y=-2mm, anchor=north]{Définition -- Convergence causale}{
    Une exécution $H$ est \structure{causalement convergente} s'il existe :
    \begin{itemize}
    \item Un \structure{ordre partiel $\rightarrow$} sur les opérations qui contient l'ordre de processus
    \item Un \structure{ordre total $\le$} sur les opérations qui contient l'ordre $\rightarrow$
    \end{itemize}
    tels que la valeur retournée par chaque opération de requête $o \in \mathit{Query}$
    est la même que dans l'exécution séquentielle de toutes les mises à jour
    qui précèdent $o$ selon $\rightarrow$, dans l'ordre $\le$.
  }
  
  \obExampleBlock<3>[y=-2mm, anchor=north]{Exemples -- Convergence causale}{
    \centering
    \begin{tikzpicture}[y=10mm]
      \draw[process] (0,2) node[left]{$p_1$} to (10,2);
      \draw[process] (0,1) node[left]{$p_2$} to (10,1);
      \draw[process] (0,0) node[left]{$p_3$} to (10,0);

      \node[example,   operation, outer sep=0pt]     (wa) at (1.25,1) {$x.\Write(a)$};
      \node[example,   operation, outer sep=0pt]     (wc) at (1.25,0) {$y.\Write(c)$};
      \node[structure, operation, outer sep=0pt]     (r0) at (4,1) {$y.\Read() \rightarrow c$};
      \node[structure, operation, outer sep=0pt]     (r1) at (1.8,2) {$x.\Read() \rightarrow a$};
      \node[example,   operation, outer sep=0pt]     (wb) at (4.7,2) {$y.\Write(b)$};
      \node[structure, operation, outer sep=0pt]     (r2) at (5.3,0) {$y.\Read() \rightarrow b$};
      \node[structure, operation, outer sep=0pt]     (r3) at (8.3,0) {$x.\Read() \rightarrow a$};
      
      \node[alert, font=\tiny, anchor=east] at (wa.east) {1}; 
      \node[alert, font=\tiny, anchor=east] at (wc.east) {2}; 
      \node[alert, font=\tiny, anchor=east] at (r0.east) {3}; 
      \node[alert, font=\tiny, anchor=east] at (r1.east) {4}; 
      \node[alert, font=\tiny, anchor=east] at (wb.east) {5}; 
      \node[alert, font=\tiny, anchor=east] at (r2.east) {6}; 
      \node[alert, font=\tiny, anchor=east] at (r3.east) {7}; 
      
      \path[->, alert] (wa) edge (r1);
      \path[->, alert] (r1) edge (wb);
      \path[->, alert] (wb) edge[bend left] (r2);
      \path[->, alert] (wc) edge (r2);
      \path[->, alert] (r2) edge (r3);
      \path[->, alert] (wc) edge (r0);
      \path[->, alert] (wa) edge (r0);
    \end{tikzpicture}
  }
  
  \obBlock<4>[y=-2mm, anchor=north]{Théorème -- L'algorithme implémente la convergence causale}{
    \begin{itemize}
    \item L'\structure{ordre partiel $\rightarrow$} est l'ordre happens before $\hb$
    \item L'\structure{ordre total $\le$} est défini par les estampilles de Lamport
    \item On a bien $\rightarrow \subseteq \le$, d'après les propriétés des horloges de Lamport
    \item La valeur retournée par les lectures est bien la dernière écrite
    \end{itemize}
  }
  
  \obAlertBlock<5>[y=-2mm, anchor=north]{L'algorithme n'est pas séquentiellement cohérent}{
    \centering
    \begin{tikzpicture}[y=10mm]
      \draw[process] (0,1) node[left]{$p_1$} to (10,1);
      \draw[process] (0,0) node[left]{$p_2$} to (10,0);

      \node[alert, operation]         (wa) at (3,1) {$x.\Write(a)$};
      \path[alert, message]           (wa.west) edge[bend left]  (wa.east);
      \path[alert, message]           (wa.west) edge[bend below] (8,0);
      \node[structure, operation]     (r1) at (6,1) {$y.\Read() \rightarrow \bot$};
      
      \node[example, operation]       (wb) at (3,0) {$y.\Write(b)$};
      \path[example, message]         (wb.west) edge[bend above] (8,1);
      \path[example, message]         (wb.west) edge[bend right] (wb.east);
      \node[structure, operation]     (r2) at (6,0) {$x.\Read() \rightarrow \bot$};
    \end{tikzpicture}
  }

  \only<2>{
    \footnoteref{M. Perrin, A. Mostéfaoui, C. Jard. \textit{Causal consistency: beyond memory}. PPoPP (2016)}
  }
  
\end{frame}

\endgroup
\endinput

 
 
\part{Synchronisation}
 
 
\section{Nécessité des quorums}
 
\subsection{Des messages à la mémoire}
% SPDX-License-Identifier: CC-BY-SA-4.0
% Author: Matthieu Perrin
% Part: 
% Section: 
% Sub-section: 
% Frame: 

\begingroup

\SetKwFunction{Acheter}{acheter}

\begin{frame}{Linéarisabilité}

  Soit $M = \langle C, R, Q, q_0, \tau, \rho \rangle$ une machine à états. 
  
  \begin{block}{Définition -- Linéarisabilité}
    Une exécution $H$ est \structure{linéarisable} pour $M$ s'il existe une exécution séquentielle\\[-2mm]
    $$
    H_S = q_0 \xrightarrow{c_1 \rightarrow r_1} q_1 \xrightarrow{c_2 \rightarrow r_2} q_2 \xrightarrow{\dots} \dots
    $$

    \vspace{-2mm} telle que :
    \begin{description}
    \item[Sémantique :] pour tout $i\ge 1$, $\tau(q_{i-1}, c_{i}) = q_{i}$ et $\rho(q_{i-1}, c_{i}) = r_{i}$.
    \item[Opérations :] $H$ et $H_S$ contiennent les mêmes opérations $o_i = c_i \rightarrow r_i$ ;
    \item[Temps réel :] si l'opération $o_i$ se termine avant que $o_j$ ne commence dans $H$,\\
      alors $o_i$ précède $o_j$ dans $H_S$ ;
    \end{description}
    
    \centering
    \begin{tikzpicture}[x=10mm,y=7mm]
      \draw[densely dotted] (0,3) node[left]{$Q$}   -- (9,3);
      \draw[process]        (0,2) node[left]{$p_1$} to (9,2);
      \draw[process]        (0,1) node[left]{$p_2$} to (9,1);
      \draw[process]        (0,0) node[left]{$p_3$} to (9,0);

      \node[alert,     operation] (o1) at (2,1) {$\Acheter(a)\rightarrow\xmark$};
      \node[example,   operation] (o2) at (4,0) {$\Acheter(a)\rightarrow\cmark$};
      \node[structure, operation] (o3) at (6,2) {$\Acheter(a)\rightarrow\xmark$};

      \scriptsize
      \node[below right] at (0,3) {$\{a, b\}$};
      \draw[alert,     lin point] (o1.east) -- +(0,2) node[below right] {$\{b\}$};
      \draw[example,   lin point] (o2.west) -- +(0,3) node[below right] {$\{b\}$};
      \draw[structure, lin point] (o3.west) -- +(0,1) node[below right] {$\{b\}$};
    \end{tikzpicture}

  \end{block}

  \on[bottom=-2mm, text]{
    \begin{citing}
    \item[PMJ16] M. Herlihy, J. Wing. \emph{Linearizability: A correctness condition for concurrent objects}. TOPLAS. 1990
    \end{citing}
  }  

\end{frame}

\endgroup
\endinput

% SPDX-License-Identifier: CC-BY-SA-4.0
% Author: Matthieu Perrin
% Part: 
% Section: 
% Sub-section: 
% Frame: 

\begingroup

\SetKwFunction{Read}{read}
\SetKwFunction{Write}{write}
\SetKwData{X}{x}
\SetKwData{Y}{y}

\begin{frame}{Motifs en messages et en mémoire}

  \on[top]{
    Deux processus $p_1$ et $p_2$ partagent de l'information de manière concurrente
  }

  \onBlock[top=5mm]{Motifs en passage de messages}{}

  \on[y=5mm, x=-38mm]{
    \begin{tikzpicture}[x=8mm, y=7mm]
      \draw[process] (0,1) node[left]{$p_1$} to (4,1);
      \draw[process] (0,0) node[left]{$p_2$} to (4,0);

      \draw[message, example]   (1,1) node[above left]{$m_1$} to[bend left=50]  (2,1);
      \draw[message, example]   (1,1)                         --                (3,0);
      \draw[message, structure] (1,0) node[below left]{$m_2$} to[bend right=50] (2,0);
      \draw[message, structure] (1,0)                         --                (3,1);

      \draw (2,-1) node{MP1};
    \end{tikzpicture}
  }
  \on[y=5mm]{
    \begin{tikzpicture}[x=8mm, y=7mm]
      \draw[process] (0,1) node[left]{$p_1$} to (4,1);
      \draw[process] (0,0) node[left]{$p_2$} to (4,0);

      \draw[message, example]   (1,1) node[above left]{$m_1$} to[bend left=50]  (2,1);
      \draw[message, example]   (1,1)                         --                (2,0);
      \draw[message, structure] (1,0) node[below left]{$m_2$} to[bend right=50] (3,0);
      \draw[message, structure] (1,0)                         --                (3,1);

      \draw (2,-1) node{MP2};
    \end{tikzpicture}
  }
  \on[y=5mm, x=38mm]{
    \begin{tikzpicture}[x=8mm, y=7mm]
      \draw[process] (0,1) node[left]{$p_1$} to (4,1);
      \draw[process] (0,0) node[left]{$p_2$} to (4,0);

      \draw[message, example]   (1,1) node[above left]{$m_1$} to[bend left=50]  (3,1);
      \draw[message, example]   (1,1)                         --                (2,0);
      \draw[message, structure] (1,0) node[below left]{$m_2$} to[bend right=50] (3,0);
      \draw[message, structure] (1,0)                         --                (2,1);

      \draw (2,-1) node{MP3};
    \end{tikzpicture}
  }


  \onBlock<2>[y=-13mm]{Motifs en mémoire partagée}{}

  \on<2>[y=-29mm, x=-38mm]{
    \begin{tikzpicture}[x=8mm, y=7mm]
      \draw (2,-1) node{RW1};
      
      \draw[process] (0,1) node[left]{$p_1$} to (4,1);
      \draw[process] (0,0) node[left]{$p_2$} to (4,0);

      \scriptsize
      \node[example,   operation, inner sep=1pt] at (1  ,1) {$\X.\Write(1)$};
      \node[structure, operation, inner sep=1pt] at (1  ,0) {$\Y.\Write(1)$};
      \node[structure, operation, inner sep=1pt] at (2.8,1) {$\Y.\Read() \rightarrow 0$};
      \node[example,   operation, inner sep=1pt] at (2.8,0) {$\X.\Read() \rightarrow 0$};
      
      \draw[alert, thick] (0,1.25) -- (4,-.25);
      \draw[alert, thick] (4,1.25) -- (0,-.25);
    \end{tikzpicture}
  }
  
  \on<2>[y=-29mm]{
    \begin{tikzpicture}[x=8mm, y=7mm]
      \draw (2,-1) node{RW2};
      
      \draw[process] (0,1) node[left]{$p_1$} to (4,1);
      \draw[process] (0,0) node[left]{$p_2$} to (4,0);

      \scriptsize
      \node[example,   operation, inner sep=1pt] at (1  ,1) {$\X.\Write(1)$};
      \node[structure, operation, inner sep=1pt] at (1  ,0) {$\Y.\Write(1)$};
      \node[structure, operation, inner sep=1pt] at (2.8,1) {$\Y.\Read() \rightarrow 0$};
      \node[example,   operation, inner sep=1pt] at (2.8,0) {$\X.\Read() \rightarrow 1$};
    \end{tikzpicture}
  }
  
  \on<2>[y=-29mm, x=38mm]{
    \begin{tikzpicture}[x=8mm, y=7mm]
      \draw (2,-1) node{RW3};
      
      \draw[process] (0,1) node[left]{$p_1$} to (4,1);
      \draw[process] (0,0) node[left]{$p_2$} to (4,0);

      \scriptsize
      \node[example,   operation, inner sep=1pt] at (1  ,1) {$\X.\Write(1)$};
      \node[structure, operation, inner sep=1pt] at (1  ,0) {$\Y.\Write(1)$};
      \node[structure, operation, inner sep=1pt] at (2.8,1) {$\Y.\Read() \rightarrow 1$};
      \node[example,   operation, inner sep=1pt] at (2.8,0) {$\X.\Read() \rightarrow 1$};
    \end{tikzpicture}
  }

\end{frame}

\endgroup
\endinput

 
\subsection{Le théorème CAP}
% SPDX-License-Identifier: CC-BY-SA-4.0
% Author: Matthieu Perrin
% Part: 
% Section: 
% Sub-section: 
% Frame: 

\begingroup

\SetKwFunction{Read}{read}
\SetKwFunction{Write}{write}

\begin{frame}{Le théorème CAP}

  \begin{block}{Conjecture -- Brewer}
    Il est impossible d'implémenter une base de données distribuée vérifiant :
    \begin{description}[Partition tolerance :]
    \item[Consistency :] toutes les répliques voient les \alert{mêmes données}
    \item[Availability :] chaque requête \alert{finit par} recevoir une réponse
    \item[Partition tolerance :] le système fonctionne même si des \alert{partitions du réseau} ne peuvent plus communiquer
    \end{description}
  \end{block}

  \uncover<2->{
    \begin{block}{Théorème -- Attiya, Bar Noy, Dolev ; Gilbert, Lynch}
      Dans le modèle par passage de message asynchrone avec crash, \\
      il est impossible d'implémenter une mémoire lire/écrire :
      \begin{description}[Partition tolerance :]
      \item[Consistency :] interdisant le motif RW1 (\alert{linéarisabilité}, \alert{c. séquentielle})
      \item[Availability :] dont les méthodes \Read et \Write \alert{terminent}
      \item[Partition tolerance :] même si \alert{$t\ge \frac{n}{2}$} processus tombent en panne
      \end{description}
    \end{block}
  }
  
  \begin{citing}
    \uncover<2->{\item[ABD95] H. Attiya, A. Bar-Noy, D. Dolev. \emph{Sharing memory robustly in message-passing systems}. JACM, 1995}
    \uncover<1->{\item[B00] E. A. Brewer. \emph{Towards robust distributed systems (abstract)}. PODC, 2000}
    \uncover<2->{\item[GL02] S. Gilbert and N. Lynch. \emph{Brewer's conjecture and the feasibility of consistent, available, partition-tolerant web services}. ACM SIGACT News, 2002}
  \end{citing}

\end{frame}

\endgroup
\endinput

% SPDX-License-Identifier: CC-BY-SA-4.0
% Author: Matthieu Perrin
% Part: 
% Section: 
% Sub-section: 
% Frame: 

\begingroup

\SetKwFunction{Read}{read}
\SetKwFunction{Write}{write}
\SetKwData{X}{x}
\SetKwData{Y}{y}

\begin{frame}{Démonstration du théorème CAP}

  \on[width=1.01\textwidth, top]{
    Supposons (par contradiction) une telle mémoire dans $\mathcal{CAMP}_{n,t}\left[t\ge\frac{n}{2}\right]$.

    \begin{itemize}
    \item Il existe une partition $X, Y$ telle que
      \begin{itemize}
      \item $X \cap Y = \emptyset$, et $X \cup Y = \{p_1, ..., p_n\}$
      \item $|X| \le t$, et $|Y| \le t$
      \end{itemize}
    \end{itemize}
  }

  \onBlock<2->[y=19mm, left=.45\textwidth, anchor=north]{Scénario S1}{
    \begin{tikzpicture}[x=10mm, y=4mm, anchor=center]
      \draw[process] (0,3) node[left] {$p_1$} to (4,3) node (x1) {}; 
      \draw[process] (0,2) node[left] {$p_2$} to (4,2) node (x2) {}; 
      \draw[crashed] (0,1) node[left] {$p_3$} to (.5,1) node (y1) {}; 
      \draw[crashed] (0,0) node[left] {$p_4$} to (.5,0) node (y2) {}; 
      
      \path (x1.north) edge[brace] node {$X$} (x2.south);
      \path (y1.north) edge[brace] node {$Y$} (y2.south);
      
      \scriptsize
      \node[example,   operation] at (1.0,3) {$\X.\Write(1)$};
      \node[alert,     operation] at (2.9,3) {$\Y.\Read() \rightarrow 0$};
    \end{tikzpicture}
  }

  \onBlock<2->[y=19mm, right=.45\textwidth, anchor=north]{Scénario S2}{
    \begin{tikzpicture}[x=10mm, y=4mm, anchor=center]
      \draw[crashed] (0,3) node[left] {$p_1$} to (.5,3) node (x1) {} ; 
      \draw[crashed] (0,2) node[left] {$p_2$} to (.5,2) node (x2) {} ; 
      \draw[process] (0,1) node[left] {$p_3$} to (4,1)  node (y1) {} ; 
      \draw[process] (0,0) node[left] {$p_4$} to (4,0)  node (y2) {} ; 
      
      \path (x1.north) edge[brace] node {$X$} (x2.south);
      \path (y1.north) edge[brace] node {$Y$} (y2.south);

      \scriptsize
      \node[alert,     operation] at (1.0,0) {$\Y.\Write(1)$};
      \node[example,   operation] at (2.9,0) {$\X.\Read() \rightarrow 0$};
    \end{tikzpicture}
  }

  \on[y=-7mm, width=1.01\textwidth, anchor=north]{
    \begin{itemize}
    \item<2-> S1 et S2 ont moins de $t$ pannes
      \begin{itemize}
      \item toutes les opérations terminent
      \item les lectures retournent 0
      \end{itemize}
    \item<3-> S3 est indistingable de S1 pour $X$
      \begin{itemize}
      \item $\Y.\Read()$ retourne $0$
      \end{itemize}
    \item<3-> S3 est indistingable de S2 pour $Y$
      \begin{itemize}
      \item $\X.\Read()$ retourne $0$
      \end{itemize}
    \item<3-> RW1 est possible : contradiction ! 
    \end{itemize}
  }

  \onBlock<3->[y=-10mm, right=.45\textwidth, anchor=north]{Scénario S3}{
    \begin{tikzpicture}[x=10mm, y=4mm, anchor=center]
      \draw[process] (0,4) node[left] {$p_1$} to (4,4) node (x1) {}; 
      \draw[process] (0,3) node[left] {$p_2$} to (4,3) node (x2) {}; 
      \draw[process] (0,1) node[left] {$p_3$} to (4,1) node (y1) {}; 
      \draw[process] (0,0) node[left] {$p_4$} to (4,0) node (y2) {}; 

      \draw[dashed]   (0,2) -- (4,2) node[above left] {\tiny réseau lent};

      \path (x1.north) edge[brace] node {$X$} (x2.south);
      \path (y1.north) edge[brace] node {$Y$} (y2.south);
      
      \scriptsize
      \node[example,   operation] at (1.0,4) {$\X.\Write(1)$};
      \node[alert,     operation] at (2.9,4) {$\Y.\Read() \rightarrow 0$};
      \node[alert,     operation] at (1.0,0) {$\Y.\Write(1)$};
      \node[example,   operation] at (2.9,0) {$\X.\Read() \rightarrow 0$};
    \end{tikzpicture}
  }
  
\end{frame}

\endgroup
\endinput

% SPDX-License-Identifier: CC-BY-SA-4.0
% Author: Matthieu Perrin
% Part: 
% Section: 
% Sub-section: 
% Frame: 

\begingroup

\SetKwFunction{Read}{read}
\SetKwFunction{Write}{write}

\begin{frame}{Surmonter CAP : la notion de quorum}

  \begin{block}{Définitions}
    Soit $\Pi = \{p_1, ..., p_n\}$ l'ensemble des processus.
    \begin{itemize}
    \item un \structure{quorum} est un sous-ensemble \alert{$Q \subseteq \Pi$}
    \item un \structure{système de quorums} est un ensemble $\mathcal{Q} \subseteq \mathcal{P}(\Pi)$ tel que :
      $$\alert{\forall Q_1, Q_2 \in \mathcal{Q}, Q_1 \cap Q_2 \neq \emptyset}$$ 
    \item le \structure{système de quorums majoritaires} est \alert{$\mathcal{Q}_{\mathrm{maj}} = \left\{Q \subseteq \Pi \,\middle|\, |Q| > \frac{n}{2}\right\}$}
    \end{itemize}
  \end{block}

  \begin{center}
    \begin{tikzpicture}[x=10mm, y=5mm, anchor=center]
      \draw[process, example  ] (0,4) node[left] {$p_1$} to (6,4) node (x1) {}; 
      \draw[process, example  ] (0,3) node[left] {$p_2$} to (6,3) node (x2) {}; 
      \draw[process, alert    ] (0,2) node[left] {$p_3$} to (6,2) node (z1) {}; 
      \draw[process, structure] (0,1) node[left] {$p_4$} to (6,1) node (y1) {}; 
      \draw[process, structure] (0,0) node[left] {$p_5$} to (6,0) node (y2) {}; 

      \path[example  ] (x1.north) edge[brace] node {$Q_x$} (z1.south);
      \path[structure] (z1.north) edge[brace] node {$Q_y$} (y2.south);
      
      \node[example,   operation] (xw)  at (1.5,4) {$x.\Write(1)$};
      \node[example,   event]    (xw3) at (1.5,3) {};
      \node[example,   event]    (xw2) at (1.5,2) {};
      \node[structure, event]    (xr2) at (4.5,2) {};
      \node[structure, event]    (xr1) at (4.5,1) {};
      \node[structure, operation] (xr)  at (4.5,0) {$x.\Read() \rightarrow 0$};

      \path[example,   message] (xw.west)  edge (xw2.west);
      \path[example,   message] (xw.west)  edge (xw3.west);
      \path[example,   message] (xw3.east) edge (xw.east) ;
      \path[example,   message] (xw2.east) edge (xw.east) ;

      \path[structure, message] (xr.west)  edge (xr2.west);
      \path[structure, message] (xr.west)  edge (xr1.west);
      \path[structure, message] (xr1.east) edge (xr.east) ;
      \path[structure, message] (xr2.east) edge (xr.east) ;

      \path[alert, thick, ->, shorten <=2mm, shorten >=2mm]   (xw2) edge[bend left=10] (xr2) ;
    \end{tikzpicture}
  \end{center}

  \vspace{-3mm}
  \begin{block}{Remarque}
    \begin{itemize}
    \item Il existe un quorum majoritaire de \alert{corrects} sous l'hypothèse \alert{$t<\frac{n}{2}$}
    \end{itemize}
  \end{block}
  
\end{frame}

\endgroup
\endinput

 
\section{Implémentation des registres}
 
\subsection{Mutual Broadcast}
% SPDX-License-Identifier: CC-BY-SA-4.0
% Author: Matthieu Perrin
% Part: 
% Section: 
% Sub-section: 
% Frame: 

\begingroup

\begin{frame}{Mutual broadcast}

  \onBlock[top=-2mm]{Propriétés héritées de \textsc{sta}.\Broadcast}{
    \begin{description}[MB-Terminaison :]
    \item[MB-Validité :]    Si $p_i$ \textsc{mb}.\Deliver $m$ de $p_j$, $p_j$ a \textsc{mb}.\Broadcast $m$
    \item[MB-Intégrité :]   $p_i$ \textsc{mb}.\Deliver $m$ au plus une fois
    \item[MB-Terminaison :] \textsc{mb}.\Broadcast par correct $\Rightarrow$ \textsc{mb}.\Deliver par corrects.
    \end{description}
  }

  \onAlertBlock[y=-2mm]{Nouvelle propriété d'ordre}{
    \begin{description}[MB-Ordering :]
    \item[\alert{MB-Ordering :}] Si $p_i$ et $p_j$ \textsc{mb}.\Broadcast $m_i$ et $m_j$, il est impossible que
      \begin{itemize}
      \item $p_i$ \textsc{mb}.\Deliver $m_i$ avant $m_j$, et en même temps
      \item $p_j$ \textsc{mb}.\Deliver $m_j$ avant $m_i$
      \end{itemize}
    \end{description}
  }

  \on[y=-27mm, x=-38mm]{
    \begin{tikzpicture}[x=7mm, y=7mm]
      \draw[process] (0,1) node[left]{$p_1$} to (4,1);
      \draw[process] (0,0) node[left]{$p_2$} to (4,0);

      \draw[message, example]   (1,1) node[above left]{$m_1$} to[bend left=50]  (2,1);
      \draw[message, example]   (1,1)                         --                (3,0);
      \draw[message, structure] (1,0) node[below left]{$m_2$} to[bend right=50] (2,0);
      \draw[message, structure] (1,0)                         --                (3,1);

      \draw[alert, thick] (0,1.25) to (4,-.25);
      \draw[alert, thick] (4,1.25) to (0,-.25);

      \draw[alert] (2,-1) node{$\lnot$ MP1};
    \end{tikzpicture}
  }
  
  \on[y=-27mm]{
    \begin{tikzpicture}[x=7mm, y=7mm]
      \draw[process] (0,1) node[left]{$p_1$} to (4,1);
      \draw[process] (0,0) node[left]{$p_2$} to (4,0);

      \draw[message, example]   (1,1) node[above left]{$m_1$} to[bend left=50]  (2,1);
      \draw[message, example]   (1,1)                         --                (2,0);
      \draw[message, structure] (1,0) node[below left]{$m_2$} to[bend right=50] (3,0);
      \draw[message, structure] (1,0)                         --                (3,1);

      \draw (2,-1) node{MP2};
    \end{tikzpicture}
  }

  \on[y=-27mm, x=38mm]{
    \begin{tikzpicture}[x=7mm, y=7mm]
      \draw[process] (0,1) node[left]{$p_1$} to (4,1);
      \draw[process] (0,0) node[left]{$p_2$} to (4,0);

      \draw[message, example]   (1,1) node[above left]{$m_1$} to[bend left=50]  (3,1);
      \draw[message, example]   (1,1)                         --                (2,0);
      \draw[message, structure] (1,0) node[below left]{$m_2$} to[bend right=50] (3,0);
      \draw[message, structure] (1,0)                         --                (2,1);

      \draw (2,-1) node{MP3};
    \end{tikzpicture}
  }

  \on[bottom=-2mm, text]{
    \begin{citing}
    \item[PMJ16] M. Déprés, A. Mostéfaoui, M. Perrin, M. Raynal. \emph{Send/Receive Patterns Versus Read/Write Patterns in Crash-Prone Asynchronous Distributed Systems} DISC. 2023
    \end{citing}
  }  
  
\end{frame}

\endgroup
\endinput

% SPDX-License-Identifier: CC-BY-SA-4.0
% Author: Matthieu Perrin
% Part: 
% Section: 
% Sub-section: 
% Frame: 

\begingroup

\begin{frame}{Exemple}

  \on{
    \begin{tikzpicture}[y=20mm]

      \uncoverb<2,4>{
        \draw (4.5, -1) node{Motif de messages MP2};
      }
      \uncoverb<3>{
        \draw (4.5, -1) node{Motif de messages MP3};
      }
      
      \draw[process, fade ob=<4>] (0,2) node[left]{$p_1$} to (10,2);
      \draw[process, fade ob=<3>] (0,1) node[left]{$p_2$} to (10,1);
      \draw[process, fade ob=<2>] (0,0) node[left]{$p_3$} to (10,0);
      
      \draw[alert, fade ob=<4>] (1,2) node (m1) {} node[below left]{$m_1$};
      \path[message, alert,     fade ob=<{3,4}>, thick] (m1.center) edge             (6,1); 
      \path[message, alert,     fade ob=<{2,4}>, thick] (m1.center) edge             (5,0); 
      \path[message, alert,     fade ob=<{4}>  , thick] (m1.center) edge[bend left]  (7,2); 

      \draw[example, fade ob=<3>] (1,1) node (m2) {} node[below left]{$m_2$};
      \path[message, example,   fade ob=<{3,4}>, thick] (m2.center) edge             (3,2); 
      \path[message, example,   fade ob=<{2,3}>, thick] (m2.center) edge             (7,0); 
      \path[message, example,   fade ob=<{3}>  , thick] (m2.center) edge[bend left]  (4,1); 

      \draw[structure, fade ob=<2>] (1,0) node (m3) {} node[above left]{$m_3$};
      \path[message, structure, fade ob=<{2,4}>, thick] (m3.center) edge             (5,2); 
      \path[message, structure, fade ob=<{2,3}>, thick] (m3.center) edge             (8,1); 
      \path[message, structure, fade ob=<{2}>  , thick] (m3.center) edge[bend right] (9,0); 
    \end{tikzpicture}
  }

\end{frame}

\endgroup
\endinput

% SPDX-License-Identifier: CC-BY-SA-4.0
% Author: Matthieu Perrin
% Part: 
% Section: 
% Sub-section: 
% Frame: 

\begingroup

\SetKwData{Clock}{clock}
\SetKwData{Delivered}{delivered}
\SetKwData{Tick}{tick}
\SetKwData{Now}{now}
\SetKwData{Update}{update}

\begin{frame}{Implémentation de FIFO Broadcast}

  \ob<-2>[top]{
    \begin{algorithm}[H]
      \lLVariables{}{$\Delivered_i \leftarrow [0, ..., 0]$}
      \Method{$\textsc{fifo}.\Broadcast(m)$}{
        \textsc{sta}.\SBroadcast $\textsc{msg}(m, \Delivered_i[i])$;
      }
      \When{\textsc{sta}.\Deliver $\textsc{msg}(m, \mathit{sn}_j)$ \From $p_j$}{
        \Wait{$\Delivered_i[j] = \mathit{sn}_j$};\\
        \textsc{fifo}.\Deliver $m$ \From $p_j$;\\
        $\Delivered_i[j] \leftarrow \mathit{sn}_j+1$;
      }
    \end{algorithm}
  }

  \on<3>[top]{
    \begin{algorithm}[H]
      \lLVariables{}{\Example{$\Clock_i$;}\tcp*[f]{$\Delivered_i[1..n]$}}
      \Method{$\textsc{fifo}.\Broadcast(m)$}{
        \textsc{sta}.\SBroadcast $\textsc{msg}(m,~\Example{\Clock_i.\Now(i)})$;\tcp*[r]{$\Delivered_i[i]$}
      }
      \When{\textsc{sta}.\Deliver $\textsc{msg}(m, \mathit{date}_j)$ \From $p_j$}{
        \Wait{\Example{$\mathit{date}_j \ge \Clock_i.\Now(j)$}};\tcp*[r]{$\Delivered_i[j]$}
        \textsc{fifo}.\Deliver $m$ \From $p_j$;\\
        \Example{$\Clock_i.\Update(j)$};\tcp*[r]{$\Delivered_i[j] \leftarrow \mathit{sn}+1$}
      }
    \end{algorithm}
  }

  \obExampleBlock<1>[anchor=north]{Exemple d'exécution}{}
  \ob<1>[y=-25mm]{
    \begin{tikzpicture}[y=12mm]
      \draw[process] (0,1) node[left]{$p_1$}                                            to (10,1);
      \draw[process] (0,0) node[left]{$p_2$} node[replica below]{$\Delivered_2[1] = 0$} to (10,0);

      \draw[alert]   (1,1) node (m1) {} node[above] {$m$} ;
      \draw[alert, message]     (m1.center) to[bend left]   (2,1) ;
      \draw[alert, message]     (m1.center) to node[swap]{$sn=0$} (4.5,0) node[replica below]{$1$};

      \draw[example] (3,1) node (m2) {} node[above] {$m'$} ;
      \draw[example, message]   (m2.center) to[bend left]   (4,1) ;
      \draw[example, message]   (m2.center) to node{$sn=1$} (3.5,.05) to[bend left] (5.5,0) node[replica below]{$2$} ;
    \end{tikzpicture}
  }

  \onBlock<2->[anchor=north]{Remarques}{
    \begin{itemize}
    \item FIFO-Ordering : pour tout $j$, $\Delivered_i[j]$ = prochain $sn$ attendu de $p_j$.
    \item On peut renforcer les propriétés en remplaçant \textsc{sta} par \textsc{rb} ou \textsc{urb}
    \item En pratique, \Wait est implémenté par des buffers
    \item<3> Pour tout processus $p_j$, $\Delivered_i[j]$ implémente une horloge logique
      \begin{itemize}
      \item Le temps logique est l'ordre de processus
      \item Les dates sont des entiers (horloge scalaire)
      \end{itemize}
    \end{itemize}
  }
  
\end{frame}

\endgroup
\endinput

 
\subsection{Simulation ABD}
% SPDX-License-Identifier: CC-BY-SA-4.0
% Author: Matthieu Perrin
% Part: 
% Section: 
% Sub-section: 
% Frame: 

\begingroup

\SetKwFunction{Read}{read}
\SetKwFunction{Write}{write}
\SetKwData{Value}{value}
\SetKwData{Time}{time}
\SetKwData{Writer}{writer}

\begin{frame}{Implémentation d'un registre linéarisable}

  \on[top=-3mm]{
    \begin{algorithm}[H]
      \lLVariables{}{
        $\Value_i  \leftarrow \bot$; $\Time_i \leftarrow 0$; $\Writer_i \leftarrow 0$;
      }
      \Method{$\Read()$}{
        \uncover<3->{\Alert<3-4,8>{\textsc{mb}.\SBroadcast $\textsc{synch}()$;}\tcp*[f]{Barrière \Read $\rightarrow$ \Write}}\\
        \Alert<-2>{\Let $v \leftarrow \Value_i$;}\\
        \uncover<8->{\Structure<8>{\textsc{mb}.\SBroadcast $\textsc{w}(v, \Time_i, \Writer_i)$;}\tcp*[f]{Barrière \Read $\rightarrow$ \Read}}\\
        \Alert<-2>{\Return $v$;}
      }
      \Method{$\Write(v)$}{
        \uncover<6->{\Structure<6>{\textsc{mb}.\SBroadcast $\textsc{synch}()$;}\tcp*[f]{Barrière \Write $\rightarrow$ \Write}}\\
        \Example<-6>{\textsc{mb}.\SBroadcast $\textsc{w}(v, \Time_i+1, i)$;}\tcp*[f]{Last-Writer-Wins register}
      }
      \When{\textsc{mb}.\Deliver $\textsc{w}(v, t, w)$ \From $p_j$}{
        \Example<-2,6>{\lIf{$\langle \Time_i, \Writer_i \rangle <_{lex} \langle t, w \rangle$}{
            $\Value_i  \leftarrow v$; $\Time_i \leftarrow t$; $\Writer_i \leftarrow w$;
        }}
      }
    \end{algorithm}
  }

  \ob<1,2>[y=-25mm]{
    \begin{tikzpicture}[y=7mm]
      \draw[process] (0,2) node[left]{$p_1$} to (10,2);
      \draw[process] (0,1) node[left]{$p_2$} to (10,1);
      \draw[process] (0,0) node[left]{$p_3$} to (10,0);
      
      \node[example, operation] (w1) at (2,0) {$\Write(1)$};
      \path[example, message]   (w1.west) edge (2.5,2);
      \path[example, message]   (w1.west) edge (1.5,1);
      \path[example, message]   (w1.west) edge[bend left] (w1.east);

      \node[example, operation] (w2) at (5,2) {$\Write(2)$};
      \path[example, message]   (w2.west) edge [bend right] (w2.east);
      \path<1>[example, message]   (w2.west) edge (4.5,1);
      \path<2>[example, message]   (w2.west) edge[bend below] (9.5,1);
      \path[example, message]   (w2.west) edge (5.5,0);

      \node[alert,   operation] (r)  at (8,1) {$\Read() \rightarrow \alt<1>{2}{1}$};
    \end{tikzpicture}
  }
  
  \ob<3,4>[y=-25mm]{
    \begin{tikzpicture}[y=7mm]
      \draw[process] (0,2) node[left]{$p_1$} to (10,2);
      \draw[process] (0,1) node[left]{$p_2$} to (10,1);
      \draw[process] (0,0) node[left]{$p_3$} to (10,0);
      
      \node[fade, operation]         (w1) at (2,0) {$\Write(1)$};
      \path[fade, message]           (w1.west) edge (2.5,2);
      \path[fade, message]           (w1.west) edge (1.5,1);
      \path[fade, message]           (w1.west) edge[bend left] (w1.east);
      
      \node[alert,   operation]      (r)  at (8,1) {$\Read() \rightarrow 1$};
      \path[alert,   message, thick] (r.west) edge (8,2);
      \path[alert,   message]        (r.west) edge[bend left] (r.east);
      \path[alert,   message]        (r.west) edge (8,0);
      
      \node[example, operation]      (w2) at (5,2) {$\Write(2)$};
      \path[example, message]        (w2.west) edge[bend right] (w2.east);
      \path<3>[example, message, thick] (w2.west) edge[bend below] (9.5,1);
      \path<4>[example, message, thick] (w2.west) edge[bend below] (r.center);
      \path[example, message]        (w2.west) edge (5.5,0);
      
      \node<3>[alert] at (6,0.5) {MP1};
    \end{tikzpicture}
  }

  \ob<5>[y=-25mm]{
    \begin{tikzpicture}[y=7mm]
      \draw[process] (0,2) node[left]{$p_1$} to (10,2);
      \draw[process] (0,1) node[left]{$p_2$} to (10,1);
      \draw[process] (0,0) node[left]{$p_3$} to (10,0);

      \node[example, operation]      (w2)      at             (5,2) {$\Write(2)$};
      \draw[example, message]        (w2.west) to[bend right] (w2.east);
      \draw[example, message]        (w2.west) to             (4.5,1) node[below]{\small $\langle 1,1 \rangle$};
      \draw[example, message]        (w2.west) to             (5.5,0);
      
      \node[example, operation]      (w1)      at             (2,0) {$\Write(1)$};
      \draw[example, message, thick] (w1.west) to[bend above] (w2.center);
      \draw[example, message]        (w1.west) to             (1.5,1) node[above]{\small $\langle 1,3 \rangle$};
      \draw[example, message]        (w1.west) to[bend left]  (w1.east);
      
      \node[alert,   operation]      (r)       at             (8,1) {$\Read() \rightarrow 1$};
      \draw[alert,   message]        (r.west)  to             (8,2);
      \draw[alert,   message]        (r.west)  to[bend left]  (r.east);
      \draw[alert,   message]        (r.west)  to             (8,0);
    \end{tikzpicture}
  }

  \ob<6>[y=-25mm]{
    \begin{tikzpicture}[y=7mm]
      \draw[process] (0,2) node[left]{$p_1$} to (10,2);
      \draw[process] (0,1) node[left]{$p_2$} to (10,1);
      \draw[process] (0,0) node[left]{$p_3$} to (10,0);

      \node[example, operation]        (w2)        at             (5,2)         {$\Write(2)$};
      \draw[structure, message]        (w2.west)   to[bend right] (w2.center);
      \draw[structure, message]        (w2.west)   to             (4.5,1);
      \draw[structure, message, thick] (w2.west)   to             (5.5,0);
      \draw[example, message]          (w2.center) to[bend right] (w2.east);
      \draw[example, message]          (w2.center) to             (5.5,1)       node[below]{\small $\langle 2,1 \rangle$};
      \draw[example, message]          (w2.center) to             (6.5,0);
      
      \node[example, operation]        (w1)        at             (2,0)         {$\Write(1)$};
      \draw[structure, message]        (w1.west)   to             (2.5,2);
      \draw[structure, message]        (w1.west)   to             (1.5,1);
      \draw[structure, message]        (w1.west)   to[bend left]  (w1.center);
      \draw[example, message, thick]   (w1.center) to[bend above] (4.5,2);
      \draw[example, message]          (w1.center) to             (2.5,1)       node[above]{\small $\langle 1,3 \rangle$};
      \draw[example, message]          (w1.center) to[bend left]  (w1.east);
      
      \node[alert,   operation]        (r)         at             (8,1)         {$\Read() \rightarrow 2$};
      \draw[alert,   message]          (r.west)    to             (8,2);
      \draw[alert,   message]          (r.west)    to[bend left]  (r.east);
      \draw[alert,   message]          (r.west)    to             (8,0);
    \end{tikzpicture}
  }

  \ob<7>[y=-25mm]{
    \begin{tikzpicture}[y=7mm]
      \draw[process] (0,2) node[left]{$p_1$} to (10,2);
      \draw[process] (0,1) node[left]{$p_2$} to (10,1);
      \draw[process] (0,0) node[left]{$p_3$} to (10,0);

      \node[fade, operation]                     (w1)        at               (2,0)         {$\Write(1)$};
      \draw[fade, message]                       (w1.west)   to               (2.0,2);
      \draw[fade, message]                       (w1.west)   to               (1.5,1);
      \draw[fade, message]                       (w1.west)   to[bend left]    (w1.center);
      \draw[fade, message]                       (w1.center) to               (2.8,2);
      \draw[fade, message]                       (w1.center) to               (2.3,1);
      \draw[fade, message]                       (w1.center) to[bend left]    (w1.east);
      
      \node[example, operation, text width=55mm] (w2)        at               (6,2)         {$\Write(2)$};
      \coordinate                                (w2middle)  at               ([xshift=5mm]w2.west);
      \draw[structure, message]                  (w2.west)   to[bend right]   (w2middle);
      \draw[structure, message]                  (w2.west)   to               (3.5,1);
      \draw[structure, message]                  (w2.west)   to               (3.5,0);
      \draw[example, message]                    (w2middle)  to[bend right=8] (w2.east);
      \draw[example, message]                    (w2middle)  to[bend below]   (9.5,1);
      \draw[example, message]                    (w2middle)  to               (4,0);
      
      \node[alert,   operation]                  (r2)        at               (5.5,0)       {$\Read() \rightarrow 2$};
      \draw[alert,   message]                    (r2.west)   to               (5,2);
      \draw[alert,   message]                    (r2.west)   to               (5,1);
      \draw[alert,   message]                    (r2.west)   to[bend left]    (r2.east);
      
      \node[alert,   operation]                  (r1)        at               (8,1)         {$\Read() \rightarrow 1$};
      \draw[alert,   message]                    (r1.west)   to               (8,2);
      \draw[alert,   message]                    (r1.west)   to[bend left]    (r1.east);
      \draw[alert,   message]                    (r1.west)   to               (8,0);
    \end{tikzpicture}
  }

  \on<8>[y=-25mm]{
    \begin{tikzpicture}[y=7mm]

      \node[fade, operation]                     (w1)        at               (2,0)         {$\Write(1)$};
      \draw[fade, message]                       (w1.west)   to               (2.0,2);
      \draw[fade, message]                       (w1.west)   to               (1.5,1);
      \draw[fade, message]                       (w1.west)   to[bend left]    (w1.center);
      \draw[fade, message]                       (w1.center) to               (2.8,2);
      \draw[fade, message]                       (w1.center) to               (2.3,1);
      \draw[fade, message]                       (w1.center) to[bend left]    (w1.east);
      
      \node[example, operation, text width=55mm] (w2)        at               (6,2)         {$\Write(2)$};
      \coordinate                                (w2middle)  at               ([xshift=5mm]w2.west);
      \draw[structure, message]                  (w2.west)   to[bend right]   (w2middle);
      \draw[structure, message]                  (w2.west)   to               (3.5,1);
      \draw[structure, message]                  (w2.west)   to               (3.5,0);
      \draw[example, message]                    (w2middle)  to[bend right=8] (w2.east);
      \draw[example, message]                    (w2middle)  to[bend below]   (9.5,1);
      \draw[example, message]                    (w2middle)  to               (4,0);
      
      \node[alert,     operation]                (r1)        at               (8,1)         {$\Read() \rightarrow 2$};
      \draw[alert,     message]                  (r1.west)   to               (8,2);
      \draw[alert,     message]                  (r1.west)   to[bend left]    (r1.center);
      \draw[alert,     message, thick]           (r1.west)   to               (8,0);
      \draw[structure, message]                  (r1.center) to               (8.5,2);
      \draw[structure, message]                  (r1.center) to[bend left]    (r1.east);
      \draw[structure, message]                  (r1.center) to               (8.5,0);

      \node[alert,     operation]                (r2)        at               (5.5,0)       {$\Read() \rightarrow 2$};
      \draw[alert,     message]                  (r2.west)   to               (5,2);
      \draw[alert,     message]                  (r2.west)   to               (5,1);
      \draw[alert,     message]                  (r2.west)   to[bend left]    (r2.center);
      \draw[structure, message]                  (r2.center) to               (6,2);
      \draw[structure, message, thick]           (r2.center) to[bend above]   (7.5,1);
      \draw[structure, message]                  (r2.center) to[bend left]    (r2.east);

      \draw[process] (0,2) node[left]{$p_1$} to (10,2);
      \draw[process] (0,1) node[left]{$p_2$} to (10,1);
      \draw[process] (0,0) node[left]{$p_3$} to (10,0);

    \end{tikzpicture}
  }

  \obBlock<9>[y=-23mm]{Théorème -- L'algorithme est linéarisable}{\small
    \begin{itemize}
    \item\vspace{-1mm} Posons $\structure{\mathit{ts}(o) = \langle t, w\rangle}$ de son message \textsc{w}, et $\structure{o_1\;\mathcal{R}\;o_2}$ si : 
    \begin{itemize}
    \item $o_1$ se termine avant le début de $o_2$,
    \item $\mathit{ts}(o_1) <_{lex} \mathit{ts}(o_2)$,
    \item ou $\mathit{ts}(o_1) = \mathit{ts}(o_2)$, $o_1$ est une écriture et $o_2$ est une lecture.
    \end{itemize}
    \item\vspace{-1mm} \alert{$\mathcal{R}$ est acyclique}, et peut être étendu en un ordre total de linéarisation
    \end{itemize}
  }

  \on[bottom=-1mm]{
    \begin{citing}
    \item[ABD95] H. Attiya, A. Bar-Noy, D. Dolev. \emph{Sharing memory robustly in message-passing systems}. JACM, 1995
    \end{citing}
  }
  
\end{frame}

\endgroup
\endinput

 
 
\part{Coordination}
 
 
\section{Machine à états répliquée}
 
\subsection{Diffusion totalement ordonnée}
% SPDX-License-Identifier: CC-BY-SA-4.0
% Author: Matthieu Perrin
% Part: 
% Section: 
% Sub-section: 
% Frame: 

\begingroup

\begin{frame}{Problème d'ordre FIFO}

  \on[y=-5mm]{
    \begin{tikzpicture}
      \node     [faded background picture=Seine,  text width=\paperwidth/2] (A) at (-\paperwidth/4,0) {};
      \node<1>  [faded background picture=Jardin, text width=\paperwidth/2] (B) at ( \paperwidth/4,0) {};
      \node<2,3>[faded background picture=Metro,  text width=\paperwidth/2] (B) at ( \paperwidth/4,0) {};
      \node<4>  [faded background picture=Rue,    text width=\paperwidth/2] (B) at ( \paperwidth/4,0) {};
      \node[anchor=south, nosep] at (A.south) {\includegraphics[height=22mm]{Alice}};
      \node[anchor=south, nosep] at (B.south) {\includegraphics[height=22mm]{Bob}};
    \end{tikzpicture}
  }

  \on[x=-\paperwidth/4, y=15mm]{
    \begin{chat}[color=structure]{Bob}
      \chatSend{Qu'est ce que tu es musclé !}
      \only<4->{\chatRecv[color=example]{Toi tu es très belle}}
      \only<4->{\chatRecv[color=example]{Hélas, ça n'a pas toujours été le cas}}
    \end{chat}
  }
 
  \on[x=\paperwidth/4, y=15mm]{
    \begin{chat}[color=example]{Alice}
      \chatRecv[color=structure]{Qu'est ce que tu es musclé !}
      \only<2->{\chatSend{Hélas, ça n'a pas toujours été le cas}}
      \only<3->{\chatSend{Toi tu es très belle}}
    \end{chat}
  }

  \ob<1>[x=-\paperwidth/4, y=-15mm]{
    \chatBubble[color=structure]{Qu'est ce que tu es musclé !}
  }
  
  \ob<2>[x=\paperwidth/4, y=-15mm]{
    \chatBubble[color=example]{Hélas, ça n'a pas toujours été le cas}
  }
 
  \ob<3>[x=\paperwidth/4, y=-15mm]{
    \chatBubble[color=example]{Toi tu es très belle}
  }
   
\end{frame}

\endgroup
\endinput

% SPDX-License-Identifier: CC-BY-SA-4.0
% Author: Matthieu Perrin
% Part: 
% Section: 
% Sub-section: 
% Frame: 

\begingroup

\begin{frame}{Total-Order Broadcast (Atomic Broadcast)}

  \onBlock[top=-5mm]{Propriétés héritées de \textsc{sta}.\Broadcast}{
    \vspace{-1mm}
    \begin{description}[Total-ordering :]
    \item[TO-Validité :]    Si $p_i$ \textsc{to}.\Deliver $m$ de $p_j$, $p_j$ a \textsc{to}.\Broadcast $m$
    \item[TO-Intégrité :]   $p_i$ \textsc{to}.\Deliver $m$ au plus une fois
    \item[TO-Terminaison :] \textsc{to}.\Broadcast par correct $\Rightarrow$ \textsc{to}.\Deliver par corrects.
    \end{description}
  }

  \onAlertBlock[y=8mm]{Nouvelle propriété d'ordre}{
    \vspace{-1mm}
    \begin{description}[Total-ordering :]
    \item[Total-ordering :] Si un processus \textsc{to}.\Deliver $m$ avant $m'$, \\aucun processus ne \textsc{to}.\Deliver $m'$ avant $m$. 
    \end{description}
  }

  \onBlock[bottom=7mm]{Remarques}{
    \vspace{-1mm}
    \begin{itemize}
    \item Total-Ordering $\Rightarrow$ Causal-Ordering $\Rightarrow$ FIFO-Ordering
    \item Total-Ordering $\Rightarrow$ Mutual-Ordering 
    \item \structure{Ajout de propriétés :} \textsc{rb-to}.\Broadcast, \textsc{urb-to}.\Broadcast
    \end{itemize}
  }

  \on[y=-10mm, x=10mm]{
    \begin{tikzpicture}[y=3mm]
      \draw[process]   (0,3) node[left]{$p_1$} to (6,3);
      \draw[process]   (0,2) node[left]{$p_2$} to (6,2);
      \draw[process]   (0,1) node[left]{$p_3$} to (6,1);
      \draw[process]   (0,0) node[left]{$p_4$} to (6,0);

      \draw[structure]          (1,3) node (m1) {} node[above] {$m$};
      \draw[structure, message] (m1.center) to[bend left]  (2,3);
      \draw[structure, message] (m1.center) to[bend left]  (1.5,2);
      \draw[structure, message] (m1.center) to[bend left]  (3,1);
      \draw[structure, message] (m1.center) to[bend left]  (2.5,0);

      \draw[example]            (1,0) node (m2) {} node[below] {$m'$};
      \draw[example, message]   (m2.center) to[bend right] (2.5,3);
      \draw[example, message]   (m2.center) to[bend right] (3,2);
      \draw[example, message]   (m2.center) to[bend right] (1.5,1);
      \draw[example, message]   (m2.center) to[bend right] (2,0);

      \draw[alert, thick] (3.5,3.25) -- (5.5,-.25);
      \draw[alert, thick] (5.5,3.25) -- (3.5,-.25);
    \end{tikzpicture}
  }
  
  \on[bottom=-2mm, text]{
    \begin{citing}
    \item[BJ87]  K. P. Birman, T. A. Joseph. \emph{Reliable Communication in the Presence of Failures}. TOCS, 1987. \vspace{-1mm}
    \item[DSU04] X. Défago, A. Schiper, P. Urbán. \emph{Total Order Broadcast and Multicast Algorithms: Taxonomy and Survey}. ACM CSUR, 2004.
    \end{citing}
  }  

\end{frame}

\endgroup
\endinput

% SPDX-License-Identifier: CC-BY-SA-4.0
% Author: Matthieu Perrin
% Part: 
% Section: 
% Sub-section: 
% Frame: 

\begingroup

\SetKwFunction{Apply}{apply}
\SetKwFunction{Deliver}{deliver}
\SetKwData{State}{state}
\SetKwData{Result}{result}
\SetKwData{Command}{command}

\begin{frame}{Implémentation de la machine à états répliquée}

  \on[top=-2mm, text]{
    Soit $\langle C, R, Q, q_0, \tau, \rho \rangle$ une machine à états\\
    \begin{algorithm}[H]
      \LVariables{}{
        $\State_i \leftarrow q_0$; $\Result_i \leftarrow \bot$;
      }
      \Method{$\Apply(c \in C)$}{
        \Alert{\textsc{rb-to}.\SBroadcast $\textsc{cmd}(c)$;}\\
        \Return $\Result_i$;
      }
      \When{\textsc{rb-to}.\Deliver $\textsc{cmd}(c)$ \From $p_j$}{
        \lIf{$j=i$}{$\Result_i \leftarrow \rho(\State_i, c)$;}
        $\State_i \leftarrow \tau(\State_i, c)$;
      }
    \end{algorithm}
  }

  \onBlock[y=-22mm]{Théorème -- L'algorithme est linéarisable}{
    \vspace{-2mm}
    \begin{itemize}
    \item L'ordre de linéarisation est l'ordre de délivrance des messages
    \item Respect du temps réel car Total-Ordering $\Rightarrow$ Mutual-Ordering
    \end{itemize}

    \centering
    \begin{tikzpicture}[y=7mm]
      \draw[process] (0,2) node[left]{$p_1$} to (10,2);
      \draw[process] (0,1) node[left]{$p_2$} to (10,1);
      \draw[process] (0,0) node[left]{$p_3$} to (10,0);

      \node[replica]           at (0,1) {$q_0$};
      \node[replica,alert]     at (2.5,1) {$q_1 = \tau(q_0, c_1)$};
      \node[replica,example]   at (6,2) {$q_2 = \tau(q_1, c_2)$};
      \node[replica,structure] at (8,0) {$q_3 = \tau(q_2, c_3)$};

      \node[alert, operation]                    (c1) at (2,2) {$c_1 \rightarrow \rho(q_0, c_1)$};
      \node[example, operation, text width=30mm] (c2) at (2.8,0) {$c_2 \rightarrow \rho(q_1, c_2)$};
      \node[structure, operation]                (c3) at (7,1) {$c_3 \rightarrow \rho(q_2, c_3)$};

      \coordinate (0top) at (0 ,2.5);
      \coordinate (0bot) at (0 ,-.5);

      \begin{scope}[background]
        \fill[structure!15, rounded corners] (0bot) rectangle (10,2.5);
        \fill[example!15,   rounded corners] (0top) -- (8,2 |- 0top)     -- (8,2)     -- (c3.east) -- (7,0)     -- (7,0 |- 0bot)     -- (0bot);
        \fill[alert!15,     rounded corners] (0top) -- (4,2 |- 0top)     -- (6,2)     -- (6.5,1)   -- (c2.east) -- (c2.east |- 0bot) -- (0bot);
        \fill[black!5,      rounded corners] (0top) -- (c1.east |- 0top) -- (c1.east) -- (2,1)     -- (1.5,0)   -- (1.5,0 |- 0bot)   -- (0bot);
      \end{scope}
      
      \draw[alert, message]     (c1.west) to[bend left]  (c1.east);
      \draw[alert, message]     (c1.west) to             (2,1);
      \draw[alert, message]     (c1.west) to             (1.5,0);
      \draw[example, message]   (c2.west) to[bend above] (6,2);
      \draw[example, message]   (c2.west) to[bend above] (6.5,1);
      \draw[example, message]   (c2.west) to[bend right] (c2.east);
      \draw[structure, message] (c3.west) to             (8,2);
      \draw[structure, message] (c3.west) to[bend right] (c3.east);
      \draw[structure, message] (c3.west) to             (7,0);
    \end{tikzpicture}
  }

\end{frame}

\endgroup
\endinput

% SPDX-License-Identifier: CC-BY-SA-4.0
% Author: Matthieu Perrin
% Part: 
% Section: 
% Sub-section: 
% Frame: 

\begingroup

\tikzset{
  box/.style={draw, rounded corners, text width=20mm, minimum height=10mm, align=center, draw highlighted, fill highlighted, text highlighted},
}

\begin{frame}{Réductions à explorer}
  
  \begin{tikzpicture}[x=28mm, y=17mm, font=\scriptsize]
    \node[alert,     box] (leader) at (0  , 2) {Élection    \\ de leader};
    \node[structure, box] (cons)   at (1  , 2) {Consensus};
    \node[structure, box] (tob)    at (2  , 2) {Total-order \\ Broadcast};
    \node[example,   box] (rsm)    at (3  , 2) {Replicated  \\ State Machine};
    \node[example,   box] (mb)     at (1  , 1) {Mutual      \\ Broadcast};
    \node[alert,     box] (quorum) at (0.5, 0) {Quorums     \\ $t < \frac{n}{2}$};
    \node[example,   box] (rb)     at (1.5, 0) {Reliable    \\ Broadcast};

    \path[-latex] (leader) edge            (cons);
    \path[-latex] (cons)   edge[bend left] (tob);
    \path[-latex] (tob)    edge[bend left] (cons);
    \path[-latex] (tob)    edge[bend left] (rsm);
    \path[-latex] (rsm)    edge[bend left] (tob);
    \path[-latex] (quorum) edge            (mb);
    \path[-latex] (mb)     edge            (cons);
    \path[-latex] (rb)     edge            (mb);
    \path[-latex] (rb)     edge            (tob);
  \end{tikzpicture}

  \vspace{3mm}
  \begin{block}{Comment implémenter Total-order Broadcast ?}
    \begin{itemize}
    \item \structure{\textsc{to}.\Broadcast} est équivalent au \structure{consensus} en calculabilité
    \item Les deux ont besoin d'hypothèses sur le système $\mathcal{CAMP}_{n,t}$
      \begin{itemize}
      \item Des quorums pour empêcher les partitions \alert{$t < \frac{n}{2}$}
      \item Des périodes assez stables pour implémenter un \alert{leader}
      \end{itemize}
    \end{itemize}
  \end{block}
  
\end{frame}

\endgroup
\endinput

% SPDX-License-Identifier: CC-BY-SA-4.0
% Author: Matthieu Perrin
% Part: 
% Section: 
% Sub-section: 
% Frame: 

\begingroup

\SetKwFunction{Propose}{propose}
\SetKwData{Decided}{decided}

\begin{frame}{Le consensus}

  \on[text, top=-3mm]{
    \begin{algorithm}[H]
      \Interface{\textsc{consensus}}{
        \lMethod{\Propose$(v \in \mathcal{V}) \in \mathcal{V}$}{\tcp*[f]{Propose $v$, décide $v'$}}
      }
    \end{algorithm}

    \vspace{-1mm}
    \begin{description}[Terminaison :]
    \item[Accord :] Au plus une valeur décidée
    \item[Validité :] Toute valeur décidée a été proposée
    \item[Terminaison :] Tout processus correct finit par décider une valeur
    \end{description}
  }

  \onAlertBlock<2->[y=-4mm]{Réduction de Total-Order Broadcast au consensus}{
    \begin{algorithm}[H]
      \lLVariables{}{$\Decided_i \leftarrow \bot$;}
      \Method{\Propose$(v \in \mathcal{V}) \in \mathcal{V}$}{
        \textsc{to}.\SBroadcast $\textsc{propose}(v)$;\\
        \Return $\Decided_i$;
      }
      \When{\textsc{to}.\Deliver $\textsc{propose}(v)$ \From $p_j$}{
        \lIf{$\Decided_i = \bot$}{$\Decided_i \leftarrow v$;}
      }
    \end{algorithm}
  }

  \on<2->[y=-9mm, right=.45\textwidth]{
    \begin{tikzpicture}[y=10mm]
      \draw[process] (0,1) node[left]{$p_1$} to (4,1);
      \draw[process] (0,0) node[left]{$p_2$} to (4,0);

      \node[alert, operation]     (p1) at (2,1) {$\Propose(a) \rightarrow b$};
      \node[structure, operation] (p2) at (2,0) {$\Propose(b) \rightarrow b$};

      \draw[alert, message]       (p1.west) to[bend left]  (p1.east);
      \draw[alert, message]       (p1.west) to[bend below] (3.75,0);
      \draw[structure, message]   (p2.west) to             (2,1);
      \draw[structure, message]   (p2.west) to[bend right] (p2.east);
    \end{tikzpicture}
  }

  \onAlertBlock<3->[y=-25mm]{Théorème (FLP85)}{
    Il est impossible de résoudre le consensus dans $\mathcal{CAMP}_{n, t}[t\ge 1]$
  }

  \on[bottom=-2mm,text]{
    \begin{citing}
    \item[PSL80] M. Pease, R. Shostak, L. Lamport. \textit{Reaching agreement in the presence of faults.} JACM. 1980
    \uncover<3->{\item[FLP85] M. J. Fischer, N. A. Lynch, M. S. Paterson. \textit{Impossibility of Distributed Consensus with One Faulty Process.} JACM, 1985}
    \end{citing}
  }

\end{frame}

\endgroup
\endinput

% SPDX-License-Identifier: CC-BY-SA-4.0
% Author: Matthieu Perrin
% Part: 
% Section: 
% Sub-section: 
% Frame: 

\begingroup

\begin{frame}{Du consensus à Total-Order Broadcast}

  \on[top=-5mm]{\small
    \begin{algorithm}[H]
      \SetKwFunction{Propose}{propose}
      \lSVariables{}{$\textsc{cons}[1, 2, ...]$: sequence of consensus objects}
      \lLVariables{}{$\mathit{broadcast}_i, \mathit{delivered}_i \leftarrow \varepsilon$: lists of pairs $\langle \mathit{message}, \mathit{pid} \rangle$}
      \Method{\textsc{to}.\Broadcast $(m)$}{
        \alt<6->{
          \Structure{\textsc{rb}.\Broadcast $\textsc{msg}(m)$};\\
        }{
          \Structure<2>{$\mathit{broadcast}_i \leftarrow \mathit{broadcast}_i \cdot \langle m, i \rangle$};\\
        }
        \Example<2,5>{\Wait $\langle m, i \rangle \in \mathit{delivered}_i$};\\
      }
      \Task{}{
        \For{\Alert<-3>{$k = 1, 2, 3, ...$}}{
          \Structure<2,4>{\Wait $\mathit{broadcast}_i \setminus \mathit{delivered}_i \neq \varepsilon$};\\
          \Structure<2>{\Let $\mathit{proposed} \leftarrow \left(\mathit{broadcast}_i \setminus \mathit{delivered}_i\right)[1]$};\\
          \Alert<-3>{\Let $\langle m, j \rangle \leftarrow \textsc{cons}[k].\Propose(\mathit{proposed})$};\\
          \Alert<-3>{\textsc{to}.\Deliver $m$ \From $p_j$};\\
          \Example<2>{$\mathit{delivered}_i \leftarrow \mathit{delivered}_i \cdot \langle m, j \rangle$};
        }
      }
      \uncover<6->{
        \lWhen{\Structureb{\textsc{rb}.\Deliver $\textsc{msg}(m)$ \From $p_j$}}{
          \Structure<1>{$\mathit{broadcast}_i \leftarrow \mathit{broadcast}_i \cdot \langle m, j \rangle$;}
        }
      }
    \end{algorithm}
  }
  
  \onAlertBlock<-2>[y=-4mm, anchor=north]{Variables}{\vspace{-1mm}
    \begin{description}
    \item<2>[$\structure{\mathit{broadcast}_i[k]}$] \vspace{-1mm} $k^{\text{e}}$ message \textsc{to}.\Broadcast par $p_i$
    \item[${\textsc{cons}[k]}$] \vspace{-1mm} accord sur le $k^{\text{e}}$ message à délivrer
    \item<2>[${\color{exampleColor}\mathit{delivered}_i[k]}$] \vspace{-1mm} $k^{\text{e}}$ message \textsc{to}.\Deliver par $p_i$ (résultat de ${\textsc{cons}[k]}$)
    \end{description}
  }
  
  \on<-2>[y=-33mm]{
    \begin{tikzpicture}[y=8mm]
      \tikzset{
        box/.style={fill highlighted,text highlighted,draw highlighted,rounded corners, text width=15mm, minimum height=3mm, align=center, font=\footnotesize},
        tbox/.style={fill highlighted,rounded corners, inner sep=0mm, outer sep=0mm,},
      }
      \draw[->] (1,1) node[left]{$p_i$} -- (12,1);
      \draw[->] (1,0) node[left]{$p_j$} -- (12,0);
      
      \node[structure, box] (C1) at (3  ,.5) {$\textsc{cons}[1]$};
      \node[alert,     box] (C2) at (6.5,.5) {$\textsc{cons}[2]$};
      \node[example,   box] (C3) at (10 ,.5) {$\textsc{cons}[3]$};

      \node (i1p) at (1.5,1) {}; \node (i1d) at (4.5 ,1) {};
      \node (i2p) at (5.0,1) {}; \node (i2d) at (8.0 ,1) {};
      \node (i3p) at (8.5,1) {}; \node (i3d) at (11.5,1) {};
      \node (j1p) at (1.5,0) {}; \node (j1d) at (4.5 ,0) {};
      \node (j2p) at (5.0,0) {}; \node (j2d) at (8.0 ,0) {};

      \path[-latex, structure] (i1p.center) edge[bend right=5mm] ([yshift= .5mm]C1.west);
      \path[-latex, alert    ] (j1p.center) edge[bend left =5mm] ([yshift=-.5mm]C1.west);
      \path[-latex, example  ] (i2p.center) edge[bend right=5mm] ([yshift= .5mm]C2.west);
      \path[-latex, alert    ] (j2p.center) edge[bend left =5mm] ([yshift=-.5mm]C2.west);
      \path[-latex, example  ] (i3p.center) edge[bend right=5mm] ([yshift= .5mm]C3.west);

      \path[-latex, structure] ([yshift= .5mm]C1.east) edge[bend right=5mm] (i1d.center);
      \path[-latex, alert    ] ([yshift=-.5mm]C1.east) edge[bend left =5mm] (j1d.center);
      \path[-latex, alert    ] ([yshift= .5mm]C2.east) edge[bend right=5mm] (i2d.center);
      \path[-latex, alert    ] ([yshift=-.5mm]C2.east) edge[bend left =5mm] (j2d.center);
      \path[-latex, example  ] ([yshift= .5mm]C3.east) edge[bend right=5mm] (i3d.center);

      \node at (i1p) {$\bullet$}; \node[structure, below] at (i1p) {$m_{i, 1}$};
      \node at (i2p) {$\bullet$}; \node[alert, above    ] at (j1p) {$m_{j, 1}$};
      \node at (j1p) {$\bullet$}; \node[example, below  ] at (i2p) {$m_{i, 2}$};

      \begin{scope}[background]
        \node[structure, fit=(i1p)(i1d), tbox] {};
        \node[example, fit=(i2p)(i3d), tbox]   {};
        \node[alert, fit=(j1p)(j2d), tbox]     {};
      \end{scope}
    \end{tikzpicture}
  }

  \obAlertBlock<3>[y=-11mm, anchor=north]{Correction : sûreté}{
    \begin{itemize}
    \item Pour tout $k$, tous les processus décident le même $\langle m, j\rangle$ pour $\textsc{cons}[k]$
    \item Pour tout $p_i$, le $k^{e}$ message \textsc{to}.\Deliver par $p_i$ est le même
    \end{itemize}
  }

  \obAlertBlock<4->[y=-11mm, anchor=north]{Correction : vivacité}{
    \begin{itemize}
    \item Livraison par les processus non-émetteurs ? 
    \item<5-> Terminaison de \textsc{to}.\Broadcast ?
    \end{itemize}
    \uncover<6>{\structure{Helping} : diffuser son message ; proposer les messages des autres}
  }

  \footnoteref{T.D. Chandra, S. Toueg. \textit{Unreliable failure detectors for reliable distributed systems.} JACM (1996)}

\end{frame}

\endgroup
\endinput

 
\section{Solutions au Consensus}
 
\subsection{L'algorithme de Ben-Or}
% SPDX-License-Identifier: CC-BY-SA-4.0
% Author: Matthieu Perrin
% Part: 
% Section: 
% Sub-section: 
% Frame: 

\begingroup

\begin{frame}{Consensus -- De la difficulté de tolérer les pannes}

  \on[y=-5mm]{
    \begin{tikzpicture}
      \node       [faded background picture=Seine,  text width=\paperwidth/3] (A) at (-\paperwidth/3,0) {};
      \node<4,5>  [faded background picture=Metro,  text width=\paperwidth/3] (B) at (0,0) {};
      \node<-3,6->[faded background picture=Jardin, text width=\paperwidth/3] (B) at (0,0) {};
      \node       [faded background picture=Salon,  text width=\paperwidth/3] (C) at (\paperwidth/3,0) {};
      \node[anchor=south, inner sep=0pt, outer sep=0pt] at (A.south) {\includegraphics[height=22mm]{Alice}};
      \node[anchor=south, inner sep=0pt, outer sep=0pt] at (B.south) {\includegraphics[height=22mm]{Bob}};
      \node[anchor=south, inner sep=0pt, outer sep=0pt] at (C.south) {\includegraphics[height=22mm]{Carole}};
    \end{tikzpicture}
  }

  \on[x=-\paperwidth/3, y=15mm]{
    \begin{chat}[color=structure]{Bob, Carole}
      \chatSend{On fait quoi ce soir ?}
      \onlyb<2>{
        \chatSend{Un restaurant ?}
      }
      \onlyb<3>{
        \chatSend{Un restaurant ?}
        \chatRecv[color=example]{On se fait un resto ?}
        \chatRecv[color=alert]  {Un petit resto ?}
      }
      \only<4-6>{
        \chatSend{Un restaurant ?}
        \chatRecv[color=alert]  {Un petit resto ?}
      }
      \onlyb<7>{
        \chatSend{Un cinéma ?}
      }
      \onlyb<8-9>{
        \chatSend{Un cinéma ?}
        \chatRecv[color=example]{On se fait un ciné ?}
      }
    \end{chat}
  }
  
  \on[y=15mm]{
    \begin{chat}[color=example]{Alice, Carole}
      \chatRecv[color=structure]{On fait quoi ce soir ?}
      \onlyb<2>{
        \chatSend{On se fait un resto ?}
      }
      \onlyb<3>{
        \chatSend{On se fait un resto ?}
        \chatRecv[color=structure]{Un restaurant ?}
        \chatRecv[color=alert]  {Un petit resto ?}
      }
      \onlyb<4>{
        \chatSend{On se fait un resto ?}
      }
      \onlyb<5>{
        \chatSend{On se fait un ciné ?}
      }
      \only<6-8>{
        \chatSend{On se fait un ciné ?}
        \chatRecv[color=alert]  {Un petit resto ?}
      }
      \onlyb<9>{
        \chatSend{On se fait un ciné ?}
        \chatRecv[color=alert]  {Un petit ciné ?}
      }
    \end{chat}
  }
  
  \on[x=\paperwidth/3, y=15mm]{
    \begin{chat}[color=alert]{Alice, Bob}
      \chatRecv[color=structure]{On fait quoi ce soir ?}
      \onlyb<2>{
        \chatSend{Un petit resto ?}
      }
      \onlyb<3>{
        \chatSend{Un petit resto ?}
        \chatRecv[color=structure]{Un restaurant ?}
        \chatRecv[color=example]{On se fait un resto ?}
      }
      \onlyb<4-5>{
        \chatSend{Un petit resto ?}
        \chatRecv[color=structure]{Un restaurant ?}
      }
      \only<6-8>{
        \chatSend{Un petit resto ?}
        \chatRecv[color=example]{On se fait un ciné ?}
      }
      \onlyb<9>{
        \chatSend{Un petit ciné ?}
      }
    \end{chat}
  }
  
  \ob<1>[x=-\paperwidth/3, y=-15mm]{
    \chatBubble[color=structure]{On fait quoi ce soir ?}
  }
  
  \ob<2>[x=-\paperwidth/3, y=-15mm]{
    \chatBubble[color=structure]{Un restaurant ?}
  }
  \ob<2>[y=-15mm]{
    \chatBubble[color=example]{On se fait un resto ?}
  }
  \ob<2>[x=\paperwidth/3, y=-15mm]{
    \chatBubble[color=alert]{Un petit resto ?}
  }
  
  \ob<3>[x=-\paperwidth/3, y=-15mm]{
    \chatBubble[color=structure]{Restaurant !\\\small (Validité)}
  }
  \ob<3>[y=-15mm]{
    \chatBubble[color=example]{Restaurant !\\\small (Validité)}
  }
  \ob<3>[x=\paperwidth/3, y=-15mm]{
    \chatBubble[color=alert]{Restaurant !\\\small (Validité)}
  }
  
  \ob<4>[x=-\paperwidth/3, y=-15mm]{
    \chatBubble[color=structure]{Restaurant !\\\small (Tolérance aux fautes)}
  }
  \ob<4>[x=\paperwidth/3, y=-15mm]{
    \chatBubble[color=alert]{Restaurant !\\\small (Tolérance aux fautes)}
  }
  
  \on<5-6>[x=-\paperwidth/3, y=-15mm]{
    \chatBubble[color=structure]{Restaurant !\\\small (Indistinguabilité)}
  }
  \ob<5>[y=-15mm]{
    \chatBubble[color=example]{On se fait un ciné ?}
  }
  \ob<5>[x=\paperwidth/3, y=-15mm]{
    \chatBubble[color=alert]{Restaurant !\\\small (Indistinguabilité)}
  }
  \on<6>[y=-15mm]{
    \chatBubble[color=example]{Restaurant !\\\small (Accord)}
  }
  \on<6>[x=\paperwidth/3, y=-15mm]{
    \chatBubble[color=alert]{Restaurant !\\\small (Accord)}
  }

  \ob<7>[x=-\paperwidth/3, y=-15mm]{
    \chatBubble[color=structure]{Un cinéma ?}
  }
  \ob<8>[x=-\paperwidth/3, y=-15mm]{
    \chatBubble[color=structure]{Restaurant !\\\small (Accord)}
  }
  \ob<7-8>[y=-15mm]{
    \chatBubble[color=example]{Restaurant !\\\small (Indistinguabilité)}
  }
  \ob<7-8>[x=\paperwidth/3, y=-15mm]{
    \chatBubble[color=alert]{Restaurant !\\\small (Indistinguabilité)}
  }

  \ob<9>[x=-\paperwidth/3, y=-15mm]{
    \chatBubble[color=structure]{Restaurant !\\\small (Indistinguabilité)}
  }
  \ob<9>[y=-15mm]{
    \chatBubble[color=example]{Restaurant !\\\small (Accord)}
  }
  \ob<9>[x=\paperwidth/3, y=-15mm]{
    \chatBubble[color=alert]{Un petit ciné ?}
  }
  
\end{frame}

\endgroup
\endinput

% SPDX-License-Identifier: CC-BY-SA-4.0
% Author: Matthieu Perrin
% Part: 
% Section: 
% Sub-section: 
% Frame: 

\begingroup

\SetKwFunction{Filter}{filter}
\SetKwData{Received}{received}
\SetKwData{Values}{values}

\tikzset{
  output/.style={
    rectangle,
    minimum height=2mm,
    anchor=north,
    align=center,
    draw highlighted,
    fill highlighted,
    outer sep=0pt,
    inner sep=1pt,
    text width=10mm,
  },
}


\begin{frame}{L'algorithme de filtre $(t < \frac{n}{2})$}

  \on[top=-3mm]{
    \begin{algorithm}[H]
      \Algo{\textsc{filter}}{
        \lLVariables{}{
          $\Received_i \leftarrow 0$;
          $\Values_i \leftarrow \emptyset$;
        }
        \Method{$\Filter(v)$}{
          \textsc{sta}.\Broadcast $\textsc{filter}(v)$;\\
          \Wait $\Structure{\Received_i \ge n-t}$;\\
          \Return $\Values_i$;
        }
        \When{\textsc{sta}.\Deliver $\textsc{filter}(v)$}{
          $\Values_i \leftarrow \Values_i \cup \{v\}$;\\
          $\Received_i \leftarrow \Received_i + 1$;\\
        }
      }
    \end{algorithm}
  }

  \ob<2>[y=13mm, right=.5\textwidth]{
    \begin{tikzpicture}[y=10mm]
      \draw[process] (0,2) node[left]{$p_1$} to (5,2);
      \draw[process] (0,1) node[left]{$p_2$} to (5,1);
      \draw[process] (0,0) node[left]{$p_3$} to (5,0);

      \node[alert, operation, minimum width=35mm] (f1) at (2,2) {$\Filter(\bullet) \rightarrow \{\bullet\}$};
      \node[alert, operation, minimum width=35mm] (f2) at (2,1) {$\Filter(\bullet) \rightarrow \{\bullet\}$};
      \node[alert, operation, minimum width=35mm] (f3) at (2,0) {$\Filter(\bullet) \rightarrow \{\bullet\}$};
      
      \draw[alert!50, message] (f1.west) to[bend left=40]  (4.7,0);
      \draw[alert!50, message] (f3.west) to[bend right=40] (4.7,2);
      \draw[alert!50, message] (f3.west) to[bend above]    (4.5,1);

      \draw[alert, message]    (f1.west) to[bend left]     ([xshift=4mm]f1.west);
      \draw[alert, message]    (f1.west) to[bend below]    (f2.east);
      \draw[alert, message]    (f2.west) to[bend above]    (f1.east);
      \draw[alert, message]    (f2.west) to[bend right]    ([xshift=4mm]f2.west);
      \draw[alert, message]    (f2.west) to[bend below]    (f3.east);
      \draw[alert, message]    (f3.west) to[bend right=60] ([xshift=4mm]f3.west);
    \end{tikzpicture}
  }
  
  \ob<3>[y=13mm, right=.5\textwidth]{
    \begin{tikzpicture}[y=10mm]
      \draw[process] (0,2) node[left]{$p_1$} to (5,2);
      \draw[process] (0,1) node[left]{$p_2$} to (5,1);
      \draw[process] (0,0) node[left]{$p_3$} to (5,0);

      \node[alert,     operation, minimum width=35mm] (f1) at (2,2) {$\Filter(\bullet) \rightarrow \{\example{\bullet}, \alert{\bullet}\}$};
      \node[example, operation, minimum width=35mm] (f2) at (2,1) {$\Filter(\bullet) \rightarrow \{\example{\bullet}, \alert{\bullet}\}$};
      \node[example, operation, minimum width=35mm] (f3) at (2,0) {$\Filter(\bullet) \rightarrow \{\example{\bullet}\}$};
      
      \draw[alert!50,     message]   (f1.west) to[bend left=40]   (4.7,0);
      \draw[example!50, message]   (f3.west) to[bend right=40]  (4.7,2);
      \draw[example!50, message]   (f3.west) to[bend above]     (4.5,1);

      \draw[alert,     message]   (f1.west) to[bend left]     ([xshift=4mm]f1.west);
      \draw[alert,     message]   (f1.west) to[bend below]    (f2.east);
      \draw[example, message]   (f2.west) to[bend above]    (f1.east);
      \draw[example, message]   (f2.west) to[bend right]    ([xshift=4mm]f2.west);
      \draw[example, message]   (f2.west) to[bend below]    (f3.east);
      \draw[example, message]   (f3.west) to[bend right=60] ([xshift=4mm]f3.west);
    \end{tikzpicture}
  }

  \on<4>[y=13mm, right=.5\textwidth]{
    \begin{tikzpicture}[y=10mm]
      \draw[process] (0,2) node[left]{$p_1$} to (5,2);
      \draw[process] (0,1) node[left]{$p_2$} to (5,1);
      \draw[process] (0,0) node[left]{$p_3$} to (5,0);

      \node[alert,     operation, minimum width=35mm] (f1) at (2,2) {$\Filter(\bullet) \rightarrow \{\example{\bullet}, \alert{\bullet}\}$};
      \node[example, operation, minimum width=35mm] (f2) at (2,1) {$\Filter(\bullet) \rightarrow \{\example{\bullet}\}$};
      \node[example, operation, minimum width=35mm] (f3) at (2,0) {$\Filter(\bullet) \rightarrow \{\example{\bullet}\}$};
      
      \draw[alert!50,     message]   (f1.west) to[bend left=40]   (4.7,0);
      \draw[example!50, message]   (f3.west) to[bend right=40]  (4.7,2);
      \draw[alert!50,     message]   (f1.west) to[bend below]    (4.5,1);

      \draw[alert,     message]   (f1.west) to[bend left]     ([xshift=4mm]f1.west);
      \draw[example, message]   (f2.west) to[bend above]    (f1.east);
      \draw[example, message]   (f2.west) to[bend right]    ([xshift=4mm]f2.west);
      \draw[example, message]   (f2.west) to[bend below]    (f3.east);
      \draw[example, message]   (f3.west) to[bend above]    (f2.east);
      \draw[example, message]   (f3.west) to[bend right=60] ([xshift=4mm]f3.west);
    \end{tikzpicture}
  }
  
  \onBlock<2->[anchor=north]{Propriétés}{
    \begin{tikzpicture}[x=12mm]
      \node<2->[alert,               output] (I11) at (1,3)       {$\bullet$} ;
      \node<2->[alert,               output] (I12) at (I11.south) {$\bullet$} ;
      \node<2->[alert,               output] (I13) at (I12.south) {$\bullet$} ;
      \node<4->[alert,               output] (I21) at (3,3)       {$\bullet$} ;
      \node<4->[alert,               output] (I22) at (I21.south) {$\bullet$} ;
      \node<4->[example,             output] (I23) at (I22.south) {$\bullet$} ;
      \node<3->[alert,               output] (I31) at (5,3)       {$\bullet$} ;
      \node<3->[example,             output] (I32) at (I31.south) {$\bullet$} ;
      \node<3->[example,             output] (I33) at (I32.south) {$\bullet$} ;
      \node<2->[example,             output] (I41) at (7,3)       {$\bullet$} ;
      \node<2->[example,             output] (I42) at (I41.south) {$\bullet$} ;
      \node<2->[example,             output] (I43) at (I42.south) {$\bullet$} ;
      
      \node<2->[alert,               output] (O11) at (1,2)       {$\{\bullet\}$} ;
      \node<2->[alert,               output] (O12) at (O11.south) {$\{\bullet\}$} ;
      \node<2->[alert,               output] (O13) at (O12.south) {$\{\bullet\}$} ;
      \node<4->[alert,               output] (O21) at (2,2)       {$\{\bullet\}$} ;
      \node<4->[alert,               output] (O22) at (O21.south) {$\{\bullet\}$} ;
      \node<4->[draw, fill=black!20, output] (O23) at (O22.south) {$\{\alert{\bullet}, \example{\bullet}\}$} ;
      \node<4->[alert,               output] (O31) at (3,2)       {$\{\bullet\}$} ;
      \node<4->[draw, fill=black!20, output] (O32) at (O31.south) {$\{\alert{\bullet}, \example{\bullet}\}$} ;
      \node<4->[draw, fill=black!20, output] (O33) at (O32.south) {$\{\alert{\bullet}, \example{\bullet}\}$} ;
      \node<4->[draw, fill=black!20, output] (O41) at (4,2)       {$\{\alert{\bullet}, \example{\bullet}\}$} ;
      \node<4->[draw, fill=black!20, output] (O42) at (O41.south) {$\{\alert{\bullet}, \example{\bullet}\}$} ;
      \node<4->[draw, fill=black!20, output] (O43) at (O42.south) {$\{\alert{\bullet}, \example{\bullet}\}$} ;
      \node<3->[draw, fill=black!20, output] (O51) at (5,2)       {$\{\alert{\bullet}, \example{\bullet}\}$} ;
      \node<3->[draw, fill=black!20, output] (O52) at (O51.south) {$\{\alert{\bullet}, \example{\bullet}\}$} ;
      \node<3->[example,             output] (O53) at (O52.south) {$\{\bullet\}$} ;
      \node<4->[draw, fill=black!20, output] (O61) at (6,2)       {$\{\alert{\bullet}, \example{\bullet}\}$} ;
      \node<4->[example,             output] (O62) at (O61.south) {$\{\bullet\}$} ;
      \node<4->[example,             output] (O63) at (O62.south) {$\{\bullet\}$} ;
      \node<2->[example,             output] (O71) at (7,2)       {$\{\bullet\}$} ;
      \node<2->[example,             output] (O72) at (O71.south) {$\{\bullet\}$} ;
      \node<2->[example,             output] (O73) at (O72.south) {$\{\bullet\}$} ;

      \path<2->[-latex] (I13) edge (O11);
      \path<4->[-latex] (I23) edge (O21);
      \path<4->[-latex] (I23) edge (O31);
      \path<4->[-latex] (I23) edge (O41);
      \path<4->[-latex] (I33) edge (O41);
      \path<3->[-latex] (I33) edge (O51);
      \path<4->[-latex] (I33) edge (O61);
      \path<2->[-latex] (I43) edge (O71);

      \draw<2-> (I12.west) node[left]{Entrées :} ;
      \draw<2-> (O12-|I12.west) node[left]{Sorties :} ;
    \end{tikzpicture}

    \begin{description}
    \item<2->[Conservation :] Si une seule valeur est proposée, l'accord est conservé 
    \item<4->[Pré-accord :] Impossible que $p_i$ retourne $\{v\}$ et $p_j$ retourne $\{w\}$, avec $v\neq w$
    \end{description}
  }

\end{frame}

\endgroup
\endinput


% SPDX-License-Identifier: CC-BY-SA-4.0
% Author: Matthieu Perrin
% Part: 
% Section: 
% Sub-section: 
% Frame: 

\begingroup

\SetKwFunction{Propose}{propose}
\SetKwFunction{Filter}{filter}
\SetKwData{F}{f}

\tikzset{
  output adopt/.style={
    rectangle,
    minimum height=2.5mm,
    anchor=north,
    align=center,
    draw highlighted,
    text highlighted,
    outer sep=0pt,
    inner sep=1pt,
    text width=6mm,
  },
  algo value/.style={
    rectangle,
    minimum height=2.5mm,
    anchor=north,
    align=center,
    draw highlighted,
    fill highlighted,
    outer sep=0pt,
    inner sep=1pt,
  },
  output/.style={
    algo value,
    text width=6mm,
  },
  output grey/.style={
    draw,
    fill=black!20,
    output,
  },
  output graph/.style={
    rectangle,
    rounded corners,
    outer sep=0pt,
    inner sep=1pt,
    minimum width=12mm,
    minimum height=4mm,
    draw,
  },
}

\newcommand{\Abort}{\textsc{Abort}}
\newcommand{\Adopt}{\textsc{Adopt}}
\newcommand{\Commit}{\textsc{Commit}}
\newcommand{\Filtered}{\mathit{filtered}}

\begin{frame}{L'abstraction Abort-Adopt-Commit}

  \on[top=-3mm]{
    \begin{algorithm}[H]
      \Algo{\textsc{aac}}{
        \lSVariables{}{
          $\F_1$, $\F_2$ : \textsc{filter};
        }
        \Method{$\Propose(v)$}{
          \Let $\Filtered = \F_2.\Filter(\F_1.\Filter(v))$;\\
          \lIf{$\exists w, \Filtered = \{\{w\}\}$}{
            \Return~~\Structure[draw=structure]{$\Commit(w)$}~~;
          }
          \lElseIf{$\exists w, \{w\} \in \Filtered$}{
            \Return~~\Structure[draw=structure, fill=none]{$\Adopt(w)$}~~;
          }
          \lElse{
            \Return~~\Fade[draw, inner sep=2pt]{$\Abort$}~~;
          }
        }
      }
    \end{algorithm}
  }

  \obBlock<1>[anchor=north, y=7mm]{Exécutions possibles pour 3 processus}{}
  
  \ob<1>[y=-24mm, x=-1mm]{
    \begin{tikzpicture}[x=8mm, y=12mm]
      \tiny
      \node[alert,   output] (I11) at (1,2)       {$\bullet$} ;
      \node[alert,   output] (I12) at (I11.south) {$\bullet$} ;
      \node[alert,   output] (I13) at (I12.south) {$\bullet$} ;
      \node[alert,   output] (I21) at (5,2)       {$\bullet$} ;
      \node[alert,   output] (I22) at (I21.south) {$\bullet$} ;
      \node[example, output] (I23) at (I22.south) {$\bullet$} ;
      \node[alert,   output] (I31) at (9,2)       {$\bullet$} ;
      \node[example, output] (I32) at (I31.south) {$\bullet$} ;
      \node[example, output] (I33) at (I32.south) {$\bullet$} ;
      \node[example, output] (I41) at (13,2)      {$\bullet$} ;
      \node[example, output] (I42) at (I41.south) {$\bullet$} ;
      \node[example, output] (I43) at (I42.south) {$\bullet$} ;
      
      \node[alert,   output] (O11) at (1,1)       {$\{\bullet\}$} ;
      \node[alert,   output] (O12) at (O11.south) {$\{\bullet\}$} ;
      \node[alert,   output] (O13) at (O12.south) {$\{\bullet\}$} ;
      \node[alert,   output] (O21) at (3,1)       {$\{\bullet\}$} ;
      \node[alert,   output] (O22) at (O21.south) {$\{\bullet\}$} ;
      \node[output grey]           (O23) at (O22.south) {$\{\alert{\bullet}, \example{\bullet}\}$} ;
      \node[alert,   output] (O31) at (5,1)       {$\{\bullet\}$} ;
      \node[output grey]           (O32) at (O31.south) {$\{\alert{\bullet}, \example{\bullet}\}$} ;
      \node[output grey]           (O33) at (O32.south) {$\{\alert{\bullet}, \example{\bullet}\}$} ;
      \node[output grey]           (O41) at (7,1)       {$\{\alert{\bullet}, \example{\bullet}\}$} ;
      \node[output grey]           (O42) at (O41.south) {$\{\alert{\bullet}, \example{\bullet}\}$} ;
      \node[output grey]           (O43) at (O42.south) {$\{\alert{\bullet}, \example{\bullet}\}$} ;
      \node[output grey]           (O51) at (9,1)       {$\{\alert{\bullet}, \example{\bullet}\}$} ;
      \node[output grey]           (O52) at (O51.south) {$\{\alert{\bullet}, \example{\bullet}\}$} ;
      \node[example, output] (O53) at (O52.south) {$\{\bullet\}$} ;
      \node[output grey]           (O61) at (11,1)      {$\{\alert{\bullet}, \example{\bullet}\}$} ;
      \node[example, output] (O62) at (O61.south) {$\{\bullet\}$} ;
      \node[example, output] (O63) at (O62.south) {$\{\bullet\}$} ;
      \node[example, output] (O71) at (13,1)      {$\{\bullet\}$} ;
      \node[example, output] (O72) at (O71.south) {$\{\bullet\}$} ;
      \node[example, output] (O73) at (O72.south) {$\{\bullet\}$} ;

      \node[alert,   output]   (F011) at (1,0)        {\Commit} ;
      \node[alert,   output]   (F012) at (F011.south) {\Commit} ;
      \node[alert,   output]   (F013) at (F012.south) {\Commit} ;
      \node[alert,   output]   (F021) at (2,0)        {\Commit} ;
      \node[alert,   output]   (F022) at (F021.south) {\Commit} ;
      \node[alert,   output adopt]   (F023) at (F022.south) {\Adopt} ;
      \node[alert,   output]   (F031) at (3,0)        {\Commit} ;
      \node[alert,   output adopt]   (F032) at (F031.south) {\Adopt} ;
      \node[alert,   output adopt]   (F033) at (F032.south) {\Adopt} ;
      \node[alert,   output adopt]   (F041) at (4,0)        {\Adopt} ;
      \node[alert,   output adopt]   (F042) at (F041.south) {\Adopt} ;
      \node[alert,   output adopt]   (F043) at (F042.south) {\Adopt} ;
      \node[alert,   output adopt]   (F051) at (5,0)        {\Adopt} ;
      \node[alert,   output adopt]   (F052) at (F051.south) {\Adopt} ;
      \node[output grey]             (F053) at (F052.south) {\Abort} ;
      \node[alert,   output adopt]   (F061) at (6,0)        {\Adopt} ;
      \node[output grey]             (F062) at (F061.south) {\Abort} ;
      \node[output grey]             (F063) at (F062.south) {\Abort} ;
      \node[output grey]             (F071) at (7,0)        {\Abort} ;
      \node[output grey]             (F072) at (F071.south) {\Abort} ;
      \node[output grey]             (F073) at (F072.south) {\Abort} ;
      \node[output grey]             (F081) at (8,0)        {\Abort} ;
      \node[output grey]             (F082) at (F081.south) {\Abort} ;
      \node[example,   output adopt] (F083) at (F082.south) {\Adopt} ;
      \node[output grey]             (F091) at (9,0)        {\Abort} ;
      \node[example,   output adopt] (F092) at (F091.south) {\Adopt} ;
      \node[example,   output adopt] (F093) at (F092.south) {\Adopt} ;
      \node[example,   output adopt] (F101) at (10,0)       {\Adopt} ;
      \node[example,   output adopt] (F102) at (F101.south) {\Adopt} ;
      \node[example,   output adopt] (F103) at (F102.south) {\Adopt} ;
      \node[example,   output adopt] (F111) at (11,0)       {\Adopt} ;
      \node[example,   output adopt] (F112) at (F111.south) {\Adopt} ;
      \node[example, output]   (F113) at (F112.south) {\Commit} ;
      \node[example,   output adopt] (F121) at (12,0)       {\Adopt} ;
      \node[example, output]   (F122) at (F121.south) {\Commit} ;
      \node[example, output]   (F123) at (F122.south) {\Commit} ;
      \node[example, output]   (F131) at (13,0)       {\Commit} ;
      \node[example, output]   (F132) at (F131.south) {\Commit} ;
      \node[example, output]   (F133) at (F132.south) {\Commit} ;
      
      \path[-latex] (I13) edge (O11);
      \path[-latex] (I23) edge (O21);
      \path[-latex] (I23) edge (O31);
      \path[-latex] (I23) edge (O41);
      \path[-latex] (I33) edge (O41);
      \path[-latex] (I33) edge (O51);
      \path[-latex] (I33) edge (O61);
      \path[-latex] (I43) edge (O71);

      \path[-latex] (O13) edge (F011);
      \path[-latex] (O23) edge (F021);
      \path[-latex] (O23) edge (F031);
      \path[-latex] (O23) edge (F041);
      \path[-latex] (O33) edge (F041);
      \path[-latex] (O33) edge (F051);
      \path[-latex] (O33) edge (F061);
      \path[-latex] (O43) edge (F071);

      \path[-latex] (O53) edge (F081);
      \path[-latex] (O53) edge (F091);
      \path[-latex] (O53) edge (F101);
      \path[-latex] (O63) edge (F101);
      \path[-latex] (O63) edge (F111);
      \path[-latex] (O63) edge (F121);
      \path[-latex] (O73) edge (F131);

      \draw (I12.west) node[left]{$v$} ;
      \draw (O12-|I12.west) node[left]{$\F_1$} ;
      \draw (F012-|I12.west) node[left]{$\F_2 \circ \F_1$} ;
    \end{tikzpicture}
  }

  \onBlock<2>[anchor=north, y=7mm, left=.7\textwidth]{Propriétés}{
    \begin{description}[Conservation :]
    \item[Terminaison :] \Propose termine pour les corrects
    \item[Conservation :] Si seulement $v$ est proposée, \\ seul $\Commit(v)$ est retourné
    \item[Pré-accord :] Les valeurs retournées sont \\\alert{identiques ou adjacentes} \\dans le graphe :
    \end{description}
  }
  \on<2>[y=-15mm, x=.33\textwidth]{
    \begin{tikzpicture}[y=8mm, x=16mm]
      \footnotesize
      \node[output graph, fill=black!20]                (ab) at (0,0) {$\Abort$};
      \node[output graph, structure]                    (ax) at (90:1) {$\Adopt(x)$};
      \node[output graph, structure, fill=structure!30] (cx) at (90:2) {$\Commit(x)$};
      \node[output graph, alert]                        (ay) at (-60:1) {$\Adopt(y)$};
      \node[output graph, alert, fill=alert!30]         (cy) at (-60:2) {$\Commit(y)$};
      \node[output graph, example]                      (az) at (-120:1) {$\Adopt(z)$};
      \node[output graph, example, fill=example!30]     (cz) at (-120:2) {$\Commit(z)$};

      \path (ab) edge (ax);
      \path (ab) edge (ay);
      \path (ab) edge (az);
      \path (ax) edge (cx);
      \path (ay) edge (cy);
      \path (az) edge (cz);
    \end{tikzpicture}
  }

  \on<2>[bottom=-2mm, text]{
    \begin{citing}
    \item[PMJ16] E. Gafni. \emph{Round-by-round fault detectors unifying synchrony and asynchrony.} PODC. 1998
    \item[AW24] H. Attiya, J. Welch. \emph{Multi-Valued Connected Consensus: A New Perspective on Crusader Agreement and Adopt-Commit.} OPODIS. 2023
    \end{citing}
  }  
  
\end{frame}

\endgroup
\endinput


% SPDX-License-Identifier: CC-BY-SA-4.0
% Author: Matthieu Perrin
% Part: 
% Section: 
% Sub-section: 
% Frame: 

\begingroup

\tikzset{
  output adopt/.style={
    rectangle,
    minimum height=2.5mm,
    anchor=north,
    align=center,
    draw highlighted,
    text highlighted,
    outer sep=0pt,
    inner sep=1pt,
    text width=6mm,
  },
  algo value/.style={
    rectangle,
    minimum height=2.5mm,
    anchor=north,
    align=center,
    draw highlighted,
    fill highlighted,
    outer sep=0pt,
    inner sep=1pt,
  },
  output/.style={
    algo value,
    text width=6mm,
  },
  output grey/.style={
    draw,
    fill=black!20,
    output,
  },
  output graph/.style={
    rectangle,
    rounded corners,
    outer sep=0pt,
    inner sep=1pt,
    minimum width=12mm,
    minimum height=4mm,
    draw,
  },
}

\SetKwFunction{Decide}{decide}
\SetKwFunction{Propose}{propose}
\SetKwFunction{Filter}{filter}
\SetKwFunction{Choose}{choose}
\SetKwData{AAC}{aac}

\newcommand{\Abort}{\textsc{Abort}}
\newcommand{\Adopt}{\textsc{Adopt}}
\newcommand{\Commit}{\textsc{Commit}}


\begin{frame}{Un algorithme toujours sûr}

  \on[top=-3mm]{
    \begin{algorithm}[H]
      \lSVariables{}{
        $\AAC[0, ...]$; \tcp*[f]{tableau d'objets AAC}
      }
      \Method{$\Propose(v)$}{
        \For{$\mathit{round}$ \From $0$ \To $\infty$}{
          \Switch{$\AAC[\mathit{round}].\Propose(v)$}{
            \lCase{$\Commit(w)$}{$\Decide(w)$; \tcp*[f]{diffuse $w$ puis termine}}
            \lCase{$\Adopt(w)$}{$v \leftarrow w$; \tcp*[f]{adopte $w$}}
            \lCase{$\Abort$}{$v \leftarrow \Choose()$;\tcp*[f]{choisit une valeur déjà vue}}
          }
        }
      }
    \end{algorithm}
  }

  \obBlock<1>[anchor=north, y=7mm]{Une ronde d'exécution pour 3 processus}{}
  
  \ob<1>[y=-24mm]{
    \begin{tikzpicture}[x=8mm, y=12mm]
      \tiny
      \node[alert,   output] (I11) at (1,1)       {$\bullet$} ;
      \node[alert,   output] (I12) at (I11.south) {$\bullet$} ;
      \node[alert,   output] (I13) at (I12.south) {$\bullet$} ;
      \node[alert,   output] (I21) at (5,1)       {$\bullet$} ;
      \node[alert,   output] (I22) at (I21.south) {$\bullet$} ;
      \node[example, output] (I23) at (I22.south) {$\bullet$} ;
      \node[alert,   output] (I31) at (9,1)       {$\bullet$} ;
      \node[example, output] (I32) at (I31.south) {$\bullet$} ;
      \node[example, output] (I33) at (I32.south) {$\bullet$} ;
      \node[example, output] (I41) at (13,1)      {$\bullet$} ;
      \node[example, output] (I42) at (I41.south) {$\bullet$} ;
      \node[example, output] (I43) at (I42.south) {$\bullet$} ;
      
      \node[alert,   output]   (F011) at (1,0)        {\Commit} ;
      \node[alert,   output]   (F012) at (F011.south) {\Commit} ;
      \node[alert,   output]   (F013) at (F012.south) {\Commit} ;
      \node[alert,   output]   (F021) at (2,0)        {\Commit} ;
      \node[alert,   output]   (F022) at (F021.south) {\Commit} ;
      \node[alert,   output adopt]   (F023) at (F022.south) {\Adopt} ;
      \node[alert,   output]   (F031) at (3,0)        {\Commit} ;
      \node[alert,   output adopt]   (F032) at (F031.south) {\Adopt} ;
      \node[alert,   output adopt]   (F033) at (F032.south) {\Adopt} ;
      \node[alert,   output adopt]   (F041) at (4,0)        {\Adopt} ;
      \node[alert,   output adopt]   (F042) at (F041.south) {\Adopt} ;
      \node[alert,   output adopt]   (F043) at (F042.south) {\Adopt} ;
      \node[alert,   output adopt]   (F051) at (5,0)        {\Adopt} ;
      \node[alert,   output adopt]   (F052) at (F051.south) {\Adopt} ;
      \node[output grey]             (F053) at (F052.south) {\Abort} ;
      \node[alert,   output adopt]   (F061) at (6,0)        {\Adopt} ;
      \node[output grey]             (F062) at (F061.south) {\Abort} ;
      \node[output grey]             (F063) at (F062.south) {\Abort} ;
      \node[output grey]             (F071) at (7,0)        {\Abort} ;
      \node[output grey]             (F072) at (F071.south) {\Abort} ;
      \node[output grey]             (F073) at (F072.south) {\Abort} ;
      \node[output grey]             (F081) at (8,0)        {\Abort} ;
      \node[output grey]             (F082) at (F081.south) {\Abort} ;
      \node[example,   output adopt] (F083) at (F082.south) {\Adopt} ;
      \node[output grey]             (F091) at (9,0)        {\Abort} ;
      \node[example,   output adopt] (F092) at (F091.south) {\Adopt} ;
      \node[example,   output adopt] (F093) at (F092.south) {\Adopt} ;
      \node[example,   output adopt] (F101) at (10,0)       {\Adopt} ;
      \node[example,   output adopt] (F102) at (F101.south) {\Adopt} ;
      \node[example,   output adopt] (F103) at (F102.south) {\Adopt} ;
      \node[example,   output adopt] (F111) at (11,0)       {\Adopt} ;
      \node[example,   output adopt] (F112) at (F111.south) {\Adopt} ;
      \node[example, output]   (F113) at (F112.south) {\Commit} ;
      \node[example,   output adopt] (F121) at (12,0)       {\Adopt} ;
      \node[example, output]   (F122) at (F121.south) {\Commit} ;
      \node[example, output]   (F123) at (F122.south) {\Commit} ;
      \node[example, output]   (F131) at (13,0)       {\Commit} ;
      \node[example, output]   (F132) at (F131.south) {\Commit} ;
      \node[example, output]   (F133) at (F132.south) {\Commit} ;

      \node[alert,   output] (IB11) at (4,-1)      {$\bullet$} ;
      \node[alert,   output] (IB12) at (IB11.south) {$\bullet$} ;
      \node[alert,   output] (IB13) at (IB12.south) {$\bullet$} ;
      \node[alert,   output] (IB21) at (6,-1)      {$\bullet$} ;
      \node[alert,   output] (IB22) at (IB21.south) {$\bullet$} ;
      \node[example, output] (IB23) at (IB22.south) {$\bullet$} ;
      \node[alert,   output] (IB31) at (8,-1)      {$\bullet$} ;
      \node[example, output] (IB32) at (IB31.south) {$\bullet$} ;
      \node[example, output] (IB33) at (IB32.south) {$\bullet$} ;
      \node[example, output] (IB41) at (10,-1)     {$\bullet$} ;
      \node[example, output] (IB42) at (IB41.south) {$\bullet$} ;
      \node[example, output] (IB43) at (IB42.south) {$\bullet$} ;

      \path[-latex] (I13) edge (F011);
      \path[-latex] (I23) edge (F021);
      \path[-latex] (I23) edge (F031);
      \path[-latex] (I23) edge (F041);
      \path[-latex] (I23) edge (F041);
      \path[-latex] (I23) edge (F051);
      \path[-latex] (I23) edge (F061);
      \path[-latex] (I23) edge (F071);
      \path[-latex] (I33) edge (F071);
      \path[-latex] (I33) edge (F081);
      \path[-latex] (I33) edge (F091);
      \path[-latex] (I33) edge (F101);
      \path[-latex] (I33) edge (F101);
      \path[-latex] (I33) edge (F111);
      \path[-latex] (I33) edge (F121);
      \path[-latex] (I43) edge (F131);

      \path[-latex] (F043) edge (IB11);
      \path[-latex] (F053) edge (IB11);
      \path[-latex] (F053) edge (IB21);
      \path[-latex] (F063) edge (IB11);
      \path[-latex] (F063) edge (IB21);
      \path[-latex] (F063) edge (IB31);

      \path[-latex] (F073) edge (IB11);
      \path[-latex] (F073) edge (IB21);
      \path[-latex] (F073) edge (IB31);
      \path[-latex] (F073) edge (IB41);

      \path[-latex] (F083) edge (IB21);
      \path[-latex] (F083) edge (IB31);
      \path[-latex] (F083) edge (IB41);
      \path[-latex] (F093) edge (IB31);
      \path[-latex] (F093) edge (IB41);
      \path[-latex] (F103) edge (IB41);

      \normalsize
      \path[brace] (F033.south east) edge[brace] node{$\alert{\Decide(\bullet)}$} (F013.south west);
      \path[brace] (F133.south east) edge[brace] node{$\example{\Decide(\bullet)}$} (F113.south west);
      
    \end{tikzpicture}
  }
  
  \onBlock<2>[anchor=north, y=7mm]{Garantie de sûreté, pas de vivacité}{
    \begin{description}[Validité :]
    \item[Validité :] Toute valeur décidée a été proposée
      \begin{itemize}
      \item l'algorithme n'invente pas de valeur
      \end{itemize}
    \item[Accord :] Au plus une valeur est décidée
      \begin{itemize}
      \item si $v$ est décidée, les processus ont $\Adopt(v)$ ou $\Commit(v)$
      \item terminaison garantie à la ronde suivante par conservation
      \end{itemize}
    \item[Terminaison :] Aucune garantie déterministe possible (FLP)
      \begin{itemize}
      \item Choix d'une valeur déterministe (valeur minimale, la plus vue...)
      \item Exécutions pathologiques même pour un ordonnanceur réaliste
      \end{itemize}
    \end{description}
  }

\end{frame}

\endgroup
\endinput


% SPDX-License-Identifier: CC-BY-SA-4.0
% Author: Matthieu Perrin
% Part: 
% Section: 
% Sub-section: 
% Frame: 

\begingroup

\tikzset{
  output adopt/.style={
    rectangle,
    minimum height=2.5mm,
    anchor=north,
    align=center,
    draw highlighted,
    text highlighted,
    outer sep=0pt,
    inner sep=1pt,
    text width=6mm,
  },
  algo value/.style={
    rectangle,
    minimum height=2.5mm,
    anchor=north,
    align=center,
    draw highlighted,
    fill highlighted,
    outer sep=0pt,
    inner sep=1pt,
  },
  output/.style={
    algo value,
    text width=6mm,
  },
  output grey/.style={
    draw,
    fill=black!20,
    output,
  },
  output graph/.style={
    rectangle,
    rounded corners,
    outer sep=0pt,
    inner sep=1pt,
    minimum width=12mm,
    minimum height=4mm,
    draw,
  },
}

\SetKwFunction{Decide}{decide}
\SetKwFunction{Propose}{propose}
\SetKwFunction{Filter}{filter}
\SetKwFunction{Draw}{draw}
\SetKwFunction{Rand}{random}
\SetKwData{AAC}{aac}
\SetKwData{Coin}{coin}

\newcommand{\Abort}{\textsc{Abort}}
\newcommand{\Adopt}{\textsc{Adopt}}
\newcommand{\Commit}{\textsc{Commit}}

\begin{frame}{L'algorithme de Ben-Or (consensus binaire)}

  \on[top=-3mm]{
    \begin{algorithm}[H]
      \lSVariables{}{
        $\AAC[0, ...]$; \only<2>{\Alert{$\Coin[0, ...]$;}}
      }
      \Method{$\Propose(v \in \mathbb{B}) \in \mathbb{B}$}{
        \For{$\mathit{round}$ \From $0$ \To $\infty$}{
          \Switch{$\AAC[\mathit{round}].\Propose(v)$}{
            \lCase{$\Commit(w)$}{$\Decide(w)$; \tcp*[f]{diffuse $w$ puis termine}}
            \lCase{$\Adopt(w)$}{$v \leftarrow w$; \tcp*[f]{adopte $w$}}
            \onlyb<1>{\lCase{$\Abort$}{$\Alert{v \leftarrow \Rand()}$;\tcp*[f]{choisit une valeur aléatoire}}}
            \only<2>{\lCase{$\Abort$}{$\Alert{v \leftarrow \Coin[\mathit{round}].\Draw()}$;\tcp*[f]{tirage commun}}}
          }
        }
      }
    \end{algorithm}
  }

  \obBlock<1>[anchor=north, y=8mm]{Probabilité de terminaison}{
    \begin{itemize}
    \item\vspace{-1mm} Terminaison à la ronde suivante avec probabilité au moins $\frac{1}{2^{n-1}}$
    \item\vspace{-1mm} Complexité moyenne en $\mathcal{O}(2^n)$ rondes
    \end{itemize}
  }
  
  \ob<1>[y=-25mm]{
    \begin{tikzpicture}[x=8mm, y=17mm]
      \tiny
      \node[alert,   output]         (F011) at (1,0)        {\Commit} ;
      \node[alert,   output]         (F012) at (F011.south) {\Commit} ;
      \node[alert,   output]         (F013) at (F012.south) {\Commit} ;
      \node[alert,   output]         (F021) at (2,0)        {\Commit} ;
      \node[alert,   output]         (F022) at (F021.south) {\Commit} ;
      \node[alert,   output adopt]   (F023) at (F022.south) {\Adopt} ;
      \node[alert,   output]         (F031) at (3,0)        {\Commit} ;
      \node[alert,   output adopt]   (F032) at (F031.south) {\Adopt} ;
      \node[alert,   output adopt]   (F033) at (F032.south) {\Adopt} ;
      \node[alert,   output adopt]   (F041) at (4,0)        {\Adopt} ;
      \node[alert,   output adopt]   (F042) at (F041.south) {\Adopt} ;
      \node[alert,   output adopt]   (F043) at (F042.south) {\Adopt} ;
      \node[alert,   output adopt]   (F051) at (5,0)        {\Adopt} ;
      \node[alert,   output adopt]   (F052) at (F051.south) {\Adopt} ;
      \node[output grey]             (F053) at (F052.south) {\Abort} ;
      \node[alert,   output adopt]   (F061) at (6,0)        {\Adopt} ;
      \node[output grey]             (F062) at (F061.south) {\Abort} ;
      \node[output grey]             (F063) at (F062.south) {\Abort} ;
      \node[output grey]             (F071) at (7,0)        {\Abort} ;
      \node[output grey]             (F072) at (F071.south) {\Abort} ;
      \node[output grey]             (F073) at (F072.south) {\Abort} ;
      \node[output grey]             (F081) at (8,0)        {\Abort} ;
      \node[output grey]             (F082) at (F081.south) {\Abort} ;
      \node[example,   output adopt] (F083) at (F082.south) {\Adopt} ;
      \node[output grey]             (F091) at (9,0)        {\Abort} ;
      \node[example,   output adopt] (F092) at (F091.south) {\Adopt} ;
      \node[example,   output adopt] (F093) at (F092.south) {\Adopt} ;
      \node[example,   output adopt] (F101) at (10,0)       {\Adopt} ;
      \node[example,   output adopt] (F102) at (F101.south) {\Adopt} ;
      \node[example,   output adopt] (F103) at (F102.south) {\Adopt} ;
      \node[example,   output adopt] (F111) at (11,0)       {\Adopt} ;
      \node[example,   output adopt] (F112) at (F111.south) {\Adopt} ;
      \node[example, output]         (F113) at (F112.south) {\Commit} ;
      \node[example,   output adopt] (F121) at (12,0)       {\Adopt} ;
      \node[example, output]         (F122) at (F121.south) {\Commit} ;
      \node[example, output]         (F123) at (F122.south) {\Commit} ;
      \node[example, output]         (F131) at (13,0)       {\Commit} ;
      \node[example, output]         (F132) at (F131.south) {\Commit} ;
      \node[example, output]         (F133) at (F132.south) {\Commit} ;

      \node[alert,   output] (IB11) at (4,-1)       {$\bullet$} ;
      \node[alert,   output] (IB12) at (IB11.south) {$\bullet$} ;
      \node[alert,   output] (IB13) at (IB12.south) {$\bullet$} ;
      \node[alert,   output] (IB21) at (6,-1)       {$\bullet$} ;
      \node[alert,   output] (IB22) at (IB21.south) {$\bullet$} ;
      \node[example, output] (IB23) at (IB22.south) {$\bullet$} ;
      \node[alert,   output] (IB31) at (8,-1)       {$\bullet$} ;
      \node[example, output] (IB32) at (IB31.south) {$\bullet$} ;
      \node[example, output] (IB33) at (IB32.south) {$\bullet$} ;
      \node[example, output] (IB41) at (10,-1)      {$\bullet$} ;
      \node[example, output] (IB42) at (IB41.south) {$\bullet$} ;
      \node[example, output] (IB43) at (IB42.south) {$\bullet$} ;

      \path[-latex, black!30] (F053) edge (IB21);
      \path[-latex, black!30] (F063) edge (IB21);
      \path[-latex, black!30] (F063) edge (IB31);
      \path[-latex, black!30] (F073) edge (IB21);
      \path[-latex, black!30] (F073) edge (IB31);
      \path[-latex, black!30] (F083) edge (IB21);
      \path[-latex, black!30] (F083) edge (IB31);
      \path[-latex, black!30] (F093) edge (IB31);

      \path[-latex] (F043) edge node[left] {$1$}           (IB11);
      \path[-latex] (F053) edge node[above]{$\frac{1}{2}$} (IB11);
      \path[-latex] (F063) edge node[above]{$\frac{1}{4}$} (IB11);
      \path[-latex] (F073) edge node[below]{$\frac{1}{8}$} (IB11);
      \path[-latex] (F073) edge node[below]{$\frac{1}{8}$} (IB41);
      \path[-latex] (F083) edge node[above]{$\frac{1}{4}$} (IB41);
      \path[-latex] (F093) edge node[above]{$\frac{1}{2}$} (IB41);
      \path[-latex] (F103) edge node[right]{$1$}           (IB41);

      \footnotesize
      \node[anchor=south, outer sep=3pt] at (F011) {$p=1$} ;
      \node[anchor=south, outer sep=3pt] at (F021) {$p=1$} ;
      \node[anchor=south, outer sep=3pt] at (F031) {$p=1$} ;
      \node[anchor=south, outer sep=3pt] at (F041) {$p=1$} ;
      \node[anchor=south, outer sep=3pt] at (F051) {$p=\frac{1}{2}$} ;
      \node[anchor=south, outer sep=3pt] at (F061) {$p=\frac{1}{4}$} ;
      \node[anchor=south, outer sep=3pt] at (F071) {$p=\frac{1}{4}$} ;
      \node[anchor=south, outer sep=3pt] at (F081) {$p=\frac{1}{4}$} ;
      \node[anchor=south, outer sep=3pt] at (F091) {$p=\frac{1}{2}$} ;
      \node[anchor=south, outer sep=3pt] at (F101) {$p=1$} ;
      \node[anchor=south, outer sep=3pt] at (F111) {$p=1$} ;
      \node[anchor=south, outer sep=3pt] at (F121) {$p=1$} ;
      \node[anchor=south, outer sep=3pt] at (F131) {$p=1$} ;
      
      \normalsize
      \path[brace] (F033.south east) edge[brace] node{$\alert{\Decide(\bullet)}$} (F013.south west);
      \path[brace] (F133.south east) edge[brace] node{$\example{\Decide(\bullet)}$} (F113.south west);
    \end{tikzpicture}
  }
  
  \onBlock<2>[anchor=north, y=8mm]{Notion de common coin}{
    \begin{description}[Pire cas :]
    \item[Biais :] Il existe $\rho > 0$ tel que
      \begin{itemize}
      \item Tous les processus tirent $0$ avec probabilité au moins $\rho$
      \item Tous les processus tirent $1$ avec probabilité au moins $\rho$
      \item Aucune garantie avec probabilité au plus $1- 2 \rho$
      \end{itemize}
    \item[Pire cas :] Terminaison à la ronde suivante avec probabilité au moins $\rho$
      \begin{itemize}
      \item Complexité moyenne en $\mathcal{O}(\frac{1}{\rho} + 1)$ rondes
      \end{itemize}
      \centering
      \begin{tikzpicture}[x=16mm, y=16mm]
        \tiny
        \node[alert,   output]       (A1) at (0,0)      {$\bullet$} ;
        \node[example, output]       (A2) at (A1.south) {$\bullet$} ;
        \node[example, output]       (A3) at (A2.south) {$\bullet$} ;
        \node[alert,   output adopt] (B1) at (1,0)      {\Adopt} ;
        \node[output grey]           (B2) at (B1.south) {\Abort} ;
        \node[output grey]           (B3) at (B2.south) {\Abort} ;
        \node[alert,   output]       (C1) at (2,0)      {$\bullet$} ;
        \node[alert,   output]       (C2) at (C1.south) {$\bullet$} ;
        \node[alert,   output]       (C3) at (C2.south) {$\bullet$} ;

        \path[-latex] (B2) edge node[above] {$\rho$} (C2);
        \path[-latex] (B2) edge node[above] {$\rho$} (A2);
      \end{tikzpicture}
    \end{description}
  }

  \footnoteref{M. Ben-Or. \textit{Another advantage of free choice for completely asynchronous agreement protocols.} PODC (1983)}
  
\end{frame}

\endgroup
\endinput


% SPDX-License-Identifier: CC-BY-SA-4.0
% Author: Matthieu Perrin
% Part: 
% Section: 
% Sub-section: 
% Frame: 

\begingroup

\SetKwFunction{Draw}{draw}
\SetKwFunction{Rand}{random}
\SetKwData{Count}{count}
\SetKwData{Sum}{sum}

\begin{frame}{Un common coin simple $(t < \sqrt{n})$}

  \on[top=-3mm]{
    \begin{algorithm}[H]
      \lLVariables{}{
        $\Count_i \leftarrow 0$; $\Sum_i \leftarrow 0$;
      }
      \Method{$\Draw() \in \mathbb{B}$}{
        \textsc{sta}.\Broadcast $\textsc{draw}(\Rand())$; \tcp*[f]{Tirage local dans $\{0, 1\}$}\\
        \Wait $\Count_i \ge n-t$; \tcp*[f]{Récolte $n - t$ valeur}\\
        \Return $\Sum_i \ge \frac{\Count_i}{2}$; \tcp*[f]{Retourne la valeur la plus reçue}
      }
      \When{\textsc{sta}.\Deliver $\textsc{draw}(r_j)$ \From $p_j$}{
        $\Sum_i \leftarrow \Sum_i + r_j$;
        $\Count_i \leftarrow \Count_i + 1$;\\
      }
    \end{algorithm}
  }

  \obExampleBlock<2>[y=10mm, anchor=north]{Exemple pour $n=9$, $t=2$, et $3$ processus tirent \example{$\Rand() = 1$}}{
    \begin{itemize}
    \item Messages \textsc{draw} envoyés : 
      \begin{itemize}
      \item \alert{$6$} processus envoient \alert{$\textsc{draw}(0)$}
      \item \example{$3$} processus envoient \example{$\textsc{draw}(1)$}
      \end{itemize}
    \item Messages \textsc{draw} reçus : 
      \begin{itemize}
      \item Chaque processus reçoit au moins $6-2 = \alert{4}$ messages \alert{$\textsc{draw}(0)$}
      \item Chaque processus reçoit au plus \example{$3$} messages \example{$\textsc{draw}(1)$}
      \end{itemize}
    \item Tous les processus reçoivent plus de \alert{$\textsc{draw}(0)$} que de \example{$\textsc{draw}(1)$}
      \begin{itemize}
      \item Tous les processus retournent $0$
      \end{itemize}
    \end{itemize}
  }

  \obBlock<3>[y=10mm, anchor=north]{Analyse}{
    \begin{itemize}
    \item\vspace{-1mm} Le nombre de messages \example{$\textsc{draw}(1)$} est \structure{\footnotesize$\displaystyle \sum_{i=1}^n r_i$}
    \item\vspace{-1mm} On cherche $\rho > 0$ tel que
      \alert{\footnotesize$\displaystyle \mathbb{P}\left( \sum_{i=1}^n r_i\leq \frac{n-t}{2}\right) \geq \rho $} et
      \example{\footnotesize$\displaystyle \mathbb{P}\left( \sum_{i=1}^n r_i\geq \frac{n+t}{2}\right) \geq \rho $}
    \end{itemize}
  }
  
  \onBlock<4>[y=10mm, anchor=north]{Théorème Central-Limite}{
    Pour $n$ variables aléatoires $r_1, ..., r_n$ réelles d'espérance $\mu$ et d'écart-type $\sigma$ :\\[-2mm]
    $$\forall \alpha>0,  \lim_{n \to \infty}\mathbb{P}\left(\frac{\sum_{i=1}^n r_i - n \mu}{\sigma\sqrt{n}}\leq \alpha\right)  =  \Phi(\alpha)$$
  }
  
  \on<3->[bottom=-3mm]{
    \begin{tikzpicture}
      \def\mu{0}       % moyenne
      \def\sigma{1}    % écart-type
      \def\A{4.5}      % facteur d'échelle vertical (optionnel)
      \def\B{0.12mm}      % facteur d'échelle vertical (optionnel)

      \uncoverb<3>{
        \draw[fill=alert!30]   (-5,0) rectangle (-4,1*\B);
        \draw[fill=alert!30]   (-4,0) rectangle (-3,9*\B);
        \draw[fill=alert!30]   (-3,0) rectangle (-2,36*\B);
        \draw[fill=alert!30]   (-2,0) rectangle (-1,84*\B);
        \draw[fill=black!20]   (-1,0) rectangle ( 0,126*\B);
        \draw[fill=black!20]   ( 0,0) rectangle ( 1,126*\B);
        \draw[fill=example!30] ( 1,0) rectangle ( 2,84*\B);
        \draw[fill=example!30] ( 2,0) rectangle ( 3,36*\B);
        \draw[fill=example!30] ( 3,0) rectangle ( 4,9*\B);
        \draw[fill=example!30] ( 4,0) rectangle ( 5,1*\B);
      }
      \uncover<4>{
        \draw[domain=-5:5, samples=500, smooth] plot (\x,{\A*(1/(\sigma*sqrt(2*pi)))*exp(-((\x-\mu)^2)/(2*\sigma^2))});
        \begin{scope}[background]
          \fill<4>[alert!30,   domain=-4:-1, samples=150, smooth] plot (\x,{\A*(1/(\sigma*sqrt(2*pi)))*exp(-((\x-\mu)^2)/(2*\sigma^2))}) -- (-1,0) -- (-4,0) -- cycle;
          \fill<4>[black!20,   domain=-1:1,  samples=100, smooth] plot (\x,{\A*(1/(\sigma*sqrt(2*pi)))*exp(-((\x-\mu)^2)/(2*\sigma^2))}) -- ( 1,0) -- (-1,0) -- cycle;
          \fill<4>[example!30, domain=1:4,   samples=150, smooth] plot (\x,{\A*(1/(\sigma*sqrt(2*pi)))*exp(-((\x-\mu)^2)/(2*\sigma^2))}) -- ( 4,0) -- ( 1,0) -- cycle;
        \end{scope}
      }

      \scriptsize
      \uncoverb<3>{
        \draw (-4.5,0) node[below]{$0$}; 
        \draw (-3.5,0) node[below]{$1$}; 
        \draw (-2.5,0) node[below]{$2$}; 
        \draw (-1.5,0) node[below]{$3$}; 
        \draw (-0.5,0) node[below]{$4$}; 
        \draw ( 0.5,0) node[below]{$5$}; 
        \draw ( 1.5,0) node[below]{$6$}; 
        \draw ( 2.5,0) node[below]{$7$}; 
        \draw ( 3.5,0) node[below]{$8$}; 
        \draw ( 4.5,0) node[below]{$9 = n$}; 
      }
      \uncover<4>{
        \draw (-5,0) node[below]{$0$}; 
        \draw ( 0,0) node[below]{$n/2$}; 
        \draw ( 5,0) node[below]{$t^2 = n$}; 
      }

      \draw[->]             (-5,0) -- ( 5.2,0) node[right]{$s$};
      \draw[->]             (-5,0) -- (-5  ,2) node[below left]{$\displaystyle\mathbb{P}\left(\sum_{i=1}^n r_i = s\right)$};
      \draw[densely dotted] (-1,0) -- (-1,2);
      \draw[densely dotted] ( 0,0) -- ( 0,2);
      \draw[densely dotted] ( 1,0) -- ( 1,2);
      \path[latex-latex]    (-1,2) edge node[above]{$t\only<4>{\simeq \sqrt{n}}$} (1,2);
      \node[right]       at ( 1,2) {$\frac{n+t}{2}$};
      \node[left]        at (-1,2) {$\frac{n-t}{2}$};

      \footnotesize
      \uncoverb<3>{
        \node   at (-3,1) {Si \alert{  $\displaystyle \sum_{i=1}^n r_i \leq \frac{n-t}{2}$}, accord sur \alert{$0$}};
        \node   at ( 3,1) {Si \example{$\displaystyle \sum_{i=1}^n r_i \geq \frac{n+t}{2}$}, accord sur \example{$1$}};
      }
      \uncover<4>{
        \node[alert]   at (-3,1) {$\displaystyle \lim_{n \to \infty}\mathbb{P}\left( \sum_{i=1}^n r_i\leq \frac{n-\sqrt{n}}{2} \right) \simeq 0.159$};
        \node[example] at ( 3,1) {$\displaystyle \lim_{n \to \infty}\mathbb{P}\left( \sum_{i=1}^n r_i\geq \frac{n+\sqrt{n}}{2} \right) \simeq 0.159$};
      }

    \end{tikzpicture}
  }

  \onlyb<-2>{
    \footnoteref{M. Rabin. \textit{Randomized byzantine generals.} SFCS. (1983)}
    \footnoteref{J. Aspnes, M. Herlihy. \textit{Fast randomized consensus using shared memory.} J. of Algorithms. (1990)}
  }

\end{frame}

\endgroup
\endinput

% SPDX-License-Identifier: CC-BY-SA-4.0
% Author: Matthieu Perrin
% Part: 
% Section: 
% Sub-section: 
% Frame: 

\begingroup

\SetKwFunction{Draw}{draw}
\SetKwFunction{Rand}{random}
\SetKwData{Count}{count}
\SetKwData{Sum}{sum}

\begin{frame}{Un common coin pour $t < \frac{n}{2}$}

  \on[top=-3mm]{
    \begin{algorithm}[H]
      \lLVariables{}{
        $\Count_i \leftarrow 0$; $\Sum_i \leftarrow 0$;
      }
      \Method{$\Draw() \in \mathbb{B}$}{
        \While(\tcp*[f]{Récolte $n^2$ valeurs}){\Alert<1>{$\Count_i < n^2$}}{
          \textsc{cb-mb}.\SBroadcast \Alert<2>{$\textsc{draw}(\Rand())$;} \tcp*[f]{Tirage dans $\{0, 1\}$}
        }
        \textsc{cb-mb}.\SBroadcast \Example<2>{$\textsc{synch}()$;} \tcp*[f]{Synchronisation : causal mutual}\\
        \Return $\Sum_i \ge \frac{\Count_i}{2}$; \tcp*[f]{Retourne la valeur la plus reçue}
      }
      \When{\textsc{cb-mb}.\Deliver $\textsc{draw}(r_j)$ \From $p_j$}{
        $\Sum_i \leftarrow \Sum_i + r_j$;
        $\Count_i \leftarrow \Count_i + 1$;\\
      }
    \end{algorithm}
  }

  \ob<1>[text, y=1mm, anchor=north]{
    \begin{itemize}
    \item Tirer \alert{$n^2$} valeurs
    \item Le désaccord entre deux processus doit être limitée à \alert{$n$} valeurs
      \begin{itemize}
      \item Synchronisation nécessaire à chaque tirage
      \end{itemize}
    \end{itemize}
  }

  \ob<1>[bottom=-3.5mm]{
    \begin{tikzpicture}
      \def\mu{0}       % moyenne
      \def\sigma{1}    % écart-type
      \def\A{4.5}      % facteur d'échelle vertical (optionnel)
      
      \draw[domain=-5:5, samples=500, smooth] plot (\x,{\A*(1/(\sigma*sqrt(2*pi)))*exp(-((\x-\mu)^2)/(2*\sigma^2))});
      \begin{scope}[background]
        \fill[alert!30,   domain=-4:-1, samples=150, smooth] plot (\x,{\A*(1/(\sigma*sqrt(2*pi)))*exp(-((\x-\mu)^2)/(2*\sigma^2))}) -- (-1,0) -- (-4,0) -- cycle;
        \fill[black!20,   domain=-1:1,  samples=100, smooth] plot (\x,{\A*(1/(\sigma*sqrt(2*pi)))*exp(-((\x-\mu)^2)/(2*\sigma^2))}) -- ( 1,0) -- (-1,0) -- cycle;
        \fill[example!30, domain=1:4,   samples=150, smooth] plot (\x,{\A*(1/(\sigma*sqrt(2*pi)))*exp(-((\x-\mu)^2)/(2*\sigma^2))}) -- ( 4,0) -- ( 1,0) -- cycle;
      \end{scope}
      
      \scriptsize
      \draw (-5,0) node[below]{$0$}; 
      \draw ( 0,0) node[below]{$n^2/2$}; 
      \draw ( 5,0) node[below]{$n^2$}; 
      
      \draw[->]             (-5,0) -- ( 5.2,0) node[right]{$s$};
      \draw[->]             (-5,0) -- (-5  ,2) node[below left]{$\displaystyle\mathbb{P}\left(\sum_{i=1}^{n^2} r_i = s\right)$};
      \draw[densely dotted] (-1,0) -- (-1,2);
      \draw[densely dotted] ( 0,0) -- ( 0,2);
      \draw[densely dotted] ( 1,0) -- ( 1,2);
      \path[latex-latex]    (-1,2) edge node[above]{$n$} (1,2);
      \node[right]       at ( 1,2) {$\frac{n^2+n}{2}$};
      \node[left]        at (-1,2) {$\frac{n^2-n}{2}$};
      
      \footnotesize
      \node[alert]   at (-3,1) {$\displaystyle \lim_{n \to \infty}\mathbb{P}\left( \sum_{i=1}^{n^2} r_i\leq \frac{n^2-n}{2} \right) \simeq 0.159$};
      \node[example] at ( 3,1) {$\displaystyle \lim_{n \to \infty}\mathbb{P}\left( \sum_{i=1}^{n^2} r_i\geq \frac{n^2+n}{2} \right) \simeq 0.159$};
    \end{tikzpicture}
  }

  \onBlock<2>[y=3mm, anchor=north]{Désaccord entre deux processus}{
    \begin{itemize}
    \item $p_i$ ne peut ignorer que le dernier message \alert{$\textsc{draw}$} de $p_j$ :
      \begin{itemize}
      \item Si $p_i$ \textsc{mb}.\Deliver \alert{$\textsc{draw}$} de $p_j$ après son propre \example{$\textsc{synch}$}, $p_j$ aussi
      \item $p_j$ \textsc{cb}.\Deliver au moins $n^2$ messages \structure{$\textsc{draw}$} avant le \example{$\textsc{synch}$} de $p_i$
      \end{itemize}
    \item donc la vue de $p_i$ et $p_j$ ne peut différer que de $n$ messages
    \end{itemize}
  }

  \on<2>[bottom=-2mm]{
    \begin{tikzpicture}[x=8mm, y=5mm, anchor=mid]
      \draw[process] (0,2) node[left]{$p_i$} to (10,2);
      \draw[process] (0,1) node[left]{$p_j$} to (10,1);
      \draw[process] (0,0) node[left]{$p_k$} to (10,0);

      \draw[structure] (4,2) node{$\bullet$} node[replica, above]{$\Count_i \ge n^2$};
      \draw[structure] (7,2) node{$\bullet$} node[replica, above]{$\Return$};
      \draw[structure] (5,1) node{$\bullet$} node[replica, below]{$\Count_i \ge n^2$};

      \path[alert!30,     message]   (1,2)  edge                (2,1);
      \path[alert!30,     message]   (1,2)  edge                (1.5,0);
      \path[example!30,   message]   (5,2)  edge                (6,0);
      \path[alert!30,     message]   (3,1)  edge                (4,0);
      \path[example!30,   message]   (8,1)  edge                (9,0);
      \path[structure!30, message]   (2,0)  edge[bend right]    (3,0);
      \path[example!30,   message]   (8,1)  edge                (9,2);
      \path[alert,        message]   (1,2)  edge[bend left]     (2,2);
      \path[example,      message]   (5,2)  edge[bend left]     (6,2);
      \path[example,      message]   (5,2)  edge                (6,1);
      \path[alert,        message]   (3,1)  edge[bend above=15] (8,2);
      \path[alert,        message]   (3,1)  edge[bend left=10]  (7,1);
      \path[example,      message]   (8,1)  edge[bend right]    (9,1);
      \path[structure,    message]   (2,0)  edge                (3,2);
      \path[structure,    message]   (2,0)  edge                (4,1);
    \end{tikzpicture}
  }
  
\end{frame}

\endgroup
\endinput

 
\subsection{L'algorithme Paxos}
% SPDX-License-Identifier: CC-BY-SA-4.0
% Author: Matthieu Perrin
% Part: 
% Section: 
% Sub-section: 
% Frame: 

\begingroup

\begin{frame}{Problème d'ordre FIFO}

  \on[y=-5mm]{
    \begin{tikzpicture}
      \node     [faded background picture=Seine,  text width=\paperwidth/2] (A) at (-\paperwidth/4,0) {};
      \node<1>  [faded background picture=Jardin, text width=\paperwidth/2] (B) at ( \paperwidth/4,0) {};
      \node<2,3>[faded background picture=Metro,  text width=\paperwidth/2] (B) at ( \paperwidth/4,0) {};
      \node<4>  [faded background picture=Rue,    text width=\paperwidth/2] (B) at ( \paperwidth/4,0) {};
      \node[anchor=south, nosep] at (A.south) {\includegraphics[height=22mm]{Alice}};
      \node[anchor=south, nosep] at (B.south) {\includegraphics[height=22mm]{Bob}};
    \end{tikzpicture}
  }

  \on[x=-\paperwidth/4, y=15mm]{
    \begin{chat}[color=structure]{Bob}
      \chatSend{Qu'est ce que tu es musclé !}
      \only<4->{\chatRecv[color=example]{Toi tu es très belle}}
      \only<4->{\chatRecv[color=example]{Hélas, ça n'a pas toujours été le cas}}
    \end{chat}
  }
 
  \on[x=\paperwidth/4, y=15mm]{
    \begin{chat}[color=example]{Alice}
      \chatRecv[color=structure]{Qu'est ce que tu es musclé !}
      \only<2->{\chatSend{Hélas, ça n'a pas toujours été le cas}}
      \only<3->{\chatSend{Toi tu es très belle}}
    \end{chat}
  }

  \ob<1>[x=-\paperwidth/4, y=-15mm]{
    \chatBubble[color=structure]{Qu'est ce que tu es musclé !}
  }
  
  \ob<2>[x=\paperwidth/4, y=-15mm]{
    \chatBubble[color=example]{Hélas, ça n'a pas toujours été le cas}
  }
 
  \ob<3>[x=\paperwidth/4, y=-15mm]{
    \chatBubble[color=example]{Toi tu es très belle}
  }
   
\end{frame}

\endgroup
\endinput

% SPDX-License-Identifier: CC-BY-SA-4.0
% Author: Matthieu Perrin
% Part: 
% Section: 
% Sub-section: 
% Frame: 

\begingroup

\SetKwFunction{Leader}{leader}
\SetKwFunction{Propose}{propose}
\SetKwData{Decided}{decided}

\begin{frame}[fragile]{Un algorithme centralisé}

  \onBlock[top=-4mm]{Le consensus sans pannes}{
    \begin{algorithm}[H]
      \LVariables{}{$\Decided_i \leftarrow \bot$;}
      \Method{$\Propose(v)$}{
        \If{\Alert{$i = \Leader()$}}{
          \textsc{rb}.\Broadcast $\textsc{decide}(v)$;
        }
        \Wait $\Decided_i \neq \bot$;\\
        \Return $\Decided_i$;\\
      }
      \When{\textsc{rb}.\Deliver $\textsc{decide}(v)$ \From $p_{\Leader}$}{
        $\Decided_i \leftarrow v$;
      }
    \end{algorithm}
  }

  \on[x=30mm, y=15mm]{
    \begin{tikzpicture}
      \draw[process] (0,2) node[left]{$p_1$} to (5,2); 
      \draw[process] (0,1) node[left]{$p_2$} to (5,1); 
      \draw[process] (0,0) node[left]{$p_3$} to (5,0); 

      \node[alert,     operation, minimum width=20mm] (p1) at (3,2)   {$\Propose(a) \rightarrow a$};
      \node[structure, operation, minimum width=30mm] (p2) at (2,1)   {$\Propose(b) \rightarrow a$};
      \node[structure, operation, minimum width=40mm] (p3) at (2.5,0) {$\Propose(c) \rightarrow a$};
      \draw[alert,   message]     (p1.west) to[bend left=30] (p1.east);
      \draw[alert,   message]     (p1.west) to[bend left=20] (p2.east);
      \draw[alert,   message]     (p1.west) to[bend left=30] (p3.east);
    \end{tikzpicture}
  }

  \onBlock[anchor=north, y=-7mm]{Définition -- Leader perpétuel}{
    Une fonction $\Leader()$ qui retourne un (identifiant de) processus telle que:
    \begin{itemize}
    \item $\Leader()$ retourne \alert{toujours le même} processus à tous les processus
    \item $\Leader()$ retourne un processus \alert{correct}
    \end{itemize}
    \alert{Problème :} Impossible de savoir que $\Leader()$ est \alert{correct} !
  }

  \footnoteref{E. Chang, R. Roberts. \textit{An improved algorithm for decentralized extrema-finding in circular configurations of processes.} CACM (1979)}
  
\end{frame}

\endgroup
\endinput

% SPDX-License-Identifier: CC-BY-SA-4.0
% Author: Matthieu Perrin
% Part: 
% Section: 
% Sub-section: 
% Frame: 

\begingroup

\SetKwFunction{Leader}{leader}
\SetKwFunction{Sleep}{sleep}
\SetKwFunction{Time}{current\_time}
\SetKwData{Lastheard}{last\_heard}

\begin{frame}[fragile]{Élection de leader}

  \on[top=-3mm]{
    \begin{algorithm}[H]
      \lLVariables{}{
        $\Lastheard_i \leftarrow [\Time(), ..., \Time()]$;
      }
      \Method{$\Leader()$}{
        \For(\tcp*[f]{ou tri déterministe selon le ping moyen}){$j$ \From $1$ \To $n$}{
          \lIf{$\Time() - \Lastheard_i[j] < 2 \Delta$}{
            \Return $j$;
          }
        }
      }
      \Task{}{
        \While{\True}{
          $\Sleep(\Delta)$;
          \textsc{sta}.\Broadcast $\textsc{heartbeat}()$;
        }
      }
      \When{\textsc{sta}.\Deliver $\textsc{heartbeat}()$ \From $p_j$}{
        $\Lastheard_i[j] \leftarrow \Time()$;
      }
    \end{algorithm}
  }

  \obBlock<1>[anchor=north, y=-2mm]{Analyse en système synchrone}{
    \vspace{-2mm}
    \begin{itemize}
    \item $\Time()$ avance à la même vitesse pour tous les processus
    \item Tous les messages sont acheminés en au plus $\Delta$ unités de temps
    \end{itemize}
  }

  \ob<1>[anchor=north, y=-22mm]{
    \begin{tikzpicture}[anchor=mid, y=6mm, x=20mm]

      \begin{scope}
        \clip (0.8,1.9) rectangle (2.1,2.1);
        \fill[structure!20]                     (0.8,2) +(0,-.1) rectangle +(2,.1);
        \fill[structure!60, path fading=east]   (0.8,2) +(0,-.1) rectangle +(1,.1);
        \fill[structure!20]                     (1.7,2) +(0,-.1) rectangle +(2,.1);
        \fill[structure!60, path fading=east]   (1.7,2) +(0,-.1) rectangle +(1,.1);
      \end{scope}

      \begin{scope}
        \clip (0.8,0.9) rectangle (5.2,1.1);
        \fill[structure!20]                     (0.8,1) +(0,-.1) rectangle +(2,.1);
        \fill[structure!60, path fading=east]   (0.8,1) +(0,-.1) rectangle +(1,.1);
        \fill[alert!20]                         (2.8,1) +(0,-.1) rectangle +(2,.1);
        \fill[alert!60, path fading=east]       (2.8,1) +(0,-.1) rectangle +(1,.1);
        \fill[alert!20]                         (3.8,1) +(0,-.1) rectangle +(2,.1);
        \fill[alert!60, path fading=east]       (3.8,1) +(0,-.1) rectangle +(1,.1);
        \fill[alert!20]                         (4.8,1) +(0,-.1) rectangle +(2,.1);
        \fill[alert!60, path fading=east]       (4.8,1) +(0,-.1) rectangle +(1,.1);
        \fill[structure!20]                     (1.7,1) +(0,-.1) rectangle +(2,.1);
        \fill[structure!60, path fading=east]   (1.7,1) +(0,-.1) rectangle +(1,.1);
      \end{scope}

      \begin{scope}
        \clip (0.8,-.1) rectangle (5.2,0.1);
        \fill[structure!20]                     (0.8,0) +(0,-.1) rectangle +(2,.1);
        \fill[structure!60, path fading=east]   (0.8,0) +(0,-.1) rectangle +(1,.1);
        \fill[alert!20]                         (2.8,0) +(0,-.1) rectangle +(2,.1);
        \fill[alert!60, path fading=east]       (2.8,0) +(0,-.1) rectangle +(1,.1);
        \fill[alert!20]                         (3.8,0) +(0,-.1) rectangle +(2,.1);
        \fill[alert!60, path fading=east]       (3.8,0) +(0,-.1) rectangle +(1,.1);
        \fill[alert!20]                         (4.8,0) +(0,-.1) rectangle +(2,.1);
        \fill[alert!60, path fading=east]       (4.8,0) +(0,-.1) rectangle +(1,.1);
        \fill[structure!20]                     (1.7,0) +(0,-.1) rectangle +(2,.1);
        \fill[structure!60, path fading=east]   (1.7,0) +(0,-.1) rectangle +(1,.1);
      \end{scope}
      
      \foreach \x in {2,...,5}{
        \path (\x,-.2) edge[brace] node{$\Delta$} +(-1,0);
        \draw[dotted] (\x,-.2) -- (\x,2.2);
      }
      \draw[dotted] (1,-.2) -- (1,2.2);

      \draw[crashed] (.8,2) node[left]{$p_1$} to (2.1,2); 
      \draw[process] (.8,1) node[left]{$p_2$} to (5.2,1); 
      \draw[process] (.8,0) node[left]{$p_3$} to (5.2,0); 

      \draw[structure, message] (1.1,2) to[bend left] (1.7,2);
      \draw[structure, message] (1.1,2) to            (1.7,1);
      \draw[structure, message] (1.1,2) to            (1.7,0);
      \draw[alert, message]     (1.2,1) to            (1.8,2);
      \draw[alert, message]     (1.2,1) to[bend left] (1.8,1);
      \draw[alert, message]     (1.2,1) to            (1.8,0);
      \draw[example, message]   (1.3,0) to            (1.9,2);
      \draw[example, message]   (1.3,0) to            (1.9,1);
      \draw[example, message]   (1.3,0) to[bend left] (1.9,0);

      \draw[alert, message]     (2.2,1) to[bend left] (2.8,1);
      \draw[alert, message]     (2.2,1) to            (2.8,0);
      \draw[example, message]   (2.3,0) to            (2.9,1);
      \draw[example, message]   (2.3,0) to[bend left] (2.9,0);

      \draw[alert, message]     (3.2,1) to[bend left] (3.8,1);
      \draw[alert, message]     (3.2,1) to            (3.8,0);
      \draw[example, message]   (3.3,0) to            (3.9,1);
      \draw[example, message]   (3.3,0) to[bend left] (3.9,0);

      \draw[alert, message]     (4.2,1) to[bend left] (4.8,1);
      \draw[alert, message]     (4.2,1) to            (4.8,0);
      \draw[example, message]   (4.3,0) to            (4.9,1);
      \draw[example, message]   (4.3,0) to[bend left] (4.9,0);

    \end{tikzpicture}
  }

  \obBlock<2>[anchor=north, y=-2mm]{Analyse en système asynchrone}{
    \vspace{-2mm}
    \begin{itemize}
    \item Si un processus tombe en panne, il finit par ne plus être élu
    \item Pas de garantie de stabilité
    \end{itemize}
  }

  \ob<2>[anchor=north, y=-22mm]{
    \begin{tikzpicture}[anchor=mid, y=6mm, x=20mm]

      \begin{scope}
        \clip (0.8,1.9) rectangle (2.1,2.1);
        \fill[structure!20]                     (0.8,2) +(0,-.1) rectangle +(2,.1);
        \fill[structure!60, path fading=east]   (0.8,2) +(0,-.1) rectangle +(1,.1);
        \fill[structure!20]                     (1.7,2) +(0,-.1) rectangle +(2,.1);
        \fill[structure!60, path fading=east]   (1.7,2) +(0,-.1) rectangle +(1,.1);
      \end{scope}

      \begin{scope}
        \clip (0.8,0.9) rectangle (5.2,1.1);
        \fill[alert!20]                         (2.8,1) +(0,-.1) rectangle +(2,.1);
        \fill[alert!60, path fading=east]       (2.8,1) +(0,-.1) rectangle +(1,.1);
        \fill[alert!20]                         (3.8,1) +(0,-.1) rectangle +(2,.1);
        \fill[alert!60, path fading=east]       (3.8,1) +(0,-.1) rectangle +(1,.1);
        \fill[alert!20]                         (4.8,1) +(0,-.1) rectangle +(2,.1);
        \fill[alert!60, path fading=east]       (4.8,1) +(0,-.1) rectangle +(1,.1);

        \fill[structure!20]                     (0.8,1) +(0,-.1) rectangle +(2,.1);
        \fill[structure!60, path fading=east]   (0.8,1) +(0,-.1) rectangle +(1,.1);
        \fill[structure!20]                     (3.0,1) +(0,-.1) rectangle +(2,.1);
        \fill[structure!60, path fading=east]   (3.0,1) +(0,-.1) rectangle +(1,.1);
      \end{scope}

      \begin{scope}
        \clip (0.8,-.1) rectangle (5.2,0.1);
        \fill[example!20]                       (1.9,0) +(0,-.1) rectangle +(1,.1);
        \fill[example!60, path fading=east]     (1.9,0) +(0,-.1) rectangle +(1,.1);
        \fill[example!20]                       (2.9,0) +(0,-.1) rectangle +(1,.1);
        \fill[example!60, path fading=east]     (2.9,0) +(0,-.1) rectangle +(1,.1);
        \fill[example!20]                       (3.9,0) +(0,-.1) rectangle +(1,.1);
        \fill[example!60, path fading=east]     (3.9,0) +(0,-.1) rectangle +(1,.1);
        \fill[example!20]                       (4.9,0) +(0,-.1) rectangle +(1,.1);
        \fill[example!60, path fading=east]     (4.9,0) +(0,-.1) rectangle +(1,.1);

        \fill[alert!20]                         (1.4,0) +(0,-.1) rectangle +(1,.1);
        \fill[alert!60, path fading=east]       (1.4,0) +(0,-.1) rectangle +(1,.1);
        \fill[alert!20]                         (3.4,0) +(0,-.1) rectangle +(1,.1);
        \fill[alert!60, path fading=east]       (3.4,0) +(0,-.1) rectangle +(1,.1);
        \fill[alert!20]                         (3.5,0) +(0,-.1) rectangle +(1,.1);
        \fill[alert!60, path fading=east]       (3.5,0) +(0,-.1) rectangle +(1,.1);
        \fill[alert!20]                         (5.0,0) +(0,-.1) rectangle +(1,.1);
        \fill[alert!60, path fading=east]       (5.0,0) +(0,-.1) rectangle +(1,.1);
        
        \fill[structure!20]                     (0.8,0) +(0,-.1) rectangle +(1,.1);
        \fill[structure!60, path fading=east]   (0.8,0) +(0,-.1) rectangle +(1,.1);
        \fill[structure!20]                     (1.2,0) +(0,-.1) rectangle +(1,.1);
        \fill[structure!60, path fading=east]   (1.2,0) +(0,-.1) rectangle +(1,.1);
      \end{scope}
      
      \draw[crashed] (.8,2) node[left]{$p_1$} to (2.1,2); 
      \draw[process] (.8,1) node[left]{$p_2$} to (5.2,1); 
      \draw[process] (.8,0) node[left]{$p_3$} to (5.2,0); 

      \draw[structure, message] (1.1,2) to[bend left] (1.7,2);
      \draw[structure, message] (1.1,2) to[bend below](3.0,1);
      \draw[structure, message] (1.1,2) to            (1.2,0);
      \draw[alert, message]     (1.2,1) to[bend left] (1.8,1);
      \draw[alert, message]     (1.2,1) to            (1.5,0);
      \draw[example, message]   (1.3,0) to            (1.9,1);
      \draw[example, message]   (1.3,0) to[bend left] (1.9,0);

      \draw[alert, message]     (2.2,1) to[bend left] (2.8,1);
      \draw[alert, message]     (2.2,1) to            (3.4,0);
      \draw[example, message]   (2.3,0) to            (2.9,1);
      \draw[example, message]   (2.3,0) to[bend left] (2.9,0);

      \draw[alert, message]     (3.2,1) to[bend left] (3.8,1);
      \draw[alert, message]     (3.2,1) to            (3.5,0);
      \draw[example, message]   (3.3,0) to            (3.9,1);
      \draw[example, message]   (3.3,0) to[bend left] (3.9,0);

      \draw[alert, message]     (4.2,1) to[bend left] (4.8,1);
      \draw[alert, message]     (4.2,1) to            (5.0,0);
      \draw[example, message]   (4.3,0) to            (4.9,1);
      \draw[example, message]   (4.3,0) to[bend left] (4.9,0);

    \end{tikzpicture}
  }
  
  \obBlock<3>[anchor=north, y=-2mm]{Cas pratique : Système partiellement synchrone}{
    \begin{description}[Périodes instables :] 
    \item[Périodes stables :] réseau plus ou moins synchrone, peu de crash
      \begin{itemize}
      \item \structure{élection} d'un leader stable
      \end{itemize}
    \item[Périodes instables :] pics de latence, partitions, crashs
      \begin{itemize}
      \item \structure{désaccord} sur le leader
      \end{itemize}
    \end{description}
    \centering
    \alert{Comment s'assurer que deux leaders prennent des décisions cohérentes ?}
  }
  
  \onBlock<4>[anchor=north, y=-2mm]{Définition -- Eventual leader $\Omega$}{
    \begin{algorithm}[H]
      \Interface{$\Omega$}{
        \lMethod{$\Leader()$}{\tcp*[f]{retourne un identifiant de processus}}
      }
    \end{algorithm}
    \begin{itemize}
    \item $\Leader()$ \alert{finira par} retourner \structure{le même} processus à tous les processus
    \item $\Leader()$ \alert{finira par} retourner un processus \structure{correct}
    \end{itemize}
  }

  \only<4>{
    \footnoteref{T. Chandra, V. Hadzilacos, S. Toueg. \textit{The weakest failure detector for solving consensus.} JACM (1996)}
    \footnoteref{T. Chandra, S. Toueg. \textit{Unreliable failure detectors for reliable distributed systems.} JACM (1996)}
  }
  
\end{frame}

\endgroup
\endinput

% SPDX-License-Identifier: CC-BY-SA-4.0
% Author: Matthieu Perrin
% Part: 
% Section: 
% Sub-section: 
% Frame: 

\begingroup

\SetKwFunction{Leader}{leader}
\SetKwFunction{Propose}{propose}

\begin{frame}[fragile]{L'algorithme Single-decree Paxos}

  \on[top=-3mm]{\small
    \begin{algorithm}[H]
      \lLVariables{}{
        \Alertb<1,5>{$\mathit{decided}_i \leftarrow \bot$};
        \Structureb<2>{$\mathit{ballot}_i \leftarrow \langle 0, 0 \rangle $};
        \Exampleb<3-5>{$\mathit{val}_i \leftarrow \bot$}
      }
      \Operation{$\Propose(v)$}{
        \lIf{$\mathit{val}_i = \bot$}{\Exampleb<5>{$\mathit{val}_i \leftarrow v$}}
        \While{\Alertb<1,5>{$\mathit{decided}_i = \bot$}}{
          \If{\Alertb<1,6>{$leader() = i$}}{
            \Structureb<2>{\Let $t$ \St $\langle t, i\rangle >  ballot_i$}\;
            \Structureb<3-4>{\SBroadcast{mutual} $\textsc{begin}(t)$}\;
            \lIf{\Structureb<3-4>{$\langle t, i\rangle = ballot_i$}}{ 
              \Exampleb<3-4>{\SBroadcast{mutual} $\textsc{voted}(t, \Exampleb<5>{val_i})$}%
            }
            \lIf{\Structureb<3-4>{$\langle t, i\rangle = ballot_i$}}{ 
              \Alertb<1,4>{\Broadcast{rb} $\textsc{decide}(\Exampleb<5>{val_i})$}%
            }
          }
        }
        \Alertb<1,5>{\Return $\mathit{decided}_i$}\;
      }
      \When{\Structureb<3-4>{\Deliver{mutual} $\textsc{begin}(t)$ \From $p_j$}}{
        \lIf{\Structureb<3-4>{$\langle t, j\rangle \ge \mathit{ballot}_i$}}{
          \Structureb<3-4>{$\mathit{ballot}_i \leftarrow \langle t, j \rangle $}%
        }
      }
      \When{\Exampleb<3-4>{\Deliver{mutual} $\textsc{voted}(t,v)$ \From $p_j$}}{
        \lIf{\Structureb<3-4>{$\langle t, j\rangle \ge \mathit{ballot}_i$}}{
          \Structureb<3-4>{$\mathit{ballot}_i \leftarrow \langle t, j \rangle $};
          \Exampleb<3-5>{$\mathit{val}_i \leftarrow v$}%
        }
      }
      \When{\Alertb<1>{\Deliver{rb} $\textsc{decide}(v)$ \From $p_j$}}{
        \Alertb<1,5>{$\mathit{decided}_i \leftarrow v$}\;
      }
    \end{algorithm}
  }

  \obAlertBlock<1>[anchor=north, y=-20mm]{Un algorithme centralisé}{
    \begin{itemize}
    \item La valeur décidée est choisie par un leader
    \end{itemize}
  }

  \obAlertBlock<2>[anchor=north, y=-20mm]{Système de rondes asynchrones}{
    \begin{description}[ballot number :]
    \item[ballot number :] $\langle 0, 1 \rangle ; ... ; \langle 0, n \rangle ; \langle 1, 1 \rangle ; ... ; \langle 1, n \rangle ; \langle 2, 1 \rangle ; ... ; \langle 2, n \rangle ; ...$
    \item[leader :] Seul $p_j$ peut être \emph{leader} pour $\langle \_, j \rangle$
    \end{description}
  }
  
  \obAlertBlock<3>[anchor=north, y=-20mm]{Chaque ronde est divisée en deux phases}{
    \begin{description}[\textsc{voted} :]
    \item[\alert{\textsc{begin} :}] Arrête les rondes précédentes et récupère leur valeur
    \item[\example{\textsc{voted} :}] Transmet la valeur aux rondes suivantes
    \end{description}
  }

  \onAlertBlock<4>[anchor=north, y=-20mm]{Correction : accord (par l'absurde)}{}

  \on<4>[y=-10mm, right=35mm]{\small
    Soient $\langle t_i, i \rangle < \langle t_j, j \rangle$\\
    les deux plus petits \\ \emph{ballot numbers} tels que :
    \begin{itemize}
    \item $p_i$ envoie $\textsc{decide}(v_i)$
    \item $p_j$ envoie $\textsc{decide}(v_j)$
    \end{itemize}
    avec $v_i \neq v_j$. \\
    D'après mutual ordering :
  }

  \on<4>[bottom=-2mm, x=.25\textwidth]{
    \begin{tikzpicture}[anchor=mid,y=5mm, x=8mm]
      \draw[->] (0,1) node[left]{$p_i$} -- (5,1);
      \draw[->] (0,0) node[left]{$p_j$} -- (5,0);
      
      \node[example]   at (1,1) {$\bullet$}; \node[above left, example]   at (1,1) {\scriptsize $\textsc{voted}(t_i, v_i)$};
      \node[alert]     at (1,0) {$\bullet$}; \node[below left, alert]     at (1,0) {\scriptsize $\textsc{begin}(t_j)$};
      \node[structure] at (4,0) {$\bullet$}; \node[above, structure] at (4,0) {\scriptsize $\textsc{decide}(v_i)$};

      \path[-latex,example  ] (1,1) edge[bend left]  (2,1);
      \path[-latex,example  ] (1,1) edge             (2,0);
      \path[-latex,alert    ] (1,0) edge[bend right] (3,0);
      \path[-latex,structure] (4,0) edge +(.5,-.5);
    \end{tikzpicture}
  }

  \on<4>[bottom=-2mm, x=-.25\textwidth]{
    \begin{tikzpicture}[anchor=mid,y=5mm, x=8mm]
      \draw[->] (0,1) node[left]{$p_i$} -- (5,1);
      \draw[->] (0,0) node[left]{$p_j$} -- (5,0);
      
      \node[example]   at (1,1) {$\bullet$}; \node[above left, example]   at (1,1) {\scriptsize $\textsc{voted}(t_i, v_i)$};
      \node[alert]     at (1,0) {$\bullet$}; \node[below left, alert]     at (1,0) {\scriptsize $\textsc{begin}(t_j)$};
      \node[structure] at (4,1) {$x$}; \node[below, structure] at (4,1) {\scriptsize $\textsc{decide}(v_i)$};

      \path[-latex,example  ] (1,1) edge[bend left]  (3,1);
      \path[-latex,alert    ] (1,0) edge             (2,1);
      \path[-latex,alert    ] (1,0) edge[bend right] (2,0);
    \end{tikzpicture}
  }

  \obAlertBlock<5>[anchor=north, y=-20mm]{Correction : validité}{
    \begin{itemize}
    \item\vspace{-1mm} Seules $\bot$ et des valeurs proposées sont écrites dans $\mathit{decided}_i$ et $\mathit{val}_i$
    \item\vspace{-1mm} Seules des valeurs proposées sont envoyées par $\textsc{voted}$ et $\textsc{decide}$
    \item\vspace{-1mm} Lors de la décision, $\mathit{decided}_i\neq\bot$
    \end{itemize}
  }

  \obAlertBlock<6>[anchor=north, y=-20mm]{Correction : terminaison}{
    \begin{itemize}
    \item Si un seul leader correct, il termine sa ronde et envoie $\textsc{decide}(val_i)$
    \item Un jour, un leader correct est élu...
    \end{itemize}
  }
  
\end{frame}

\endgroup
\endinput

% SPDX-License-Identifier: CC-BY-SA-4.0
% Author: Matthieu Perrin
% Part: 
% Section: 
% Sub-section: 
% Frame: 

\begingroup

\begin{frame}{Multi-decrees Paxos en pratique}

  Machine à états répliquée, pertes de messages...

  \begin{block}{Optimisations possibles}
    \vspace{-1mm}
    \begin{itemize}
    \item Message $\textsc{begin}$ mutualisé entre plusieurs rondes
    \item Message $\textsc{msg}$ envoyé au leader seulement dans Multi-decree Paxos
    \item Pas d'abstraction intermédiaire
    \end{itemize}
  \end{block}

  \begin{block}{Rôles des processus}
    \vspace{-1mm}
    Les processus ne jouent pas tous le même rôle :
    \begin{description}
    \item[Proposer :] autorisé à proposer une valeur
      \begin{itemize}
      \item \vspace{-1mm} Serveur en contact des clients
      \end{itemize}
    \item[Leader :] Coordonne les \emph{proposers}
      \begin{itemize}
      \item \vspace{-1mm} Idéalement, un seul processus, correct. 
      \end{itemize}
    \item[Acceptor :] Participent au quorum du \Broadcast{mutual}
      \begin{itemize}
      \item \vspace{-1mm} Une majorité doit être correcte
      \end{itemize}
    \item[Learner :] Destinataire du message $\textsc{decide}(v)$ final
      \begin{itemize}
      \item \vspace{-1mm} processus sur lequel est stocké un réplica
      \end{itemize}
    \end{description}
  \end{block}

  \vfill
  \begin{citing}
  \item[L98] Leslie Lamport. \emph{The part-time parliament.} ACM TOCS (1998)
  \end{citing}

\end{frame}

\endgroup
\endinput


\end{document}

\endinput
